We are one step away from finally beginning to \textit{solve} quantum
mechanical problems using the Schr\"odinger Equation. The final step --
understanding \textit{stationary states}. 

\section{Stationary States}

We begin with the time-dependent Schr\"odinger equation

\begin{align} \label{eq:timedependent}
i\hbar \frac{\partial \Psi(x, t)}{\partial t} = \left[ -\frac{\hbar^2}{2m}
\frac{\partial^2 }{\partial x^2} + V\right] \Psi(x, t)
\end{align}

The $V$ here could, in principle, be a function of both $x$ and $t$. For
example, an electron in a cycling uniform magnetic field whose strength is
being adjusted by an experimentalist. Here we assume this is not the case: V is
assumed to be time-independent, so $V=V(x)$. We then look for a solution of the
form 

\begin{align} \label{eq:dependent}
  \Psi(x, t) = \psi(x)e^{-iEt/\hbar}
\end{align}

The $e^{-iEt/\hbar}$ factor results from solving Equation \ref{eq:timedependent}
via separation of variables. More specifically, the method of searching for
a solution in the form $\Psi(x, t) = X(x)T(t)$ and separating Equation \ref{eq:timedependent} into two
different ordinary differential equations in $x$ and $t$. The resultant $t$
solution is $e^{-iEt/\hbar}$, and is \textbf{always} the same for any time-dependent
wave-function solution. Hence in practice, Equation \ref{eq:dependent} is used, where
$\psi(x) = X(x)$ -- the solution of the time-\textit{independent} Schr\"odinger
equation

\begin{align} \label{eq:independent}
  \left[ -\frac{\hbar^2}{2m} \frac{d^2 }{d x^2} + V(x) \right] \psi(x) \equiv
  \hat{H}(x) \psi(x) = E\psi(x) 
\end{align} \vspace{3px}

Note that Equation \ref{eq:independent}, the time-independent Schr\"odinger
equation is an \textit{eigenvalue equation}, with ordinary derivatives as
$\psi(x)$ just depends on $x$. That is, we have reduced the problem to just
solving an ordinary differential equation in $x$.  In general when we solve this equation, including boundary conditions having to
do with how a particle may be confined, there will exist solutions only for
specific energy eigenvalues $E_i$. That is, our solutions are $\{\psi_i(x)\}$,
with energy eigenvalues $\{E_i\}$. These solutions are called
\textit{stationary states}.

The name comes from the fact that 

\[
|\Psi_i(x, t)|^2 = |\psi_i(x)|^2
\] \vspace{3px}

as the energy-phase $e^{-iEt/\hbar}$ cancels out. Thus for such states the
probability of finding the particle in a region $\delta x$ is stationary -- it
does not evolve in time. These states are also states of definite energy --
(the energy-time uncertainty principle!), given by the energy eigenvalue $E_i$
corresponding to its state $\psi_i(x)$. A system in a stationary state
\textit{stays in that state}, forever. Only such states can have a precise
energy. \\


I will now show that these stationary states actually form a complete
\textit{orthonormal basis} for the time-independent Schr\"odinger equation. In
other words, the final time-independent wave function  $\Psi(x, t)$, can actually be
defined as a linear combination of stationary states $\psi_i(x)$ with their
corresponding energy-phase factor $e^{-iE_i t/\hbar}$ tacked on. 

\subsection{Stationary States as an Orthonormal Basis}

In the very near future we will do several calculations where we find all the
states of the time-dependent Schr\"odinger equation -- the first being the
\textit{infinite square well}. Most examples will be problems where a particle
occupies some region of space and where the ``outcome" of a measurement could
be a specific particle location $x_i$. As there are an infinite number of
outcomes, it should not be surprising that the number of stationary states is
also infinite. These (properly normalized) states $\{\psi_i\}$ with energies
$\{ E_i \}$ are solutions of the time-independent Schr\"odinger equation for
a given potential $V(s)$. I show below that they form an orthonormal basis. We
first show that any two stationary-state wave functions belonging to different
energies are orthogonal -- which requires us to define ``orthogonality" for
wave functions. \\

We have the two stationary state wave function solutions $\psi_1$ and $\psi_2$
that solve 

\begin{align} 
  \left[ -\frac{\hbar^2}{2m} \frac{d^2 }{d x^2} + V(x) \right] \psi_1(x) &\equiv
  \hat{H}(x)\psi_1(x) = E_1\psi_1(x) \label{eq_1} \\ \left[ -\frac{\hbar^2}{2m} \frac{d^2
}{d x^2} + V(x)\right] \psi_2(x) &\equiv \hat{H}(x) \psi_2(x) = E_2\psi_2(x)
\label{eq_2}
\end{align} \vspace{3px}

Multiplying the first equation \ref{eq_1} on both sides by $\psi_2^*(x)$ and the second
equation \ref{eq_2} by $\psi_1^*(x)$, 

\begin{align}
  \psi_2^*(x) \left[ -\frac{\hbar^2}{2m} \frac{d^2 }{d x^2} +V(x) \right]
  \psi_1(x) &= E_1\psi_2^*(x) \psi_1(x) \label{eq_1a} \\ \psi_1^*(x) \left[
  -\frac{\hbar^2}{2m} \frac{d^2 }{d x^2} +V(x) \right] \psi_2(x) &=
  E_2\psi_1^*(x) \psi_2(x) \label{eq_1b}      
\end{align} \vspace{3px}

Taking the conjugate of the second equation \ref{eq_1b}, 

\begin{align} \label{eq_1c}
  \psi_1(x) \left[ -\frac{\hbar^2}{2m} +V(x) \right] \psi_2^*(x) = E_2\psi_1(x)
  \psi_2^*(x)
\end{align}\vspace{3px}

Subtracting Equation \ref{eq_1a} and Equation \ref{eq_1c} and integrating over
all $x$, 

\[
\int_{-\infty}^{\infty} \left[ \psi_2^*\left[-\frac{\hbar^2}{2m} \frac{d^2 }{d
  x^2} + V(x) \right] \psi_1(x) - \psi_1(x) \left[ -\frac{\hbar^2}{2m}
\frac{d^2 }{d x^2} + V(x) \right] \psi_2^*(x) \right] \, dx = (E_1- E_2)
\int_{-\infty}^{\infty} \psi_1(x) \psi_2^*(x) \, dx
\] \vspace{3px}

Notice that the second term on the right can be partially integrated twice to
get 

\[
  \psi_1(x) \left[ -\frac{\hbar^2}{2m} \frac{d^2 }{d x^2} + V(x) \right] \psi_2^*(x)
 \, dx = \psi_2^*(x) \left[ -\frac{\hbar^2}{2m} \frac{d^2 }{d x^2} + V(x)
 \right] \psi_1(x) \, dx
\] \vspace{3px}

which is identical to the first term. So we find 

\begin{align} \label{}
  0 = (E_1 - E_2) \int_{-\infty}^{\infty} \psi_1(x) \psi_2^*(x) \, dx
    \Rightarrow \int_{-\infty}^{\infty} \psi_1(x) \psi_2^*(x) \, dx = 0 \quad
    \text{ if } E_1 \neq E_2
\end{align}\vspace{3px}


The vanishing integral above is what is meant by the orthogonality of two
functions. But there was an exception noted above -- our conclusion of
orthogonality depended on the absence of degeneracy, that $E_1 \neq E_2$. What
if this is not the case? We know the solution from our experience with ordinary
vectors: Gram-Schmidt. If we have two normalized vectors $\vec{u}$ and
$\vec{v}$ that are linearly independent but not orthogonal, we can form a new
orthonormal basis by defining 

\[
\vec{u_1} \equiv \vec{u} \quad \vec{u_2} = \vec{v} - \vec{u} \cdot
\vec{v} \; \vec{u} \quad \text{ so that } \vec{u_1} \cdot \vec{u_2} = \vec{u} \cdot
\vec{v} - \vec{u} \cdot \vec{v} = 0
\] \vspace{3px}

then normalizing $\vec{u_2}$. We can do the same if we have two normalized
functions $\psi_1(x)$ and $\psi_2(x)$ that are not orthogonal, with the same
energy eigenvalue $E$. 

\[
\psi_1(x) \rightarrow \psi_1(x) \quad \psi_2(x) \rightarrow \psi_2(x)
- \psi_1(x) \int_{-\infty}^{\infty} \psi_1^*(x) \psi_2(x) \, dx \equiv
\psi_2'(x)
\] \vspace{3px}

Then

\[
\int_{-\infty}^{\infty} \psi_1^*(x) \psi_2'(x) \, dx = \int_{-\infty}^{\infty}
\psi_1^*(x) \psi_2(x) - \int_{-\infty}^{\infty} \psi_1^*(x) \psi_1(x)
\int_{-\infty}^{\infty} \psi_1^*(x) \psi_2(x) \, dx = 0
\] \vspace{3px}

So if we normalize $\psi_2'(x)$ we then have two orthogonal basis functions
$\psi_1(x), \psi_2'(x)$, and by the principle of superposition, $\psi_2'(x)$ is
also a solution of the time-independent Schr\"odinger equation with energy
eigenvalue $E$. \\

In practice the tedious Gram-Schmidt process is almost never needed: degeneracies usually arise
for a reason known to the quantum mechanic, and she chooses wave function labels that reflect the
physics. As we will discuss later, this amounts to finding operators other than
$\hat{H}$ that commute
with $\hat{H}$. If we find such an operator, we can label our eigenstates by the quantum numbers of
both $\hat{H}$ (energy) and this other operator. These other labels distinguish the degenerate state, and
guarantee their orthogonality.\\

This past derivation was heavy in content; before moving on, you should be fully clear on how
orthogonality is defined and how stationary states of different energies are
orthogonal. 

\subsection{Stationary States Form a Complete Basis}

The basis just formed above, $\{E_i, \psi_i(x), I = 1, \cdots , \infty\}$ is
a \textit{complete orthonormal basis} for the time-independent Schr\"odinger
equation. I'll not provide a general proof, but we will encounter bases that
soon you will recognize as complete, such as the Fourier series. \\

Normalized stationary-state solutions form an orthonormal basis: 

\[
\boxed{ \int_{-\infty}^{\infty} \psi_i^*(x) \psi_i(x) = 1 \qquad
\int_{-\infty}^{\infty} \psi_j^*(x)\psi_i(x) \, dx = 0, \; i\neq j }
\] \vspace{3px}

Consequently any general function in the space can be expanded in terms of the
basis of stationary states, with coefficients that follow from the
orthogonality condition. 

\begin{mainbox}{Expansion of time-independent arbitrary wave function $\Psi(x)$ in terms of
  stationary states}
  \begin{align} \label{ci}
    \Psi(x) = \sum_{i=1}^\infty c_i \psi_i(x) \qquad c_i
    = \int_{-\infty}^{\infty} \psi_i^*(x) \Psi(x) \, dx
  \end{align} 
\end{mainbox}

\section{The ``Prime Directive"}

If you were going to choose to understand any one portion of this book -- choose this
one. We finally come to a result so important that we can dub it, in homage to
Star Trek, the \textit{prime directive}. 

Suppose some experimentalist has started up some experiment at time $t_0$ that
is governed by quantum mechanics -- perhaps some interesting wave packet
$\Psi(x, t_0)$ that is arbitrary, not corresponding to any one of the
stationary states. This wave packet might describe the possible position
outcomes for a particle, should we interrogate it at time $t_0$. The wave
function would be normalized -- $\int_{-\infty}^{\infty}  |\Psi(x, t_0)|^2 \,
dx = 1$. \\

Because the stationary states form a complete set, we know 

\begin{align} \label{eq_5}
\Psi(x, t_0) = \sum_i c_i \psi_i(x)
\end{align} \vspace{3px}

But the wave packet is normalized and the stationary states are an orthonormal
set. So 

\[
  1 = \int_{-\infty}^{\infty}  |\Psi(x, t_0)|^2 \, dx = \sum_{j=1}^\infty
  \sum_{i=1}^\infty c_j^* c_i \int_{-\infty}^{\infty} \psi_j^*(x) \psi_i(x) \,
  dx = \sum_{j=1}^\infty \sum_{i=1}^\infty c_j^* c_i \delta_{ji}
  = \sum_{i=1}^\infty |c_i|^2
\] \vspace{3px}

Do not get confused by all the math above, I just substitute in Equation
\ref{eq_5} to ultimately show that $|c_i|^2$ is the initial probability of
being in the $i$th stationary state. To simplify notation let's set our clock
to start at $t_0 = 0$. Then consider the wave function 

\[ \Psi(x, t) = \sum_i c_i \psi_i(x) e^{-iE_i t/\hbar}, t > 0 \] 


Plugging this into the time-dependent Schr\"odinger equation (superposition
principle) yields 

\[
  \sum_i E_i c_i \psi_i(x) e^{-iE_i t/\hbar} = \sum_i E_i c_i \psi_i(x)
  e^{-iE_i t\hbar}
\] \vspace{3px}

So we have a solution! Consequently we have what you might call the
\textit{prime directive} of quantum mechanics: 

\begin{subbox}{The Prime Directive}
  
    The prime directive: Let $\{\psi_i(x)\}$ and $\{E_i\}$ denote the
    complete set of stationary-state solutions and eigenvalues, that is 

    \[
    \left[ -\frac{\hbar^2}{2m} \frac{d^2 }{d x^2} + V(x) \right] \psi_i(x)
    = E_i \psi_i(x) 
  \] \vspace{3px}

  Given a wave packet at $t=0, \Psi(x, t = 0) = \sum_i c_i \psi_i(x)$,
    then the solution of the full time-dependent Schr\"odinger equation is
    \begin{align}\label{primedirec}
    \Psi(x, t) = \sum_i c_i \psi_i(x) e^{-iE_i t/\hbar}
  \end{align}
\end{subbox}

This is such a powerful result. It implies that apart from the special case of a pure stationary state,
the stationary components of wave functions propagate with different phases,
interfering in a time-dependent way. Thus the probability at some point $x$,
$|\Psi(x, t)|^2$, is not fixed -- not stationary --
but instead varies in time. But it also states the probabilities $|c_i|^2$ do
not evolve in time -- all of the weird quantum physics comes from time-varying interference.\\

Hence, according to our \textit{prime directive}, to solve for the time
dependent wave function of any particle or wave packet, we should 

\begin{itemize}
  \item[1.] Find the stationary state and their eigenvalues via the
    time-independent Schr\"odinger Equation \ref{eq:independent}
  \item[2.] Solve for the $c_i$'s using an initial starting wave packet via
    Equation \ref{ci}
  \item[3.] Solve for the full time-dependent $\Psi(x, t)$ by summing over
    Equation \ref{primedirec}
\end{itemize} 

So now we roll up our sleeves and start doing quantum mechanics. 
