We will now implement the prime directive for several different potentials, the
first of which is the infinite square well. 


\section{The Infinite Square Well}

A particle of mass $m$ is confined to a region of width $a$, $-a/2 < x < a/2$,
by the potential 

 \[
V(x) = \begin{cases}
  0 &\qquad |x| < a/2 \\
  \infty &\qquad \text{otherwise} 
\end{cases} 
\] \vspace{3px}

\begin{figure}[H]
  \centering
    \includegraphics[width = 6.8cm]{infinitesquarewell.pdf}
    \caption{The Infinite Square Well}
\end{figure}

Aligning ourselves with the prime directive, we look for solutions of Schr\"odinger equation of the
form 

\[
  \Psi(x, t) = \sum_i \psi(x)e^{-i E_i t/\hbar} \qquad \left[
  -\frac{\hbar^2}{2m} \frac{d^2 }{d x^2} + V(x) \right] \psi_i(x) = E_i
  \psi_i(x) 
\] \vspace{3px}

The solution outside the well is $\psi_i(x) = 0$. This will be demonstrated
later in the chapter by solving the finite square well, then taking the
infinite well limit. But intuitively it makes sense, since as the outside
potential is infinite, the wave function can not penetrate past the boundary at
all. \\

Inside the well we solve 

\[
-\frac{\hbar^2}{2m} \frac{d^2 }{d x^2} \psi_i(x) = E_i \psi_i(x) \quad
\Rightarrow \quad \frac{d^2 }{d x^2} \psi_i(x) = -k_i^2 \psi_i(x)
\] \vspace{3px}   


where $k_i = \frac{\sqrt{2mE_i}}{\hbar}$. This differential equation has the
general solution (verify) of 

\[
\psi_i(x) = A\sin k_ix + B\cos k_i x
\] \vspace{3px}

We now deal with the boundary conditions. The wave functions must vanish at
$|x| = a/2$, as it vanishes for all  $|x| > \frac{a}{2}$, and the wave function
must also be continuous. This requires 


\[
  \lim_{x \to \pm a} \psi_i(x) = 0
\] \vspace{3px}

Substituting in for $\psi_i(x)$, and evaluating the limit, we get

\begin{align}
&A\sin \frac{k_i a}{2} + B\cos \frac{k_i a}{2} = 0 \label{first}\\
&A\sin \left( -\frac{k_ia}{2} \right) + B\cos \left( -\frac{k_i a}{2} \right)
= -A \sin \frac{k_i a}{2} + B\cos \frac{k_i a}{2} = 0 \label{second} 
\end{align} \vspace{3px}

Adding and subtracting Equations \ref{first} and \ref{second},

\begin{align} \label{}
  &A\sin \frac{k_i a}{2} + B\cos \frac{k_i a}{2} + \left( - A \sin \frac{k_i
  a}{2} + B\cos \frac{k_i a}{2}\right)  \quad \Rightarrow \quad B\cos \frac{k_i
a}{2} = 0 \\
  &A\sin \frac{k_i a}{2} + B\cos \frac{k_i a}{2} - \left( - A \sin \frac{k_i
  a}{2} + B\cos \frac{k_i a}{2} \right) \quad \Rightarrow \quad A \sin
  \frac{k_i a}{2} = 0 
\end{align}\vspace{3px}

Therefore we yield two sets of solutions, 

\begin{align} \label{}
  \psi_n(x) = \begin{cases}
    B \cos k_n x &\quad k_n = \{ \frac{\pi}{a}, \frac{3\pi}{a}, \cdots \}
    = \frac{\pi n }{a}, \;  n = 1, 3, 5, \cdots \qquad \text{ even parity } \\
    A \sin k_n x &\quad k_n = \{ \frac{2\pi}{a}, \frac{4\pi}{a}, \cdots \}
    = \frac{\pi n}{a}, \;  n = 2, 4, 6, \cdots \qquad \text{odd parity} 
  \end{cases} 
\end{align}\vspace{3px}

We replace $i$ with $n$, the quantum number that ``labels" our solutions. The
even parity solutions for odd $n$ occur when the stationary state wave function is symmetric
about the origin. The odd parity solutions are antisymmetric. Now, knowing that 

\[
k_n = \frac{\sqrt{2mE_n}}{\hbar} \Rightarrow E_n = \frac{\hbar^2 k_n^2}{2m}
= \frac{\hbar^2 n^2 \pi^2}{2ma^2}
\] \vspace{3px}

We get the allowed energy eigenvalues that correspond to the stationary states
labeled with quantum number $n$. Now we move onto normalization to determine
the coefficients $A$ and $B$. Using Equation \ref{eq:normalization}, 

\begin{align} \label{}
  &\int_{-a/2}^{a/2} |\psi_{i, \text{even parity}} (x)|^2 \, dx = \int_{-a/2}^{a/2} B^2 \cos^2
  \frac{\pi n x}{a} \, dx = B^2 \frac{a}{2} = 1 \quad \Rightarrow \quad
  B = \sqrt{\frac{2}{a}} \\ 
  &\int_{-a/2}^{a/2} |\psi_{i, \text{ odd parity}} (x) |^2  \, dx
  = \int_{-a/2}^{a/2} A^2 \sin^2 \frac{\pi n x}{a} \, dx = A^2 \frac{a}{2}
  = 1 \quad \Rightarrow \quad A = \sqrt{\frac{2}{a}}
\end{align}\vspace{3px}

Therefore, our normalized stationary state solutions, \\

\begin{mainbox}{Infinite Square Well Stationary States}
  \[ \psi_n(x) = \begin{cases}
    & \sqrt{\frac{2}{a}}\cos \frac{\pi n x}{a}, \; n = 1,3,5, \cdots \quad
    \text{ even parity} \\ & \sqrt{\frac{2}{a}} \sin \frac{\pi n x}{a}, \;
    n = 2, 4, 6, \cdots \quad \text{ odd parity} 
  \end{cases}  \] \[ E_n = \frac{\hbar^2 k_n^2}{2m} = \frac{\hbar^2 n^2
\pi^2}{2ma^2} \]
\end{mainbox} \vspace{5px}

I plot the first four stationary state solutions along with their defined energies
in Figure \ref{fig:infsquarewell}.
\begin{figure}[!ht]
  \centering
    \includegraphics[width = 11cm]{infinitesquarewellsolutions.pdf}
    \caption{Infinite Square Well Stationary States}
    \label{fig:infsquarewell}
\end{figure}
\subsection{Various Comments about the Solutions}

\begin{itemize}
  \item[1.] There are an infinite number of allowed energy eigenvalues and
    eigenfunctions labeled by the discrete quantum number index $n$. This
    is a consequence of the boundary condition that $\psi(x)$ vanish at $|x|
    < a/2$, limiting solutions to integer and half-integer wavelengths. 
  \item[2.] The basis is orthonormal as  \[ \int_{-a/2}^{a/2} \psi_n'^*(x)
    \psi_n(x) \, dx = \delta_n'n \] can be readily verified. 
  \item[3.] The basis is complete for any function $\Psi(x)$ satisfying the
    boundary condition $\Psi(a/2) = \Psi(-a/2) = 0$ and defined on the interval
    $[-a/2, a/2]$. Any function satisfying these conditions can be expanded in
    this basis. Those of you familiar with Fourier Series may have noticed that
    the odd basis functions \[ \sqrt{\frac{2}{a}}\sin \left( \frac{\pi n x}{a}
      \right), \; n = 2, 4, 6, \cdots \; \text{ which can be written as
      } \; \sqrt{\frac{2}{a}}\sin \left( \frac{\pi n' x}{a/2} \right), \; n' = 1, 2,
    3, \cdots \] \vspace{3px} are the standard odd functions of a Fourier Series, while the
    even basis functions \[ \sqrt{\frac{2}{a}}\cos \frac{\pi n x}{a}, \; n = 1, 3,
      5, \cdots \text{ which can be written as } \; \sqrt{\frac{2}{a}}\cos \left(
    \frac{\pi (n' - \frac{1}{2})x}{a/2}\right), \; n' = 1, 2, 3, \cdots \] \vspace{5px} have
    been shifted in index, and the constant term  $(n' = 0)$ is absent. These
    modifications reflect restrictions imposed on the basis by our use of
    specific boundary conditions.  
  \item[4.] These wave functions have  $n-1$ interior zeroes -- coordinates at
    which the probability to find the trapped particle vanishes. 
  \item[5.] The eigenfunctions have alternating definite parity -- even or odd
    -- a consequence of the reflection symmetry of the potential.
\end{itemize}

\subsection{Wave Function Curvature}

Within the well interior the particle propagates as a free particle --
a particle in the absence of any confining potential. One can understand the
physics of our well solutions from the correlation between wave function
curvature and momentum. The momentum operator \textit{measures} curvature, and
our energy is quadratic in $p$. This can be made explicit by evaluating the
expectation value of $\langle p^2 \rangle$ between the stationary states

\[
  \langle \hat{p}^2 \rangle = \frac{\hbar^2 \pi^2 n^2}{a^2} \quad \text{ so
  that }  \quad \frac{1}{2m}\langle \hat{p}^2 \rangle = \langle \hat{H} \rangle 
\] \vspace{3px}
For a fixed number of nodes, doubling $a$ will half the curvature. Therefore,
energies depend \textit{inversely} on $a^2$. Conversely doubling the number of
nodes doubles the curvature. Consequently energies scale as $n^2$. 

The square well is exceptional in that it confines all wave functions in the
same way. This leads to the steep $n^2$ dependence of energy eigenvalues. In
a finite well -- a future exercise -- where the boundaries are not infinitely
strong -- the wave functions at higher excitation energies \textit{penetrate}
into the classically-forbidden region of the potential, reducing the curvature
and thus producing energies that increase less steeply than $n^2$. This is
known as \textit{\textbf{quantum tunneling}}. In the harmonic oscillator --
another future exercise -- the widening $r^2$ potential leads to a spectrum
that is evenly spaced, with eigenvalues rising with $n$. 

\subsection{Example Problem -- The Prime Directive}

We calculated the stationary states for the infinite square well above. But
what is the full, time-dependent normalized wave function? How do you find the
$c_i's$ in Equation \ref{primedirec}?. \\

To determine these, we have to have an additional boundary condition, namely
the wave function value at some time  $t$, call it $t=0$. 

\paragraph{Example} A particle in an infinite square well has the initial wave
function shown in Figure \ref{fig:prob}.

\begin{align} \label{theproblem}
  \Psi(x, 0) = \begin{cases}
    Ax(a-x) &\qquad 0 \leq x \leq a \\ 0 &\qquad x > 0
  \end{cases} 
\end{align}\vspace{3px}

for some constant $A$. Find $\Psi(x, t)$. 

\begin{figure}[htp]
  \centering
    \includegraphics[width = 12cm]{infinitesquarewell_problem.pdf}
    \caption{$\Psi(x, 0)$}
    \label{fig:prob}
\end{figure}

First we must determine $A$ using the normalization condition. If $\Psi(x,
0)$ is normalized, $\Psi(x, t)$ will stay normalized. 

\begin{align} \label{}
  \int_{0}^{a} |\Psi(x, 0)|^2 \, dx = |A|^2\int_{0}^{a} x^2(a-x)^2 \, dx
  &= |A|^2 \int_{0}^{a} (a^2x^2 - 2ax^3 + x^4) \, dx \\ &= |A|^2\left( a^2
  \frac{x^3}{3} - 2a \frac{x^4}{4} + \frac{x^5}{5} \right) \Bigg|_0^a = |A|^2
  \left( \frac{a^5}{3} - \frac{2a^5}{4} + \frac{a^5}{5} \right) = |A|^2
  \frac{a^5}{30} = 1 \\ &\Rightarrow |A| = \sqrt{\frac{30}{a^5}} 
\end{align}\vspace{3px}

Now we can determine the coefficients $c_n$ using Equation \ref{primedirec}. 

\begin{align*}
  \Psi(x, 0) &= \sum_{n=1}^{\infty} c_n \psi_n(x) \\
  c_n &= \int \psi_n(x) \Psi(x, 0) \, dx \\
      &= \sqrt{\frac{2}{a}}\int_{0}^{a} \sin\left( \frac{n\pi}{a}x \right)
      \sqrt{\frac{30}{a^5}}x(a-x)\,dx \\
      &= \frac{2\sqrt{15}}{a^3}\left[a\int_{0}^{a} x\sin\left( \frac{n\pi}{a}x
      \right) \, dx - \int_{0}^{a} x^2 \sin \left( \frac{n\pi}{a}x \right) \,
dx\right] \\ &=\frac{2\sqrt{15}}{a^3} \left[ a \left[ \left( \frac{a}{n\pi}
    \right) ^2 \sin \left( \frac{n\pi}{a}x \right) - \frac{ax}{n\pi}\cos \left(
    \frac{n\pi}{a}x\right) \right]_0^a - \left[ 2 \left( \frac{a}{n\pi} \right)
  ^2 x \sin \left( \frac{n\pi}{a}x \right) - \frac{(n\pi x a)^2 - 2}{(n\pi
a)^3}\cos \left( \frac{n\pi}{a}x \right)  \right]_0^a \right]     \\
             &= \frac{2\sqrt{15}}{a^3} \left[ -\frac{a^3}{n\pi}\cos(n\pi)
               + a^3\frac{(n\pi)^2 - 2}{(n\pi)^3}\cos(n\pi) + a^3
             \frac{2}{(n\pi)^3}\cos(0)\right] \\
             &= \frac{4\sqrt{15}}{(n\pi)^3}[\cos(0) - \cos(n\pi)]\\
             &=\begin{cases}
             0 &\qquad \text{if $n$ even} \\
               \frac{8\sqrt{15}}{(n\pi)^3} &\qquad \text{if $n$ odd}
             \end{cases}
\end{align*}

So now we can form our time-dependent wave function using the prime directive by putting everything
together. We get 

\begin{align}
  \Psi(x, t) &= \sum_{n=1}^\infty c_n \psi_n(x) e^{-iE_n t / \hbar} \\ 
  \Psi(x, t) &= \sqrt{\frac{30}{a}} \left( \frac{2}{\pi} \right) ^3 \sum_{n
= 1,3,5,\cdots} \frac{1}{n^3}\sin \left( \frac{n\pi x}{a} \right) e^{-in^2\pi^2
\hbar t / 2ma^2} \label{final}
\end{align} \vspace{3px}

Notice the solution only contains the $\sin \left( \frac{n\pi x}{a} \right)$
component. This is because the square well in this problem is
defined from $0 < x < a$ and not reflection symmetric around the origin. The final
solution in Equation \ref{final} describes how the wave function changes as
a function of time, provided the particle is in a square well. \\

Now let us study this wave function, and its implications on observables. We
first determine what the expectation value of the energy is. 

\begin{align}
  \langle E \rangle &= \int \Psi(x, t)^* \hat{H} \Psi(x, t) \, dx \\ 
                    &= \int \Psi(x, t)^* E_n \Psi(x, t)\, dx
\end{align}

Performing the calculation, one will find 

\[
\langle E \rangle = \sum_n |c_n|^2 E_n
\] \vspace{3px}

So we can think of $|c_n|^2$ as the probability to measure the energy
eigenvalue $E_n$. Therefore, 

\[
\sum_m |c_n|^2 = 1
\] \vspace{3px}

The $c_n^2$ can also be thought of as telling us the ``amount" of $\Psi_n$ that
is in the \textit{total} wave function. In the example above, we can see that
the initial wave function closely resembles $\Psi_1$. If we look at $c_1$, 

\[
|c_1|^2 = \left( \frac{8\sqrt{15}}{\pi^3} \right) ^2 = 0.998555... 
\] \vspace{3px}

we see that it is very close to $1$, indicating that the $n=1$ state dominates.
\\

The Infinite Square Well is the classic introductory quantum mechanics problem.
Before moving on, ensure you understand every derivation in this section
\textit{completely.} 

\section{The Harmonic Oscillator}

This next section is \textit{very} important. In nature, ``friggin' everything is
a harmonic oscillator" -- Reddit commenter. The harmonic oscillator stationary-state basis is
arguably the most versatile and important in physics. Every field in physics
includes key problems that require one to understand small-amplitude behavior
that maps onto the harmonic oscillator. 

The \textit{quantum} 1D harmonic oscillator is to solve the Schr\"odinger
equation for the potential shown in Figure \ref{fig:harmonicoscillator}

\begin{align} \label{HOpotential}
  V(x) = \frac{1}{2}m\omega^2 x^2
\end{align}\vspace{3px}

\begin{figure}[!ht]
  \centering
    \includegraphics[width = 8cm]{harmonicoscillator.pdf}
    \caption{Quantum Harmonic Oscillator Potential}
    \label{fig:harmonicoscillator}
\end{figure}

Therefore, by the prime directive, the time-independent Schr\"odinger equation
reads

\begin{align} \label{HOschr}
  -\frac{h^2}{2m} \frac{d^2 \psi}{d x^2} + \frac{1}{2}m\omega^2 x^2 \psi
  = E\psi
\end{align}\vspace{3px}

There are two methods to solve this problem. The first is the ``brute force"
attempt to solve the differential equation using power series. The second,
according to Griffiths, is a ``diabolically clever" technique using
\textit{ladder operators}. We will start with the ladder operator technique. 

\subsection{Ladder Operator Algebraic Technique}

We rewrite Equation \ref{HOschr} utilizing the momentum operator: 

\begin{align} \label{}
  \frac{1}{2m}\left[ \hat{p}^2 + (m\omega w)^2 \right] \psi = E\psi
\end{align}\vspace{3px}

where $\hat{p}\equiv -i\hbar \frac{d}{dx}$ is the momentum operator. We define
the ladder operators as the following: 

\begin{align} \label{}
  \hat{a}_\pm \equiv \frac{1}{\sqrt{2\hbar m\omega}}(\mp i \hat{p} + m\omega x)
\end{align}\vspace{3px}

You might be wondering why the following operators are known as \textit{ladder}
operators -- we'll get to that in a bit, but for now, let us determine the
product $\hat{a}_- \hat{a}_+$. 

\begin{align} \label{}
  \hat{a}_- \hat{a}_+ &= \frac{1}{2\hbar m \omega} (i \hat{p} + m\omega x)(-i
  \hat{p} + m\omega x) \\ &= \frac{1}{2\hbar m \omega} \left[ \hat{p}^2
  + (m\omega x)^2 - im\omega (x \hat{p} - \hat{p} x) \right]
\end{align}\vspace{3px} 

Note above, we do not combine the terms $(m\omega x)(-i \hat{p})$ and $(i
\hat{p})(m\omega x)$. This is because in we are dealing with operators.
Operators do not, in general, \textbf{commute} ($x \hat{p} \neq \hat{p}x$ ). So
we have to separate the two. As a result, there is an extra term involving ( $x
\hat{p} - \hat{p} x$ ). We call this the \textbf{commutator} of $x$ and
$\hat{p}$. In general, the commutator of operators $\hat{A}$ and $\hat{B}$ is 

\begin{align} \label{commutator}
  [\hat{A}, \hat{B}] \equiv \hat{A}\hat{B} - \hat{B}\hat{A}
\end{align}\vspace{3px}

Using this notation, 

\begin{align} \label{a-a+}
  \hat{a}_- \hat{a}_+ = \frac{1}{2\hbar m \omega}\left[ \hat{p}^2 + (m\omega
  x)^2 \right] - \frac{i}{2\hbar}[x, \hat{p}]
\end{align}\vspace{3px}

And so we need to figure out the commutator of $x$ and $\hat{p}$. To do this,
we employ an arbitrary ``test`` function $f$ to see what the \textit{effect} of
the commutator is. At the end we can then throw away $f$ to determine the value
of the commutator. We have 

\begin{align} \label{}
  [x, \hat{p}] f(x) &= \left[ x(-i\hbar) \frac{d }{d x} f(x) - (-i\hbar) \frac{d
  }{d x} (xf) \right] \\ &= -i\hbar \left( x \frac{d f}{d x} - x \frac{d f}{d x}
- f\right) \\ &= i\hbar f(x)
\end{align}\vspace{3px}   

Therefore, $[x, \hat{p}] = i\hbar$. This formula is known as the \textbf{canonical
commutation relation}. With this, Equation \ref{a-a+} becomes 

\begin{align} \label{}
  \hat{a}_- \hat{a}_+ &= \frac{1}{\hbar \omega} \hat{H} + \frac{1}{2}  \\
                      &\Rightarrow \hat{H} = \hbar \omega \left( \hat{a}_-
                      \hat{a}_+ - \frac{1}{2} \right)      
\end{align}\vspace{3px}

Notice that the ordering of $\hat{a}_+$ and $\hat{a}_-$ is important here; the
same argument with $\hat{a}_+$ on the left, yields 

\begin{align} \label{hami}
  \hat{a}_+ \hat{a}_- = \frac{1}{\hbar \omega} \hat{H} - \frac{1}{2}
\end{align}\vspace{3px}   

Therefore, another expression for the Hamiltonian can be derived by rearranging
Equation \ref{hami}. 

\begin{align} \label{}
  \hat{H} = \hbar \omega \left(\hat{a}_+ \hat{a}_- + \frac{1}{2}\right)
\end{align}\vspace{3px}


Then, in terms of $\hat{a}_\pm$, the Schr\"odinger equation for the harmonic
oscillator can be written as 

\begin{align} \label{intermsof}
  \hbar\omega \left( \hat{a}_\pm \hat{a}_\mp \pm \frac{1}{2} \right) \psi = E\psi 
\end{align}\vspace{3px}

Now comes the pinnacle of the ladder operator method -- the reason it is
``diabolically clever." \\

\begin{mainbox}{Energy Eigenvalues of Ladder Operator}
If $\psi$ satisfies the Schr\"odinger equation with energy $E$
(that is, $\hat{H}\psi = E\psi$), then $a_+ \psi$ satisfies the Schr\"odinger
equation with energy $(E+\hbar \omega)$ (that is, $\hat{H}(\hat{a}_+ \psi)
= (E+\hbar \omega)(\hat{a}_+\psi)$.
\end{mainbox} \vspace{3px}

Because this is so important and may seem like it came out of the blue,
I provide a proof. 

\begin{align}
  \hat{H}(\hat{a}_+\psi) &= \hbar\omega\left(\hat{a}_+\hat{a}_-
  + \frac{1}{2}\right)(\hat{a}_+\psi) = \hbar\omega \left(
\hat{a}_+\hat{a}_-\hat{a}_+ + \frac{1}{2}\hat{a}_+ \right) \psi \\ 
                         &= \hbar \omega \hat{a}_+ \left( \hat{a}_-\hat{a}_+
                         + \frac{1}{2} \right) \psi = \hat{a}_+ \left[
                         \hbar\omega \left( \hat{a}_+ \hat{a}_-
                       + 1 + \frac{1}{2} \right) \psi\right] \\
      &= \hat{a}_+(\hat{H} + \hbar \omega) \psi = \hat{a}_+(E+\hbar\omega)\psi
      = (E +\hbar\omega)(\hat{a}_+\psi). \qquad \text{QED}.
  \end{align} \vspace{3px}

Note that in the second line I replaced $\hat{a}_-\hat{a}_+$ by
$(\hat{a}_+\hat{a}_- + 1)$. I can do this because (verify) $[\hat{a}_-,
\hat{a}_+] = 1$. By the same procedure, $\hat{a}_-\psi$ is a solution as well,
with energy eigenvalue $(E - \hbar\omega)$. 

\begin{align} \label{}
  \hat{H}(\hat{a}_-\psi) &= \hbar\omega \left( \hat{a}_-\hat{a}_+ - \frac{1}{2}
  \right) (\hat{a}_-\psi) = \hbar\omega \hat{a}_- \left( \hat{a}_+ \hat{a}_-
  - \frac{1}{2} \right) \psi \\ &= \hat{a}_-\left[ \hbar\omega \left(
\hat{a}_-\hat{a}_+ - 1 -\frac{1}{2} \right) \psi\right] = \hat{a}_- \left(
  \hat{H} - \hbar\omega \right) \psi = \hat{a}_-(E-\hbar\omega)\psi \\ &=
  (E-\hbar\omega)(\hat{a}_-\psi).   
\end{align}\vspace{3px}


This is why we call $\hat{a}_\pm$ ladder operators. If we could just find
\textit{one} solution, we can use $\hat{a}_\pm$ to ``climb up and down" in
energy, getting \textit{all} the possible energy eigenvalues. Hence, we call
$\hat{a}_+$ the \textbf{raising operator} and $\hat{a}_-$ the \textbf{lowering
operator}. 

Let us try to find the \textit{lowest rung}, $\psi_0$, such that
$\hat{a}_-\psi_0 = 0$. Therefore, 

\begin{align} \label{}
  \frac{1}{\sqrt{2\hbar m \omega}} \left( \hbar \frac{d }{d x} + m\omega
  x \right) \psi_0 = 0 
\end{align}\vspace{3px}

Rearranging gives 

\begin{align} \label{}
  \frac{d \psi_0}{d x} = -\frac{m\omega}{\hbar}x\psi_0
\end{align}\vspace{3px}

Implementing separation of variables, 

\begin{align} \label{}
  \int \frac{1}{\psi_0} \, d\psi_0 = -\frac{m\omega}{\hbar} \int x \, dx \quad
  \Rightarrow \quad \ln \psi_0 = -\frac{m\omega}{2\hbar} x^2 + \text{const.} 
\end{align}\vspace{3px}

Hence, 

\begin{align} \label{}
  \psi_0(x) = A e^{-\frac{m\omega}{2\hbar}x^2}
\end{align}\vspace{3px}

Normalizing, 

\begin{align} \label{}
  1 = |A|^2 \int_{-\infty}^{\infty} e^{-m\omega x^2 / \hbar} \, dx
    = |A|\sqrt{\frac{\pi \hbar}{m\omega}} \quad \Rightarrow \quad A^2
    = \sqrt{m\omega / \pi \hbar}
\end{align}\vspace{3px}

Therefore, we get the final ``lowest-rung" stationary state 

\begin{align} \label{lowestrung}
  \psi_0(x) = \left( \frac{m\omega}{\pi \hbar} \right) ^{1/4}
  e^{-\frac{m\omega}{2\hbar}x^2} 
\end{align}\vspace{3px}

To determine the energy of this state we plug it into the Schr\"odinger
Equation \ref{intermsof} and exploit the fact that $\hat{a}_-\psi_0 = 0$. ]


\begin{align} \label{}
  \hbar\omega\left(\hat{a}_+\hat{a}_- + \frac{1}{2}\right)\psi_0 &= E_0\psi_0\\
  \hbar\omega\left(\hat{a}_+(\hat{a}_-\psi_0) + \frac{1}{2}\psi_0\right) &=
  E_0\psi_0 \\ \hbar\omega \left( 0  + \frac{1}{2}\psi_0\right)  = E_0\psi_0
\end{align}\vspace{3px}

Therefore, $E_0 = \frac{1}{2}\hbar\omega$. We have secured the \textbf{ground
state} of the quantum harmonic oscillator. We now can just apply $\hat{a}_+$
repeatedly to generate the excited states, increasing the energy by
$\hbar\omega$ each step as we go along. Therefore, every stationary state,
labeled by a quantum number $n$ can be defined as follows:

\begin{align} \label{harmenergy}
   \psi_n(x) = A_n(\hat{a}_+)^n \psi_0(x), \qquad E_n = \left(
  n+\frac{1}{2} \right) \hbar\omega 
\end{align}\vspace{3px}

where $A_n$ is the normalization constant. You can get $A_n$ algebraically,
however, and the proof is laid out in Griffiths. It turns out that $A_n
= \frac{1}{\sqrt{n!}}$. Thus, 

\begin{align} \label{laddersolution}
  \psi_n &= \frac{1}{\sqrt{n!}}(\hat{a}_+)^n\psi_0 
\end{align}
 \[ \boxed{
  \psi_n = \frac{1}{\sqrt{n!}} (\hat{a}_+)^n \left( \frac{m\omega}{\pi\hbar}
\right)^{\frac{1}{4}} e^{-\frac{m\omega}{2\hbar}x^2} }
  \] \vspace{3px}

\subsection{Example -- Expectation Value of V(x)}

\paragraph{Example} Find the expectation value of the potential energy in the
$n $th stationary state of the harmonic oscillator. 

\[
\langle V \rangle = \langle \frac{1}{2}m\omega^2 x^2\rangle
= \frac{1}{2}m\omega^2 \int_{-\infty}^{\infty} \psi_n^* x^2 \psi_n \, dx
\] \vspace{3px}

Expressing $x$ and $\hat{p}$ in terms of the raising and lowering operators, 

\[ \boxed{ x = \sqrt{\frac{\hbar}{2m\omega}} (\hat{a}_+ + \hat{a}_-); \qquad
\hat{p} = i\frac{\hbar m \omega}{2} (\hat{a}_+ - \hat{a}_-) } \]\vspace{3px}

Therefore, 

\[
x^2 = \frac{\hbar}{2m\omega} \left[ (\hat{a}_+^2 + \hat{a}_+\hat{a}_-
+ \hat{a}_-\hat{a}_+ + \hat{a}_-^2 \right] 
\] \vspace{3px}

Hence, 

\[
\langle V \rangle = \frac{\hbar\omega}{4}\int_{-\infty}^{\infty}  \psi_n^*
\left[ (\hat{a}_+^2 + (\hat{a}_+\hat{a}_-) + (\hat{a}_-\hat{a}_+)
+ \hat{a}_-^2\right] \psi_n \, dx
\] \vspace{3px}

But $\hat{a}_+^2\psi_n$ is (apart from normalization), just $\psi_{n+2}$, which
is orthogonal to $\psi_n$ and the same goes for $\hat{a}_-^2\psi_n \sim
\psi_{n-2}$. So those terms drop out, and we are left with 

\[
\langle V \rangle = \frac{\hbar\omega}{4}\int_{-\infty}^{\infty}  \psi_n^*
\left[ (\hat{a}_+\hat{a}_-) + (\hat{a}_-\hat{a}_+) \right] \psi_n \, dx
\] \vspace{3px}

Now, without proof (but check Griffiths) I will use the following relations: 

\[
\hat{a}_+\hat{a}_- \psi_n = n\psi_n, \qquad \hat{a}_-\hat{a}_+\psi_n
= (n+1)\psi_n
\] \vspace{3px}

Therefore, we get the final result, 

\[
\langle V \rangle = \frac{\hbar \omega}{4} (n + n + 1) = \frac{1}{2}\hbar\omega
(n + \frac{1}{2}). 
\] \vspace{3px}

We see that the expectation value of the potential energy $V(x)$ is exactly
\textit{half} of the total energy $E = \left(n+\frac{1}{2}\right)\hbar\omega$.
This is a beautiful fact of the harmonic oscillator and a reason as to why it
shows up literally everywhere in physics. 

\subsection{Power Series Analytic Method}

I will now show the second, `brute-force' power series method to solve the
Schr\"odinger equation for the harmonic oscillator, 

\begin{align} \label{1Dharmonic}
  -\frac{\hbar^2}{2m} \frac{d^2 \psi}{d x^2} + \frac{1}{2}m\omega^2x^2\psi
  = E\psi
\end{align}\vspace{3px}

and solve it directly. We first introduce a dimensionless variable $\xi$ to
make things cleaner, where 

\begin{align} \label{xi}
\xi \equiv \sqrt{\frac{m\omega}{\hbar}}x
\end{align} \vspace{3px}

Hence, the Schr\"odinger equation now reads 

\begin{align} \label{1dharmonic}
  \frac{d^2 \psi}{d \xi^2} = (\xi^2 - K)\psi
\end{align}\vspace{3px}   

where $K \equiv \frac{2E}{\hbar\omega}$. Our problem is to solve Equation
\ref{1dharmonic} and in the process obtain the allowed values of $K$, and thus
$E$. To begin with, we notice that at very large $\xi$, (very large $x$ ),
$\xi^2$ completely dominates over the constant $K$, so in this circumstance, 

\begin{align} \label{approx_1}
  \frac{d^2 \psi}{d \xi^2} \approx \xi^2\psi
\end{align}

which has the approximate solution (verify) 

\begin{align} \label{}
  \psi(\xi) \approx Ae^{-\xi^2 / 2} + Be^{\xi^2 / 2}
\end{align}\vspace{3px}

And since the $B$ term blows up as $|x| \rightarrow \infty$, we get rid of it
to ensure $\psi(\xi)$ is normalizable. The physically acceptable solutions are
then 

\begin{align} \label{large xi}
  \psi(\xi) \rightarrow h(\xi)e^{\xi^2 / 2}, \quad \text{ at large $\xi$} 
\end{align}

where we replaced the constant $A$ with another function $h(\xi)$ in hopes that
it has a simpler functional form that $\psi(xi)$ itself. Differentiating
Equation \ref{large xi}, 


\begin{align} \label{}
  &\frac{d \psi}{d \xi} = \left( \frac{d h}{d \xi} -\xi h \right) e^{\xi^2 / 2}
  \\
  &\frac{d^2 \psi}{d \xi^2} = \left( \frac{d^2 h}{d \xi^2} - 2\xi \frac{d h}{d
  \xi}  + (\xi^2 - 1)h \right) e^{-\xi^2 / 2} \label{eq_2scr}
\end{align}\vspace{3px}

And therefore, equating Equation \ref{1dharmonic} and Equation \ref{eq_2scr}
transforms the Schr\"odinger equation into 

\begin{align} \label{1dscrf}
  \frac{d^2 h}{d \xi^2} - 2\xi \frac{d h}{d \xi}  + (K-1)h = 0
\end{align}\vspace{3px}

And here is where we utilize power series. We search for solutions to Equation
\ref{1dscrf} in the form of a power series in $\xi$. 

\begin{align} \label{}
  h(\xi) = a_0 + a_1\xi + a_2\xi^2 + \cdots = \sum_{j=0}^\infty a_j \xi^j
\end{align}\vspace{3px}

Differentiating the power series term by term to determine $ \frac{d h}{d \xi}
$ and $ \frac{d^2 h}{d \xi^2} $, 

\begin{align} \label{xidiffeqs}
  \frac{d h}{d \xi} &= a_1 + 2a_2\xi + 3a_3\xi^2 + &&\cdots = \sum_{j=0} ja_j
  \xi^{j-1} \\ 
  \frac{d^2 h}{d \xi^2} &= 2a_2 + 2 \cdot 3a_3\xi + 3 \cdot 4a_4\xi^2 + &&\cdots
  = \sum_{j=0}^\infty (j+1)(j+2)a_{j+2}\xi^j
\end{align}\vspace{3px}

Putting all these equations into Equation \ref{1dscrf}, we find

\begin{align} \label{ps}
  \sum_{j=0}^\infty [(j+1)(j+2)a_{j+2} - 2ja_j + (K-1)a_j]\xi^j =  0
\end{align}\vspace{3px}

It follows that the coefficient of \textit{each power} of $\xi$ must vanish
(since the whole thing equals 0). In other words

\[
  (j+1)(j+2) a_{j+2} - 2ja_j + (K-1)a_j = 0
\] \vspace{3px}

Rearranging gives us a recurrence relation, 

\begin{align} \label{rec}
  a_{j+2} = \frac{(2j+1-K)}{(j+1)(j+2)}a_j
\end{align}\vspace{3px}

Starting with $a_0$, we can generate all the even-numbered coefficients.
Starting with $a_1$, we can generate all the odd-numbered coefficients.
Therefore, we can write $h(\xi)$ as 

\begin{align} \label{}
  h(\xi) = h_\text{even} (\xi) + h_\text{odd} (\xi)
\end{align}\vspace{3px}

where 

\begin{align} \label{}
  h_\text{even} &= a_0 + a_2\xi^2 + a_4\xi^4 \\ 
  h_\text{odd} &= a_1\xi + a_3\xi^3 + a_5\xi^5
\end{align}\vspace{3px}

For very large $j$, the recursion formula becomes approximately, 

\[
  a_{j+2} \approx \frac{2}{j}a_j
\] \vspace{3px}

with the solution 

\[
a_j \approx \frac{C}{(j/2)!}
\] \vspace{3px}

for some constant  $C$. This yields (for very large $\xi$ ), 

\[
  h(\xi) \approx C\sum \frac{1}{(j/2)!}\xi^j \approx C\sum \frac{1}{j!}\xi^{2j}
  \approx Ce^{\xi^2}
\] \vspace{3px}

However, something seems wrong. If $h \sim e^{+\xi^2}$, then $\psi(\xi)
= h(\xi)e^{-\xi^2/2} \sim e^{+\xi^2/2}$ -- which blows up for large $\xi$, (and
hence for large $x$ ). There is only one way to resolve this. For normalizable
solutions, \textbf{the series must terminate at some j}. There must occur some
``highest $j$ ". (call it $j_\text{max} $, such that the recursion formula
spits out $a_{j_\text{max}  + 2} = 0$, which will make every coefficient
afterwards also 0. Therefore, for physically acceptable solutions, Equation
\ref{rec} requires 

\[
K = 2j_\text{max} + 1
\] \vspace{3px}

for some positive integer $j_\text{max} $. And since $K
= \frac{2E}{\hbar\omega}$, we recover the energy equation for a harmonic
oscillator 

\begin{align} \label{}
  E_j = \left( j + \frac{1}{2} \right) \hbar\omega, \quad \text{ for } j = 0,
  1, 2, \hdots  
\end{align}\vspace{3px}

We recover, by a completely different method, the fundamental quantization
condition we algebraically uncovered in Equation \ref{harmenergy}. 

For the allowed values of $K$, the recursion formula reads

\begin{align} \label{allowedrec}
  a_{j+2} = \frac{-2(j_\text{max} - j)}{(j+1)(j+2)}a_j
\end{align}\vspace{3px} 

If $j_\text{max} $ = 0, there exists only one term in the series, $h_0(\xi)
= a_0$. And hence 

\[
  \psi_0(\xi) = a_0e^{-\xi^2 / 2} \qquad \text{ if $j_\text{max} = 0 $ }
\] \vspace{3px}

which, apart from normalization is equivalent to Equation \ref{lowestrung}. For
$j_\text{max} =1$, we take $a_0 = 0$, and Equation \ref{allowedrec} with $j=1$ yields $a_3
= 0$, so 

\begin{align} \label{}
  h_1(\xi) = a_1\xi \quad \Rightarrow \quad \psi_1(\xi) a_1\xi e^{-\xi^2 / 2}
\end{align}\vspace{3px}

For $j_\text{max} = 2$, $j=0$ yields $a_2 = -2a_0$, and $j=2$ gives $a_4 = 0$,
so 

\begin{align} \label{}
  h_2(\xi) = a_0 \left( 1 - 2\xi^2 \right) \quad \Rightarrow \quad \psi_2(\xi)
  = a_0 (1-2\xi^2) e^{-\xi^2 / 2}
\end{align}\vspace{3px}   

and so on. The polynomials for $h(\xi)$ that are generated are known as
\textbf{Hermite Polynomials}. There are actually two different forms of Hermite
Polynomials -- the ``physicist's" and the ``probabilist's," that are related.
Obviously, the type us superior physicists use are the physicist's Hermite
Polynomials. By convention, the arbitrary multiplicative factor is chosen so
that the coefficient of the highest power of $\xi$ is $2^{j_\text{max} }$.
Then, the normalized stationary states for the harmonic oscillator are 

\begin{align} \label{powerseriessoln}
  \psi_n(x) = \left( \frac{m\omega}{\pi \hbar}\right)^{1/4} \frac{1}{\sqrt{2^n
  n!}} H_n(\xi) e^{-\xi^2 / 2}
\end{align}\vspace{3px}

where we use $j_\text{max} = n$ to label the quantum number of each state. These are identicial of course to the ones we obtained algebraically in
Equation \ref{laddersolution}. I plot the solutions $\psi_n(x)$ in Figure
\ref{graphsoln} below. 

\begin{figure}[!ht]
  \centering
    \includegraphics[width = 8cm]{harmonicoscillatorsoln.pdf}
    \caption{Stationary State Wave Functions for 1D Harmonic Oscillator}
    \label{graphsoln}
\end{figure}

Note the energies are evenly spaced as $E_n = \left( n + \frac{1}{2} \right)
\hbar\omega $. We can afterwards determine $\Psi(x, t)$ via the prime
directive using Equation \ref{primedirec}. 

\section{The Free Particle}

We have previously (in studying the infinite square well) considered a particle
that could propagate freely in the region $-a/2 < x < a/2$, but was confined at
the boundaries by an infinitely strong potential. The solutions of the
time-independent Schr\"odinger equation were states of definite energy. We
described these states as $\sin$ and $\cos$ functions -- standing waves with
fixed nodes, formed by combining left-moving with right-moving amplitudes.
These states were normalizable and provided a complete orthonormal basis for
describing the propagation of any wave packet $\Psi(x, t)$. 

Here we consider a similarly free propagating particle, but one not confined by
any potential. Instead the particle is free -- able to move over the region
$-\infty < x < \infty$. Here however, we will choose to describe the waves as
plane waves in the form $e^{ikx}$ rather than with $\sin$ and $\cos$. These
states, as you shall soon see, are not normalizable and are thus not true
stationary states, yet they are still of use as they form a basis for expanding
physical states -- wave packets -- that \textit{are} normalizable physical
states. 

As the particle is free $(V(x) = 0)$ for all $-\infty < x < \infty$, the
Schr\"odinger equation for the time-independent stationary states is the
following: 

\begin{align} \label{}
  -\frac{\hbar^2}{2m} \frac{d^2 \psi(x)}{d x^2} = E\psi(x)
\end{align}\vspace{3px}

In terms of the wave number 

\[
k = \frac{\sqrt{2mE}}{\hbar} \quad \Rightarrow \quad \frac{d^2 \psi(x)}{d x^2}
= -k^2\psi(x)
\] \vspace{3px}

And therefore the general time-independent solution is 

\begin{align} \label{}
  \psi(x) = Ae^{ikx} + Be^{-ikx}
\end{align}\vspace{3px}

And the full time-dependent solution is therefore, via the prime directive, 

\begin{align} \label{}
  \Psi(x, t) = Ae^{ikx - iEt/\hbar} + Be^{-ikx - iEt/\hbar} \quad k \text{
  positive} 
\end{align}\vspace{3px}


If we allow $k$ to run over both positive and negative values, then $k = \pm
\frac{\sqrt{2mE}}{\hbar}$, and this simplifies the time-dependent solution into 


\begin{align} \label{}
  \Psi(x, t) = Ae^{ikx - iEt/\hbar} = Ae^{i\left( kx - \frac{\hbar k^2}{2m}t
  \right) } \quad k \text{ positive or negative}   
\end{align}\vspace{3px}


We can identify the \textit{velocity} of our solutions by jumping on the wave
function -- hanging onto a point of fixed phase -- and measuring which way we
travel. We take a positive step in time $\Delta t$ and demand that the phase
remain fixed 

\begin{align} \label{}
  kx - \frac{\hbar k^2}{2m}t \rightarrow k(x + \Delta x) &- \frac{\hbar
  k^2}{2m}(t + \Delta t) \Rightarrow k\Delta x - \frac{\hbar k^2}{2m}\Delta
    t = 0 \\ &\Rightarrow \frac{\Delta x}{\Delta t} \equiv v = \frac{\hbar
    k }{2m} \label{quantumv}
\end{align}\vspace{3px}

Therefore, our solutions with positive $k$ have a positive velocity (traveling
to the right) while those with negative $k$ have negative velocity (traveling
to the left). 

\begin{alignat*}{3}
  k &= +\frac{\sqrt{2mE}}{\hbar} > 0 \rightarrow \text{ traveling to the right }&& (+x)  
    \\ k &= -\frac{\sqrt{2mE}}{\hbar} < 0 \rightarrow \text{ traveling to the left } && (-x)
  \end{alignat*}

We can define the wavelength as a positive number 

\begin{align} \label{wavelengthk}
  \lambda = \frac{2\pi}{|k|}
\end{align}\vspace{3px}

but include a sign in the de Broglie relationship for momentum, 

\[
 p = \frac{2\pi\hbar}{\lambda}\rightarrow (2\pi\hbar) \left( \frac{k}{2\pi}
 \right) = \hbar k 
\] \vspace{3px}

so that momentum becomes a signed quantity (positive for waves moving to the
right, negative for those moving to the left). 

However, there is are two issues with our calculations. The first is regarding
our calculated wave velocity. The velocity of the
waves we calculated in Equation \ref{quantumv} is 

\begin{align} \label{}
  v_\text{quantum} = \frac{\hbar |k|}{2m} = \sqrt{\frac{E}{2m}}
\end{align}\vspace{3px}


On the other hand, the \textit{classical} speed of a free particle with energy
$E$ is given by $E = \frac{1}{2}mv^2$, so 

\begin{align} \label{}
  v_\text{classical} = \sqrt{\frac{2E}{m}} = 2v_\text{quantum} 
\end{align}\vspace{3px}

We get a quantum mechanical wave function that travels at \textit{half} the
speed of what the particle \textit{should}. The second issue is the fact that
our calculated wave function \textit{is not normalizable}. 

\begin{align} \label{}
  \int_{-\infty}^{\infty}  |\psi(x, t)|^2 \, dx = |A|^2 \int_{-\infty}^{\infty}
   \left( e^{-ikx + iEt/\hbar} \right) \left(
  e^{ikx - iEt/\hbar} \right)  \, dx \quad \rightarrow \quad |A|^2 (\infty)   
\end{align}\vspace{3px}


Therefore, our solutions do not represent physically realizable states. A free
particle can not exist in a stationary state -- \textit{ there is no such thing
as a free particle of definite energy.} But like I said previously, they still
serve purpose. They play a \textit{mathematical} role entirely dependent of
their \textit{physical} interpretation. The general solution to the
time-dependent Schr\"odinger equation via the prime directive is still a linear
combination of our stationary states. Only this time its an \textit{integral}
over the continuous variable $k$ instead of a discrete sum over index $n$. 

\begin{align} \label{freeparticlesolution}
  \Psi(x, t) = \frac{1}{\sqrt{2\pi}}\int_{-\infty}^{\infty} \phi(k) e^{i\left(
  kx - \frac{\hbar k^2}{2m}\right) } \, dk 
\end{align}\vspace{3px}

The quantity $\frac{1}{\sqrt{2\pi}}$ is factored out for convenience.e This
wave function \textit{can} be normalized. But it carries a \textit{range} of
$k$s, and hence a range of energies and velocities. This is what we call
a \textbf{wave packet}.  

In a normal quantum mechanics problem, we are provided $\Psi(x, 0)$ and asked
to find $\Psi(x, t)$. For a free particle the solution takes the form of
Equation \ref{freeparticlesolution}. But how do we determine $\phi(k)$? so as
to match the initial wave function

\[
  \Psi(x, 0) = \frac{1}{\sqrt{2\pi}} \int_{-\infty}^{\infty} \phi(k)e^{ikx} \,
  dk
\] \vspace{3px}

The answer, as you may have noticed given the form of Equation
\ref{freeparticlesolution} is via Fourier/Inverse fourier Transform. 

\[\boxed{
  f(x) = \frac{1}{\sqrt{2\pi}}\int_{-\infty}^{\infty} F(k)e^{ikx} \, dk \quad
  \Longleftrightarrow \quad F(k) = \frac{1}{\sqrt{2\pi}} \int_{-\infty}^{\infty}
f(x) e^{-ikx} \, dx}
\] \vspace{3px}


$F(k)$ is the \textit{fourier transform} of $f(x)$, and likewise $f(x)$ is the
\textit{inverse fourier transform} of $F(k)$. These integrals exist if the
initial wave packet $\Psi(x, 0)$ is normalized. The solution for a free
particle is Equation \ref{freeparticlesolution} with 

\begin{align} \label{solvingphi}
  \phi(k) = \frac{1}{\sqrt{2\pi}} \int_{-\infty}^{\infty} \Psi(x, 0)e^{-ikx} \,
  dx
\end{align}\vspace{3px}

Putting everything together, 

\begin{mainbox}{Wave Packet $\rightarrow$ Wave Function}
  Given an arbitrary initial wave packet $\Psi(x, 0)$. \\

  For a discrete system (Harmonic Oscillator, Infinite Square Well,  $\hdots$)

   \[
  \Psi(x, 0) = \sum_n c_n \psi_n(x) \qquad c_n = \int_\Omega \psi_n^*(x)\Psi(x,
  0) \, dx
  \] \[ \Psi(x, t) = \sum_n c_n \psi_n(x) e^{-iE_n t/\hbar} \] \vspace{3px}
  
where $\psi_n(x)$ and $E_n$ are the solutions to the discrete problem, and
where the integration in $x$ is over the region $\Omega$ where $\psi_n$ is
nonzero. \\

For a free particle, with its continuous plane-wave basis, 

\[ \Psi(x, 0) = \int_{-\infty}^{\infty} \phi(k)\psi(k, x) \, dk \qquad \phi(k)
= \int_{-\infty}^{\infty} \psi^*(k, x)\Psi(x, 0) \, dx \] \[ \Psi(x, t)
= \int_{-\infty}^{\infty} \phi(k)\psi(k, x) e^{-iE(k) t /\hbar} \, dk \]
\vspace{3px}
  
where $\psi(k, x) \equiv \sqrt{\frac{1}{2\pi}} e^{ikx}$ and $E(k)
= \frac{\hbar^2 k^2}{2m}$.
\end{mainbox}


\subsection{Worked Example} 

We start with a normalized wave packet that has been confined to a region of
width $2a$, 

\begin{align} \label{problemfree}
\Psi(x, 0) = \begin{cases}
  \frac{1}{\sqrt{2a}} &\qquad -a < x < a \\ 0 &\qquad |x| > a
\end{cases}
\end{align} \vspace{3px}

Using Equation \ref{solvingphi}, 

\[
  \phi(k) = \int_{-\infty}^{\infty} \left( \frac{1}{\sqrt{2\pi}}e^{-ikx}
  \right) \Psi(x, 0) \, dx = \frac{1}{\sqrt{2\pi}}\frac{1}{\sqrt{2a}}
  \int_{-a}^{a} \cos kx \, dx = \frac{1}{\sqrt{\pi a}} \frac{\sin ak}{k}
\] \vspace{3px}

Then finally, via the prime directive, or by using Equation
\ref{freeparticlesolution}, 

\begin{align} \label{solutionfree}
  \Psi(x, t) = \int_{-\infty}^{\infty} \phi(k)\psi(k, x) e^{-iE(k)t/\hbar} \,
  dk = \int_{-\infty}^{\infty} \frac{1}{\sqrt{\pi a}} \frac{\sin ak}{k} \left(
  \frac{1}{\sqrt{2\pi}} e^{ikx}\right)  e^{-i\hbar k^2 t / 2m} \, dk
\end{align} \vspace{3px}

Below I plot the initial probability density function at $|\Psi(x, 0)|^2$
along with the probability density function $|\Psi(x, ma^2 / \hbar)|^2$, at
a later time $t = ma^2 / \hbar$ (the curve) in Figure \ref{freegraph}. As you
can see, the probability density function and the wave function itself begin
to delocalize over time.

\begin{figure}[H]
  \centering
    \includegraphics[width = 16cm]{wavepacket.pdf}
    \caption{Plot of $|\Psi(x, 0)|^2$ and $|\Psi(x, ma^2 / \hbar)|^2$.}
    \label{freegraph}
\end{figure}

\section{The Finite Square Well} 

We have previously solved problems involving \textit{bound} stationary states
(Infinite Square Well, Harmonic Oscillator) and where the basis consists
entirely of plane waves (Free Particle). Here we tackle the first problem where
free and bound states coexist. The \textit{Finite Square Well} depicted in
Figure \ref{finitewell} below. 

\begin{figure}[H] 
  \centering
    \includegraphics[width = 13cm]{finitewell.pdf}
    \caption{Finite Well Potential}
    \label{finitewell}
\end{figure}



\subsection{Bound States} We seek solutions where $E < 0$. In region
I the  Schr\"odinger equation is 

\[
-\frac{\hbar^2}{2m} \frac{d^2 \psi(x)}{d x^2} = E \psi(x) = -|E|\psi(x) \quad
\Rightarrow \frac{d^2 \psi(x)}{d x^2} = \kappa^2 \psi(x); \qquad \kappa
\equiv \frac{\sqrt{2m|E|}}{\hbar} > 0
\] \vspace{3px}

The general solution is 

\[
  \psi(x) = Ae^{-\kappa x} + Be^{\kappa x} 
\] \vspace{3px}

However, the first term diverges as $x \rightarrow -\infty$. Therefore the
solution in region I is 

 \[
   \text{Region I: } \psi_I(x) = Be^{\kappa x}, \quad x < -\frac{a}{2}
\] \vspace{3px}

The same arguments lead to 

\[
  \text{Region III: }  \psi_{III}(x) = Ae^{-\kappa x}, \quad x > \frac{a}{2}
\] \vspace{3px}


In region II we have

\[
-\frac{\hbar^2}{2m} \frac{d^2 \psi(x)}{d x^2} - V_0 \psi(x) = -|E|\psi(x) \quad
\Rightarrow \quad \frac{d^2 \psi(x)}{d x^2} = -k^2\psi(x); \qquad k \equiv
\frac{\sqrt{2m(V_0 - |E|}}{\hbar}
\] \vspace{3px}

We require $V_0 > |E|$ -- the eigenvalue above the floor of the potential -- as
is the case classically. The general solution is 

\[
  \text{Region II: } \psi_{II}(x) = C\cos kx + D \sin ks, \qquad -\frac{a}{2}
  < x < \frac{a}{2}
\] \vspace{3px}

As the potential is mirror-symmetric, we can label the wave function by parity
-- whether even or odd around zero. In this way we can construct solutions
corresponding to either the even or odd terms in region II. 

\paragraph{\textit{Even Solution:}} Choosing the even-parity solution of region
II, 

\[
  \text{Region II: } \psi_{II}(x) = C\cos kx, \qquad -\frac{a}{2}
  < x < \frac{a}{2} 
\] \vspace{3px}

Second order differential equations require two constraints as initial
conditions to solve. We can use the required continuity of the wave function
across regions for both. The value and first derivative of the wave function at
the boundary between regions must match to guarantee continuity. Thus at the
boundary of II and III, 

\begin{align} \label{}
  C\cos kx = Ae^{-\kappa x}\Big|_{x = a/2} \quad &\Rightarrow  \quad C\cos
  \frac{ka}{2} = Ae^{-\frac{\kappa a }{2}} \\ -Ck\sin kx = -\kappa Ae^{-\kappa
x}\Big|_{x = a/2} \quad &\Rightarrow \quad -Ck\sin \frac{ka}{2} = -A\kappa
e^{-\frac{\kappa a}{2}}
\end{align}\vspace{3px}

These yield two constraints: 

\[
  A = Ce^{\frac{\kappa a}{2}} \cos \left( \frac{ka}{2} \right) \qquad k\tan
  \left( \frac{ka}{2} \right) = \kappa,   
\] \vspace{3px}

the second of which, the eigenvalue condition, is obtained by taking the ratio
of the two equations. \\
The same matching at the boundary of I and II yields

\begin{align} \label{}
  C\cos \left( -\frac{ka}{2} \right)  = Be^{-\frac{ka}{2}} \quad &\text{ and
  }  \quad -Ck\sin \left( -\frac{ka}{2} \right) = B\kappa e^{-\frac{-\kappa
  a}{2}} \\ C\cos \left( \frac{ka}{2} \right) = Be^{-\frac{ka}{2}} \quad
  &\text{ and } \quad Ck\sin\left( \frac{ka}{2} \right) = B\kappa
e^{-\frac{\kappa a}{2}}  
\end{align}\vspace{3px}

These yield the same eigenvalue constraints as above with an additional
condition, 

\[
  B = Ce^{\frac{\kappa a}{2}} \cos \left( \frac{ka}{2} \right)  
\] \vspace{3px}

The matching at the second boundary was unnecessary -- we could have just used
the fact that the wave function is even under parity to get the same
constraint. We finally obtain

\begin{subbox}{Even Solution}
  \[
  \psi(x) = \begin{cases}
    Ce^{\frac{\kappa a}{2}} \cos \left( \frac{ka}{2} \right) e^{\kappa x}
    \qquad &x < -\frac{a}{2} \\ C \cos kx \qquad &-\frac{a}{2}
    < x < \frac{a}{2} \\ Ce^{\frac{\kappa a }{2}} \cos \left( \frac{ka}{2}
  \right) e^{-\kappa x} \qquad &x > \frac{a}{2}  
  \end{cases} 
\] 

\[ \text{eigenvalue condition: } k \tan \left( \frac{ka}{2} \right) = \kappa \]  
\end{subbox}


The remaining work is to solve for the allowed eigenvalues, and find $C$ by
requiring the solution to be normalized. This will be shown after deriving the
odd-parity solution.

\paragraph{\textit{Odd Solution:}} Choosing the odd-parity solution of region
II, 

\[
  \text{Region II: } \psi_{II}(x) = D\sin kx, \qquad -\frac{a}{2}
  < x < \frac{a}{2} 
\] \vspace{3px}

We match the solutions at the boundary of II and III, 

\begin{align} \label{}
  D\sin kx = Ae^{-\kappa x}\Big|_{x = a/2} \quad &\Rightarrow \quad D \sin
  \frac{ka}{2} = Ae^{-\frac{\kappa a}{2}} \\ Dk\cos kx = -\kappa Ae^{-\kappa
x}\Big|_{x = a/2} \quad &\Rightarrow \quad Dk\cos \frac{ka}{2} = -A\kappa
e^{-\frac{\kappa a}{2}}
\end{align}\vspace{3px}

These yield two constraints, 

\[
  A = De^{\frac{\kappa a}{2}} \sin \left( \frac{ka}{2} \right) \qquad k\cot
  \left( \frac{ka}{2} \right) = -\kappa.
\] \vspace{3px}

The same matching at the boundary of I and II yields 

\begin{align}
  D\sin \left( -\frac{ka}{2} \right) = Be^{-\frac{ka}{2}} \quad &\text{ and
  } \quad Dk \cos \left( -\frac{ka}{2} \right) = B\kappa e^{-\frac{\kappa
  a}{2}} \quad \Rightarrow \\ -D \sin \left( \frac{ka}{2} \right)
  = Be^{-\frac{ka}{2}} \quad &\text{ and } \quad Dk\cos \left( \frac{ka}{2}
  \right) = B\kappa e^{-\frac{\kappa a}{2}}     
\end{align} \vspace{3px}

These yield the same eigenvalue constraints as above with the additional
condition

\[
  -B = A = De^{\frac{\kappa a}{2}}\cos \left( \frac{ka}{2} \right)  
\] \vspace{3px}

which makes sense as the solution in region I is the odd-mirror of the solution
in region II. Therefore we obtain

\begin{mainbox}{Odd Solution}
  \[
  \psi(x) = \begin{cases}
    -De^{\frac{\kappa a}{2}} \sin \left( \frac{ka}{2} \right) e^{\kappa x}
    \qquad &x < -\frac{a}{2} \\ D\sin ks \qquad &-\frac{a}{2} < x < \frac{a}{2}
    \\ De^{\frac{\kappa a}{2}} \sin \left( \frac{ka}{2} \right) e^{-\kappa x}
    \qquad & x>\frac{a}{2} 
  \end{cases} 
  \] 
  \[ \text{eigenvalue condition: } k \cot \left( \frac{ka}{2} \right) = -\kappa
  \]
\end{mainbox}

\begin{subbox}{Even Solution}
\[
  \psi(x) = \begin{cases}
    Ce^{\frac{\kappa a}{2}} \cos (\frac{ka}{2}) e^{\kappa x}
      &x < -\frac{a}{2} \\ C \cos kx   &-\frac{a}{2}
    < x < \frac{a}{2} \\ Ce^{\frac{\kappa a }{2}} \cos (\frac{ka}{2})
    e^{-\kappa x}   &x > \frac{a}{2}
  \end{cases}
\]
\[ \text{eigenvalue condition: } k \tan \left( \frac{ka}{2} \right) = \kappa \]
\end{subbox}



The kinematic variables are 

\[
k = \frac{\sqrt{2m(V_0 - |E|}}{\hbar} \qquad \kappa = \frac{\sqrt{2m|E|}}{\hbar}
\] \vspace{3px}

Now all that is left is to calculate the eigenvalues of the bound states, then
normalize. 

\subsection{Counting Eigenvalues}

The allowed bound-state energies are
obtained by solving the eigenvalue conditions 

\[
k\tan\left( \frac{ka}{2} \right) = \kappa \quad \text{ and } \quad k \cot
\left( \frac{ka}{2} \right) = -\kappa   
\] \vspace{3px}

for the even and odd cases, respectively, subject to the constraint 

\[
k^2 + \kappa^2 = \frac{2mV_0}{\hbar^2} \quad \Rightarrow \quad \kappa^2
= \frac{2mV_0}{\hbar^2} - k^2
\] \vspace{3px}

Multiplying the top equations by $a/2$ and the bottom equations by $ (a/2)^2$,
we can re-express them in a simpler form: 

\begin{mainbox}{}
  \[z \equiv \frac{ka}{2} = \frac{a\sqrt{2m(V_0 - |E|)}}{2\hbar} \qquad z_0
  \equiv \frac{a}{2\hbar}\sqrt{2mV_0} \quad \text{ so } \frac{\kappa a}{2}
= \sqrt{z_0^2 - z^2}\]
\[ \text{even eigenvalues: }  \tan z = \frac{1}{z} \sqrt{z_0^2 - z^2}
  = \sqrt{\frac{z_0^2}{z^2} - 1}\] \[ \text{odd eigenvalues: } \cot
z = -\frac{1}{z} \sqrt{z_0^2 - z^2} = -\sqrt{\frac{z_0^2}{z^2} - 1}\]  
\end{mainbox}


$z_0$ is a unit-less number derived by the potential we are given -- the
parameters $V_0$ and $a$. \\
Before we draw graphs to visualize this in Figure \ref{z0evenodd}, consider the process of gradually
increasing $V_0$, which allows more and more states to be captured in the
potential well. When does a new state enter? When $|E| \sim 0 \Rightarrow
\kappa = 0 \Rightarrow z \sim z_0$. This means  

\[
\text{even: } \tan z \sim \tan z_0 = 0 \quad \Rightarrow \quad z_0
= n\frac{\pi}{2}, n = 0, 2, 4, \hdots \quad \Rightarrow \quad a^2V_0 = n^2\pi^2
\frac{\hbar^2}{2m}, n = 0, 2, 4, \hdots
\] \vspace{3px}

Therefore if 

\[
  (n+2)^2\pi^2 \frac{\hbar^2}{2m} > a^2V_0 > n^2\pi^2 \frac{\hbar^2}{2m}, \;
  n = 0, 2, 4, \hdots \quad \text{ there are } \frac{n}{2} + 1 \text{ even
  bound states} 
\] \vspace{3px}

Note that there is always at least one bound state that persists even as $V_0
\to 0$. \\
For the odd-parity states, 

\[
\text{odd: } \cot z \sim \cot z_0 = 0 \quad \Rightarrow \quad z_0
= n\frac{\pi}{2}, n = 1, 3, 5, \hdots \quad \Rightarrow \quad a^2V_0 = n^2\pi^2
\frac{\hbar^2}{2m}, n = 1, 3, 5, \hdots
\] \vspace{3px}

Therefore if 

\[
  (n+2)^2 \pi^2 \frac{\hbar^2}{2m} > a^2V_0 > n^2\pi^2 \frac{\hbar^2}{2m},
  n = 1,3, 5, \cdots \quad \text{ there are } \frac{n+1}{2} \text{ odd bound
  states} 
\] \vspace{3px}

Putting these two results together: 

\begin{subbox}{Number of Energy Eigenstates}
  For a finite well of width $a$ and depth $V_0$, if

  \[
  (n+1)^2 \pi^2 \frac{\hbar^2}{2m} > a^2V_0 > n^2\pi^2
  \frac{\hbar^2}{2m}, n = 0, 1, 2, \hdots
  \] \vspace{3px}
  
  there are $n+1$ bound states. The equivalent condition on $z_0$ is 

  \begin{align} \label{finite_condition}
    (n+1)\frac{\pi}{2} > z_0 > n\frac{\pi}{2}
  \end{align} \vspace{3px}
\end{subbox}

\subsection{Finding Eigenvalues} Our eigenvalue equations are 

\[
\tan z = \sqrt{\frac{z_0^2}{z^2} - 1} \qquad \cot z = -\sqrt{\frac{z_0^2}{z^2}
- 1}
\] \vspace{3px}


which we must solve as a function of $z_0$. We do so by plotting the left and
right-handed sides, for a series of $z_0^2$ and finding the intersections. This
is shown in Figure \ref{z0evenodd} for the cases of $z_0 = \pi /4$ and $z_0
= 3\pi / 4$. The
pattern of solutions follows the discussion above. One can also use Mathematica
or MATLAB to get accurate results. 

\paragraph{\textit{Consistency with the Infinite Square Well:}}  In our calculation we can take $V_0 \rightarrow \infty$ or equivalently $z_0
\rightarrow \infty$. Then our even eigenvalue equation becomes

\[
\tan z = \tan \frac{ka}{2} = \frac{1}{z} \sqrt{z_0^2 - z^2} \rightarrow \infty
\quad \Rightarrow \quad \frac{k_n a}{2} = \frac{n\pi}{2}, \quad n = 1, 3, 5,
\hdots
\] \vspace{3px}


and our odd equation becomes 

\[
-\cot z = -\cot \frac{ka}{2} = \frac{1}{z} \sqrt{z_0^2 - z^2} \rightarrow
\infty \quad \Rightarrow \quad \frac{k_n a}{2} = \frac{n\pi}{2}, \quad n = 2,
4, 6,\hdots
\] \vspace{3px}

Thus

\[
k_n a = n\pi, \quad n = 1,2,3,\hdots \quad \Rightarrow \quad E_n
= \frac{\hbar^2k_n^2}{2m} = \frac{n^2\pi^2\hbar^2}{2ma^2}, \quad
n = 1,2,3,\hdots
\] \vspace{3px}

which is the same result as before. 




\subsection{Finding Bound Stationary States Example}

We consider a specific case where 

\[
z_0 = \frac{\pi}{20}
\] \vspace{3px}

Let's first determine how many bound states this potential can hold. Referring
to our eigenvalue condition on $z_0$ from Equation \ref{finite_condition},  

\[
  (n+1)\frac{\pi}{2} > \frac{\pi}{20} > n\frac{\pi}{2} 
\] \vspace{3px}

The only $n$ that satisfies this equation is $n = 0$. Therefore this potential
will have only a single, even-parity bound state. This choice leads to

 \[
z_0 \equiv \frac{a}{2\hbar} \sqrt{2mV_0} \quad \Rightarrow \quad V_0 = \left(
\frac{\pi}{20} \right) ^2 \frac{2\hbar^2}{ma^2} 
\] \vspace{3px}

This is a very weak potential, one that supports a barely-bound state. We can
compute the wave number and wavelength of the wave function inside the well, 

\[
k = \frac{\sqrt{2m(V_0 - |E|)}}{\hbar} < \frac{\sqrt{2mV_0}}{\hbar}
= \frac{\pi}{10}\frac{1}{a} \quad \Rightarrow \lambda = \frac{2\pi}{k} > 20a
\] \vspace{3px}

The wave length is \textit{much} longer than the width of the well. We will
return to this fact soon when we study \textit{quantum tunneling}. Solving our
even eigenvalue equation yields 

\[
z \equiv \frac{ka}{2} = 0.1551918
\] \vspace{3px}

But from our earlier formulae, 

\[
\frac{\kappa a}{2} = \sqrt{z_0^2 - \left( \frac{ka}{2} \right) ^2 } = 0.02428
\quad \text{ and } \quad |E| = \left( \frac{\kappa a}{2} \right) ^2
\frac{2\hbar^2}{ma^2} = \left( \frac{\kappa a}{2} \right) ^2 V_0 \left(
\frac{20}{\pi} \right) 2 = 0.02245V_0   
\] \vspace{3px}

This state is very near the top of the well -- 98\% of the well is below. The
wave function is almost free! Only about 5\% of the probability density is
within the well -- the rest is spread out in the classically forbidden region.
We could make this solution less and less bound, forcing it to approach
arbitrarily close to a free particle wave of the form 

\[
  \frac{1}{2}\left[ e^{ikx} + e^{-ikx} \right]
\] \vspace{3px}

\begin{figure}[H]
  \centering
    \includegraphics[width = 16cm]{z0evenodd.pdf}
    \caption{Finite Well even (left) and odd (right) parity plots to identify
    eigenvalues. For $z_0 = \pi /4$, one even solution is found. For $z_0 = 3\pi
  / 4$, one even and one odd solution is found.}
    \label{z0evenodd}
\end{figure}

\section{The Delta Function Potential} 

Here we define the delta function $\delta(x)$ as the limit 

\[
  \delta(x - 0) = \delta(x) \equiv \lim_{b\to 0}
  \frac{1}{|b|\sqrt{\pi}}e^{-(x/b)^2}
\] \vspace{3px}

If we integrate the Gaussian over all $x$, 

\[
  \int_{-\infty}^{\infty} e^{-(x/b)^2} \, dx = \sqrt{\pi}|b| \quad \text{ so
  } \quad \int_{-\infty}^{\infty} \delta(x) \, dx = 1
\] \vspace{3px}

\begin{mainbox}{Delta Function Definition and Properties}
  \[
    \delta(x) \equiv \lim_{b\to 0} \frac{1}{|b|\sqrt{\pi}}e^{-(x/b)^2} \qquad
    \int_{-\infty}^{\infty} f(x)\delta(x-\beta) \, dx = f(\beta) \qquad
    \delta(\beta x) = \frac{1}{|\beta|}\delta(x)
  \] \vspace{3px}  
\end{mainbox}

\paragraph{\textit{Delta Function Potential:}} Consider the case of a square
well potential and a weakly bound state, where $\lambda \gg a$. Were we to
consider the expectation value of $V(x)$ for this low-momentum state, we could
approximate

\begin{align}
  \int_{-\infty}^{\infty} &\psi^*(x) V(x) \psi(x) \, dx \sim
  \int_{-\infty}^{\infty} [\psi*(0) + x\psi'^*(0) + \cdots] V(x) [\psi(0)
  + x\psi'(0) + \cdots] \, dx \\
                          &= \psi^*(0)\psi(0) \int_{-\infty}^{\infty} V(x) \,
                          dx + (\psi'^*(0)\psi(0) + \psi^*(0)\psi'(0))
                          \int_{-\infty}^{\infty} xV(x) \, dx + \cdots 
\end{align}

For the limits of the well we get,  

\begin{align} 
  \psi^*(0) & \psi(0) \int_{-a/2}^{a/2} (-V_0) \, dx + (\psi'^*(0)\psi(0)
  + \psi^*(0)\psi'(0)) \int_{-a/2}^{a/2} x(-V_0) \, dx + \cdots \\
  &= -\psi^*(0)V_0 a\psi(0) + 0(\psi'^*(0)\psi(0)
  + \psi^*(0)\psi'(0)) + \cdots \\ 
  &= -V_0a \int \psi^*(x) \delta(x) \psi(x) \, dx 
\end{align} \vspace{3px}

where the second term vanishes because our potential is even. Therefore, our
work above identifies 

\[
\int_{-\infty}^{\infty} \psi^*(x) V(x) \psi(x) \, dx \sim -a V_0 \int \psi^*(x)
\delta(x) \psi(x) \, dx
\] \vspace{3px}

These two potentials -- $V(x)$ and $-aV_0 \delta(x)$ would give us equivalent
answers if we restrict our attention to low energies. For such low-momentum,
long-wavelength states, this derivation suggests we can replace our square well
interaction $V(x)$ with a simpler one: 

\[
V(x) \rightarrow V_\delta(x) \equiv -aV_0\delta(x)
\] \vspace{3px}


This derivation showed that the delta function `strength' in general is given
by the integral 

\[
\int_{-a/2}^{a/2} V(x) \, dx
\] \vspace{3px}

where $V(x)$ is the potential and $a$ defines the region over which the
unspecified potential is nonzero. We define $\alpha = aV_0$ to be the strength
of the attractive finite well potential. The finite square well potential
$V(x)$ and $V_\delta = \alpha\delta(x)$ should yield equivalent solutions of
the Schr\"odinger equation that have wavelength $\lambda \gg a$. 

But this
also guarantees that we can take a similar limit of the square well potential
itself, shrinking $a$ and proportionally increasing $V_0$, since the proper
delta function strength depends on the product $\alpha = aV_0$. This is nice
because we have solved the square well, and can use those results for any $a$.

When counting the number of possible eigenstates in Equation
\ref{finite_condition}, we found 

\[ \text{If } \pi^2 \frac{\hbar^2}{2m} > a^2V_0 > 0, \text{ we have one even
bound state} \]
\[ \text{If } 4\pi^2\frac{\hbar^2}{2m} > a^2V_0 > \pi^2\frac{\hbar^2}{2m},
  \text{ we have one even and one odd bound state} \] \vspace{3px}

In taking the delta function limit, we keep the product $aV_0$ constant. But
the factor appearing above is $a^2V_0 = a\alpha$. Thus as we make $a$ smaller,
clearly at some point the second equation is unsatisfied. But the first states 

\[
a^2V_0 = \alpha a > 0
\] \vspace{3px}

No matter how small we make $a$, keeping $aV_0$ fixed, this condition remains
satisfied. We conclude 

\begin{subbox}{}
  An attractive delta function potential $V_\delta = -\alpha V_0 \delta(x),
  \alpha > 0$, has exactly one even bound state, independent of the potential's
  strength $\alpha$. 
\end{subbox}

What is the strength of this bound state? The finite square well eigenvalue
condition is 

\[
  \frac{ka}{2}\tan \left( \frac{ka}{2} \right) = \frac{\kappa a}{2} \qquad
  k = \frac{\sqrt{2m(V_0 - |E|)}}{\hbar} \qquad \kappa
  = \frac{\sqrt{2m|E|}}{\hbar} 
\] \vspace{3px}

But as $a\rightarrow 0$, 

\[
\frac{ka}{2} = \sqrt{a}\frac{\sqrt{2ma(V_0 - |E|)}}{2\hbar}
= \sqrt{a}\frac{\sqrt{2maV_0(1 - \frac{|E|}{V_0})}}{2\hbar} \rightarrow 0 \quad
\text{ so } \quad \frac{ka}{2}\tan \left( \frac{ka}{2} \right) \rightarrow
\left( \frac{ka}{2} \right) ^2  
\] \vspace{3px}

So our eigenvalue equation becomes 

\begin{align} \label{}
  \frac{ka}{2}\tan\left( \frac{ka}{2} \right) = \frac{\kappa a}{2} \quad
&\Rightarrow \quad \left( \frac{ka}{2} \right) ^2 = \frac{\kappa a }{2} \quad
\text{ exact as } a \rightarrow 0 \quad \Rightarrow \\
  \frac{a}{4\hbar^2}2maV_0 \left( 1 - \frac{|E|}{V_0} \right)
  = \frac{a}{2\hbar}\sqrt{2m|E|} \quad &\Rightarrow \quad
  \frac{m(aV_0)}{\hbar}\left( 1- \frac{|E|}{V_0} \right)  = \sqrt{2m|E|} \\
                                       &\Rightarrow \quad \frac{m(aV_0)}{\hbar}
  = \sqrt{2m|E|} \quad \text{ exact as
                                       } a \rightarrow 0
\end{align}\vspace{3px}

Solving for $|E|$, 

\[
|E| = \frac{m(aV_0)^2}{2\hbar^2} = \frac{m\alpha^2}{2\hbar^2}
\] \vspace{3px}

This result is also valid (for long wavelength $\psi$ ) for our finite well
potential and a weakly bound state, but as noted, becomes \textit{exact} when
we take the delta function limit, keeping $\alpha = aV_0$ fixed. This makes
sense, as in the limit \textit{every} wave function is long compared to the
delta function potential. In summary 


\begin{mainbox}{}
  The bound state of a delta function potential $V_\delta = -\alpha \delta(x)$
  has a binding energy of 
  \[ |E| = \frac{m\alpha^2}{2\hbar^2} \]
\end{mainbox}

The other result we can derive concerns the derivative of the wave function at
the delta function potential,
or more correctly, the discontinuity of the derivative across the delta function. We saw in the case
of the infinite square well that the derivative could be discontinuous at the
interface with an infinite wall. (In all other cases, where there is no
infinite potential, the wave function derivative must be
continuous). Here the situation is not so clear, because the potential is going to infinity, but its
width is going to zero...

Our finite square well solution left of the potential well is 

\[
  \psi_L (x) = Ce^{\frac{\kappa a}{2}} \cos \frac{ka}{2} e^{\kappa x}
  \rightarrow Ce^{\kappa\left( x + \frac{a}{2} \right) } 
\] \vspace{3px}


And because we just showed in the delta function limit that $ka \sim \sqrt{a}
\rightarrow 0$, 

\[
  \psi'_L(x) = C\kappa e^{\kappa \left( x + \frac{a}{2} \right)} \rightarrow
  C\kappa \text{ as } x \rightarrow -\frac{a}{2}
\] \vspace{3px}

And similarly on the right side, 

\[
  \psi_R(x) \rightarrow Ce^{-\kappa \left( x - \frac{a}{2} \right) } \qquad
  \psi'_R(x) = -C\kappa e^{-\kappa \left( x - \frac{a}{2} \right) } \rightarrow
  -C\kappa \text{ as } x \rightarrow \frac{a}{2}
\] \vspace{3px}

and as $\psi(x = 0) = C$, one concludes

\begin{subbox}{}
  The discontinuity of the derivative across $V_\delta = -\alpha \delta(x),
  \alpha > 0$ is 

  \begin{align} \label{delta_discon} 
    \psi'_R(0) - \psi'_L(0) = -2\kappa \psi(0) = -2
    \frac{\sqrt{2m|E|}}{\hbar}\psi(0)
  = -\frac{2m\alpha}{\hbar^2}\psi(0).
  \end{align}

  The wave functions are continuous and $ \frac{d \psi}{d x} $ is continuous
  except at points where the potential is infinite. 
\end{subbox}


\subsection{Wave Function Solution} 

Now 

\[
\kappa = \frac{\sqrt{2m|E|}}{\hbar} = \frac{m\alpha}{\hbar^2}
\] \vspace{3px}

Our wave function normalization condition is 

\[
  \int_{-\infty}^{0} |C|^2e^{2\kappa x} \, dx + \int_{0}^{\infty} |C|^2
  e^{-2\kappa x} \, dx = \frac{|C|^2}{\kappa} \quad \Rightarrow \quad
  C = \sqrt{\kappa} = \frac{\sqrt{m\alpha}}{\hbar}
\] \vspace{3px}

Putting everything together, we have a universal bound state solution for an
attractive delta function potential:

\begin{mainbox}{Delta Function Potential Solution}
  \[ V_\delta (x) = -\alpha \delta(x), \alpha > 0 \] 
  \[ |E| = \frac{m\alpha^2}{2\hbar^2} \quad \kappa = \frac{m\alpha}{\hbar^2} \]
  \[ \psi(x) = \begin{cases}
    \sqrt{\kappa}e^{\kappa x}, &\quad x < 0 \\ \sqrt{\kappa} e^{-\kappa x},
                               &\quad x > 0
  \end{cases} \qquad  \psi'(x) = \begin{cases}
    \kappa^{3/2} e^{\kappa x}, &\quad x < 0 \\ -\kappa^{3/2} e^{-\kappa x},
                               &\quad x > 0
  \end{cases} \]
\end{mainbox}

\section{Review and Outlook} 

Let's review what we have done so far 

\begin{itemize}
  \item[1.] We have studied bound states of the infinite square well, the
    harmonic oscillator, and the finite square well. In the last, the bound
    states were the states defined by $E<0$. 
  \item[2.] In these problems we sought the stationary states, the solutions of
    the time-independent Schr\"odinger equation. This required us to solve
    eigenvalue equations -- finding solutions only at certain energies.
    Consequently the energies were discrete. The eigenvalue conditions were
    somewhat different in detail, but conceptually similar: 
    \begin{itemize}
      \item[-] For the infinite square well, the wave functions were required
        to vanish at the infinite potential wall. 
      \item[-] For the bound states of the finite square well, we required
        continuity of the wave function and its derivative at the finite wall
        boundary. The eigenvalue equation came from this formula 
        \[ \frac{\psi'(x)}{\psi(x)}\Bigg|_{x = -\frac{a}{2} - \epsilon}
          \leftrightarrow \frac{\psi'(x)}{\psi(x)}\Bigg|_{x = -\frac{a}{2}
        + \epsilon} \] 
      \item[-] For the harmonic oscillator, the eigenvalue equation comes from
        demanding that the wave function not increase at large $r$ to maintain
        normalize ability. At certain energies this proved possible, yielding
        wave functions that die off as a Gaussian $e^{-\xi^2/2}$, times finite
        Hermite polynomials in $\xi$. 
    \end{itemize}
  \item[3.] The resulting stationary states are normalizable and labeled by
    a discrete index $n$. Thus the stationary states are physical states we can
    prepare in a laboratory. 
  \item[4.] We also introduced free-particle states. These stationary states
    were labeled by a continuous index $k$, extended over all space, and were
    not normalizable. These states are still useful as a basis, but a single
    stationary state is not a candidate for a real physical state. We can not
    envision ever preparing such a state. 
  \item[5.] Our finite square well gave us another example of free but
    distorted states -- states with $E>0$ that extend over all space. These
    wave functions near the well get distorted by the well. 
  \item[6.] In all cases, the (complete set) stationary states can be used to
    expand any wave packet presented to us -- a physical state. Under
    subsequent time evolution, the expansion coefficients remain, but each
    accumulates a distinct phase governed by the stationary state energy. Thus
    we obtain general solutions of the time-dependent Schr\"odinger equation.
    This is the prescription we call the \textit{prime directive}. Note that
    for the finite well the complete basis requires both bound and continuum
    states. 
\end{itemize}

This is amazing progress. However, there are a number of phenomena we know
about classically which we can examine in quantum mechanics, as well as some
phenomena with no classical analog. Phenomena without any classical analog
often involve free ($E>0$ ) but interacting states that encounter potential
wells or hills, generating phenomena like reflection and transmission, or
particle \textit{spin} -- which we will get to later. The next chapter deals
with understanding reflection and transmission, the main concepts behind
\textit{quantum tunneling}. Afterwards we begin studying QM less tightly
connected to wave functions. 


\section{Transmission and Reflection}

\subsection{Scattering off $V_\delta$} 

We can utilize a simple delta
function potential -- either a well or a barrier -- to scatter. As we have
learned, this will give us a very accurate representation of scattering of much
more complicated potentials, provided the deBroglie wavelength of the incident particle
is significantly larger than the width of the barrier. 

Consider scattering for a delta function potential at $x=0$. $E > 0$ and the
time-independent Schr\"odinger equation for $x < 0$ is 

\[
\frac{d^2 \psi(x)}{d x^2}  = -k^2 \psi(x) \qquad k = \frac{\sqrt{2mE}}{\hbar}
\quad \Rightarrow \quad \psi(x) = Ae^{ikx} + Be^{-ikx}
\] \vspace{3px}

Likewise for $x>0$, 

\[
  \psi(x) = Fe^{ikx} + Ge^{-ikx} 
\] \vspace{3px}

Continuity at $x =0$ requires 

\[
A+B = F+G
\] \vspace{3px}

And we can compute the derivative on either side of the delta function, 

\begin{align} \label{}
  \frac{d \psi}{d x} \Big|_R &= ik(F-G) \\ \frac{d \psi}{d x} \Big|_L &=
  ik(A-B)
\end{align}\vspace{3px}

So for an attractive well, combining both continuity conditions and our
calculated derivative discontinuity across the delta potential from Equation
\ref{delta_discon}, 

\begin{align} \label{}
  V_\delta &= -\alpha \delta(x) \quad \Rightarrow \quad ik(F-G-A+B)
  = -\frac{2m\alpha}{\hbar^2}\psi(0) = -\frac{2m\alpha}{\hbar^2}(A+B) \\ 
  \Rightarrow \; F - G &= A(1+2i\beta) - B(1-2i\beta) \qquad \beta \equiv
  \frac{m\alpha}{\hbar^2 k} = \frac{\alpha}{\hbar}\sqrt{\frac{m}{2E}} \quad
  \beta^2 = \frac{m\alpha^2}{2\hbar^2 E}
\end{align}\vspace{3px}

There are two constraints and four unknowns -- $A, B, F, G$, so to get
a solution we need more information. Normalization is of no help, as these are
free-particle non-normalizable states.

Recall that when we convert to time-dependent solutions, per the prime
directive, a phase  $e^{-iEt / \hbar}$ is added. When that is done, one sees 

\begin{itemize}
  \item[1.] $e^{ikx}$ generates a wave traveling with positive velocity
    (traveling to the right): it would be associated with a wave packet
    component $e^{i(kx - \omega t)}$ that travels to the right. 
  \item[2.] $e^{-ikx}$ generates a wave with negative velocity (travels to the
    left). 
\end{itemize}

So suppose we wish to describe the scattering of a wave packet coming in from
the left, off our potential. Part of the wave will be reflected, moving back to
the left. Some of it will be transmitted, going to the right, through the
potential. But there is no wave to generate at positive $x$ that is moving to
the left. So if we set our initial conditions to reflect the experiment just
described, we should require 

 \[
G=0
\] \vspace{3px}

The experiment described is depicted in Figure \ref{reftrans} below. 


\begin{figure}[H]
  \centering
    \includegraphics[width = 16cm]{reftrans.pdf}
    \caption{Scattering by a beam incident on a delta function potential from
    the left} 
    \label{reftrans}
\end{figure}


$A$ then corresponds to the amplitude of the incident wave coming in from the
left. We can then solve our two equations to find 

\begin{align} \label{}
  B = \frac{i\beta}{1 - i\beta}A \qquad F = \frac{1}{1-i\beta}A
\end{align}\vspace{3px}

which should be interpreted as the amplitude of the reflected and transmitted
wave, respectively. The probability of reflection is then 

\begin{align} \label{}
  R = \frac{|B|^2}{|A|^2} \equiv
  \frac{|\text{Amplitude}_\text{reflected}|^2}{|\text{Amplitude}_\text{incoming}
  |^2} = \frac{\beta^2}{1 + \beta^2}
\end{align}\vspace{3px}

Similarly, the probability of transmission is 

\begin{align} \label{}
  T = \frac{|F|^2}{|A|^2} \equiv \frac{|\text{Amplitude}_\text{transmitted}
  |^2}{|\text{Amplitude}_\text{incoming}|^2} = \frac{1}{1 + \beta^2}
\end{align}\vspace{3px}

You can check that $R + T = 1$. Plugging in our expression for $\beta$, 

\begin{subbox}{Transmission and Reflection Coefficients -- Delta Potential}
  \begin{align} \label{reftranscoeff}
    V_\delta = \pm \alpha \delta(x) \quad \Rightarrow \quad T = \frac{1}{1
    + \beta^2} \quad R = \frac{\beta^2}{1 + \beta^2} \quad \beta^2
    = \frac{m\alpha^2}{2\hbar^2 E}
  \end{align}\vspace{3px}
\end{subbox}

Note that we allowed for the scattering to be off a well or a \textit{barrier},
as we found that the answer does not depend on the sign of $\alpha$. We see 

\begin{itemize}
  \item[1.] As $E \rightarrow 0$, $T \rightarrow 0$, $R \rightarrow 1$. The
    wave is fully reflected. 
  \item[2.] As $E\rightarrow \infty, \; T \rightarrow 1, \; R \rightarrow 0$. The
    wave is fully transmitted. 
  \item[3.] As $\alpha \rightarrow \infty, \; T \rightarrow 0,\;  R \rightarrow 1$.
    The wave is fully reflected. 
  \item[4.] As  $\alpha \rightarrow 0,\;  T \rightarrow 1,\;  R \rightarrow 0$. The
    wave is fully transmitted.
\end{itemize}

All of this is as expected classically. However, our barrier is infinitely high
for any finite $\alpha$, and yet there is always some transmission for any
nonzero $E$. This is quantum mechanical tunneling. 


\subsection{Scattering off a Square Well} 

There are additional unique phenomena that are not captured in the
delta-potential well calculation just outlined, but do emerge from scattering
off a realistic well of equivalent strength $\alpha = aV_0$. The situation is
depicted in Figure \ref{reftranswell} below. 


\begin{figure}[H]
  \centering
    \includegraphics[width = 13cm]{reftranswell.pdf}
    \caption{Scattering by a beam incident from the left on a well of depth
    $V_0$ and width $a$.}
    \label{reftranswell}
\end{figure}


We will now repeat our previous calculations to see what new aspects arise from
details of the well, beyond its net strength $\alpha \equiv aV_0$. 

This is a slightly new problem for us as we are solving a problem involving the
free states \textit{above} the well, not bound states within. We define our
kinematic variables in the usual way, remembering by assumption that $E > 0$, 

\[
k \equiv \frac{\sqrt{2mE}}{\hbar} \quad l = \frac{\sqrt{2m(E+V_0)}}{\hbar}
\] \vspace{3px}

Because of our experience above, we can speed things along by setting $G = 0$
as before. Only a transmitted wave is allowed in Region III. 

I have previously stressed in earlier problems how much efficiency is gained by
utilizing reflection symmetry. However now we require an asymmetric solution by
setting $G = 0$. Consequently we have to deal with all three regions. The
solutions in each region are

\[
\psi(x) = \begin{cases}
  \qquad F e^{ikx} &\quad \quad x > \frac{a}{2} \\ Ce^{ilx} + D^{-ilx} &\quad -\frac{a}{2}
  < x< \frac{a}{2} \\ Ae^{ikx} + Be^{-ilx} &\quad \quad x < -\frac{a}{2}
\end{cases} 
\] \vspace{3px}

We match solutions and their derivatives at the boundary of regions II and III
($x = \frac{a}{2}$ ). We yield 

\begin{align*}
  Fe^{ika/2} &= Ce^{ila/2} + De^{-ila/2} \\ ikFe^{ika/2} &= il\left( Ce^{ila/2}
  - D^{-ila/2}\right)  
\end{align*}

As there are two constraints, $F$ determines $C$ and $D$. Doing (a lot of) the
algebra, 

\begin{align} \label{}
  C = F\left( \frac{l + k}{2l} \right) e^{i(k - l) \frac{a}{2}} \qquad
  D = F\left( \frac{l-k}{2l} \right) e^{i(k+l) \frac{a}{2}}  
\end{align}\vspace{3px}


Now we need to match the boundary of regions I and II ($x = -\frac{a}{2})$.
Let's first use the information above to write out the solutions in the two
regions. 

\[
\psi(x) = \begin{cases}
  F \left( \frac{l+k}{2l}\right) e^{ika / 2} e^{il(x - a/2)} + F \left(
  \frac{l-k}{2l}\right) e^{ika/2} e^{-il(x - a/2)} &\quad -\frac{a}{2}
  < x < \frac{a}{2} \\ \qquad \qquad \qquad \qquad Ae^{ikx} + Be^{-ikx}
                                                   &\quad \quad x < -\frac{a}{2}  
\end{cases} 
\] \vspace{3px}

The continuity of the wave function and its derivative yields 

\begin{align*}
  F \left( \frac{l+k}{2l} \right) e^{ika/2}e^{-ila} + F \left( \frac{l-k}{2l}
  \right) e^{ika /2 }e^{ila} &= Ae^{-ika/2} + Be^{ika/2}  \\
  il\left[ F \left( \frac{l+k}{2l} \right) e^{ika/2}e^{-ila} - F \left(
\frac{l-k}{2l} \right) e^{ika/2}e^{ila} \right] &= ik\left[ Ae^{-ika/2}
  - Be^{ika/2} \right]  
\end{align*}

These two constraints can be used to eliminate $A$ and $B$. Doing (again a lot
of) algebra, 

\begin{align} \label{}
  A &= F\frac{e^{ika}}{4lk} \left[ (l+k)^2 e^{-ila} - (l-k)^2 e^{ila} \right] \\
  B &= F \frac{(l+k)(l-k)}{4lk} \left[ e^{ila} - e^{-ila}\right]
\end{align}\vspace{3px}

When all the dust clears, we have a solution in which only one unknown constant
$F$ remains. This is no problem since  $F$ will cancel when calculating the
transmission and reflection ratios. 

\begin{align}
  \psi(x) = \begin{cases}
    \qquad \qquad \qquad \qquad \qquad \qquad \qquad  Fe^{ikx} &\quad \quad x > \frac{a}{2}, \\ 
    \qquad \qquad   F\frac{e^{ika/2}}{2l} \left[ (l+k)e^{il(x - a/2)} + (l-k)e^{-il(x
    - a/2)}\right] &\quad -\frac{a}{2} < x < \frac{a}{2}, \\ 
    Fe^{ikx}\frac{1}{4lk} \left[(l+k)^2 e^{i(k-l)a} + (l-k)^2e^{i(k+l)a}\right]
    + Fe^{-ikx} \frac{(l^2 - k^2)}{4lk} 2i\sin la &\quad \quad x < -\frac{a}{2}.
\end{cases} 
\end{align} \vspace{3px}


From this we identify

\begin{align}
  \text{transmitted} &\text{ wave : } &&\quad Fe^{ikx} &&& \quad x > \frac{a}{2} \\
  \text{incoming} &\text{ wave : } &&\quad Fe^{ikx} \frac{1}{4lk} \left[ (l
  + k)^2 e^{i(k-l)a} - (l-k)^2 e^{i(k+l)a} \right] &&&\quad x < -\frac{a}{2} \\
  \text{reflected} &\text{ wave : } &&\quad Fe^{-ikx} \frac{(l^2 - k^2)}{4lk}
  2i\sin la &&&\quad x < -\frac{a}{2}
\end{align}

We can now calculate the ratios of reflection and transmission. 

\begin{mainbox}{Transmission and Reflection Coefficients -- Finite Well}
  \begin{align} 
    R = \text{ Prob. of reflection } = \frac{|A_\text{reflect}
    |^2}{|A_\text{incoming} |^2} &= \frac{4(l^2 - k^2)\sin^2 la}{(4lk)^2 + 4(l^2
    -k^2)^2 \sin^2 la} \\ &= \frac{(l^2 - k^2) \sin^2 la}{(2lk)^2 + (l^2 - k^2)^2
    \sin^2 la}
  \end{align}

  \begin{align} \label{}
    T = \text{ Prob. of transm } = \frac{|A_\text{transmission}
    |^2}{|A_\text{incoming} |^2} &= \frac{(4lk)^2}{(4lk)^2 + 4(l^2 - k^2)^2
    \sin^2 la} \\ &= \frac{(2lk)^2}{(2lk)^2 + (l^2 - k^2)^2 \sin^2 la}
  \end{align}\vspace{3px}
\end{mainbox}

Below in Figure \ref{reftransplots} I plot graphs of the transmission and
reflection coefficients for square wells and delta potential wells of the same
strength $aV_0$, for two different $V_0$. 

\begin{figure}[H]
  \centering
    \includegraphics[width = 14cm]{reftransplots.pdf}
    \caption{Transmission and Reflection Comparison between Square Well and
    Delta Potential.}
    \label{reftransplots}
\end{figure}

Try plotting the coefficients for $V_0 = \frac{128\hbar^2}{ma^2}$. See how
unique the graphs become. \\ 


This concludes our introduction to the fundamental problems in Quantum
Mechanics. By now, you should have a strong grasp of the methods used to solve
the time-independent Schrödinger equation for basic potentials, providing
a solid foundation for the material to come. It's not necessary to memorize
every algebraic step or derivation for each potential discussed. Instead, focus
on developing an intuitive sense of how these problems are approached and
solved. What matters most is understanding the general procedure for solving
these problems and truly comprehending the underlying principles of each
derivation. 

Next chapter presents ways of thinking of
Quantum Mechanics less tightly connected to wave functions. We are diving
directly into Quantum theory. 
