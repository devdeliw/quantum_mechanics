We have so far focused on wave functions and their properties, developing some
intuition about wave mechanics. A few curiosities may have struck you however.
For example, we found in the harmonic oscillator that we could express the
Hamiltonian (once we chose the right variables) in terms of a dimensionless
operator 

\[
\frac{\hbar\omega}{2} \left( \hat{p}_\xi^2 + \hat{\xi}^2\right)  \quad
\rightarrow \frac{\hbar\omega}{2}\left( \hat{p}_\xi^2 + \xi^2\right)  
\] \vspace{3px}

Most textbooks, notably Griffiths, have the habit of indicating $\hat{p}$ is an
operator, but dropping the ``hat" on $\hat{x} \rightarrow x$. Why is this? At
an even more basic level, the harmonic oscillator is manifestly symmetric in
$\hat{p}$ and $\hat{x}$, yet our wave function formalism favored $x$. When we
operate on our wave function with the $\hat{x}$ operator we just get back the
value $x$. But when we act with $\hat{p}$, we have to take a derivative. Where
did we lose the $p-x$ symmetry? 


It has been lost because we choose a basis for representing our states --
a position basis. We will now review some of what we know about vectors and their representations. Take the example of a
vector $\vec{v}$ in 3D space. We are aware that this is a geometrical object, with a length and a direction, that exists independent of the basis we select to represent the vector. We will denote this abstract
vector, without a specified basis, with the following notation of Dirac -- 
$|\vec{v}\rangle$.

\section{Dirac Notation} 

Suppose we have a vector in ordinary 3D space. That vector is a geometric
object, consisting of a length and direction, independent on the basis we
choose. We could choose to \textit{represent} the vector in a specific basis.
For example, the basis could be taken to be the three Cartesian unit vectors

\[
\hat{e}_x, \; \hat{e}_y, \; \hat{e}_y \quad \Rightarrow \quad | \hat{e}_x
\rangle, \; |\hat{e}_y  \rangle, \; | \hat{e}_z \rangle     
\] \vspace{3px}

The bases we use satisfy two important conditions. The first requirement on our
basis vectors is \textit{orthonormality}, 

\[
  \langle \hat{e}_i  | \hat{e}_j \rangle = \langle i | j \rangle = \delta_{ij}
  \qquad \{i , j\} \in \{x, \; y, \; z\}
\] \vspace{3px}

That is, our basis vectors are unit vectors orthogonal to one another. The
second, implicit when we expand a vector $| \alpha \rangle$ in a basis, is
\textit{completeness}. 

\[
  \sum_{i\in\{x, y, z\}} | \hat{e}_i \langle \hat{e}_i | \rangle \equiv \sum_i
  | i \rangle\langle i | = 1
\] \vspace{3px}

We use this result when we represent a vector in terms of an orthonormal basis.
Thus

\[
|\vec{v} \rangle = \sum_i | \hat{e}_i \rangle \langle \hat{e}_i  | \vec{v}
\rangle = | \hat{e}_x \rangle v_x + | \hat{e}_y \rangle v_y + | \hat{e}_z
\rangle v_z
\] \vspace{3px}

These requirements involve both vectors, e.g., $|\alpha\rangle$, and adjoint
vectors, $\langle \alpha |$, which I explain below. 

One is free to choose any basis satisfying the conditions above, a freedom that
can be used to simplify many problems. For example, if one is working in 3D and
interested in a problem with spherical symmetry, an alternative basis that
transforms more simply under rotations is the spherical basis: 

\[
|\hat{e}_1\rangle = -\frac{1}{\sqrt{2}}(|\hat{e}_x\rangle + i|\hat{e}_y\rangle)
\qquad |\hat{e}_0\rangle = |\hat{e}_z \rangle \qquad |\hat{e}_{-1}\rangle
= \frac{1}{\sqrt{2}}(|\hat{e}_x\rangle - i|\hat{e}_y\rangle )
\] \vspace{3px}

Here the adjoint basis vectors are 

\[
\langle \hat{e}_1| = -\frac{1}{\sqrt{2}}(\langle \hat{e}_x| - i\langle \hat{e}_y|
\qquad \langle \hat{e}_0| = \langle \hat{e}_z | \qquad \langle \hat{e}_{-1} |
= \frac{1}{\sqrt{2}}(\langle \hat{e}_x| + i\langle \hat{e}_y| )
\] \vspace{3px}

That is, $\langle \alpha |$ is the \textit{conjugate transpose}  of $| \alpha
\rangle$. For this basis one can readily verify orthonormality and
completeness. Consequently we can also represent a vector in 3D in a spherical
basis 

\[
| \vec{v} \rangle = \sum_i |\hat{e}_i\rangle \langle \hat{e}_i  | \vec{v}
\rangle = | \hat{e}_{-1} \rangle v_{-1} + | \hat{e}_0 \rangle v_0 + | \hat{e}_1
\rangle v_1
\] \vspace{3px}

The discussion above can be generalized beyond simple vectors in 3D, as we can
represent any vector $|\alpha\rangle$ of finite dimension $N$ similarly, 

\[
| \alpha \rangle = \sum_{i=1}^{N} | i \rangle \langle i  | \alpha \rangle
= \sum_{i=1}^{N} \alpha_i |i\rangle
\] \vspace{3px}
 
in terms of a chosen $N$-dimensional set of unit basis vectors. Here $\langle
i | \alpha \rangle$ is a scalar, the inner product defined below. The adjoint
(conjugate transpose) of a vector is defined by 

\[
\langle \alpha | \equiv | \alpha \rangle^\dagger = \sum_{i=1}^{N} \langle
\alpha | i \rangle \langle i \rangle = \sum_{i=1}^{N}  \alpha_i^* \langle
i\rangle 
\] \vspace{3px}

That is, the inner product satisfies 

\[
\langle i | \alpha \rangle^* = \langle \alpha | i \rangle. 
\] \vspace{3px}


We can think of $| \alpha \rangle$ represented in a basis as a column vector
whose entries are the expansion coefficients in that basis

\[
| \alpha  \rangle = \begin{pmatrix}
  \alpha_1 \\ \alpha_2 \\ \vdots \\ \alpha_N
\end{pmatrix}
\] \vspace{3px}

Correspondingly, the adjoint vector can be viewed as a row matrix in that
basis, 

\[
\langle \alpha | = (\alpha_1^*, \; \alpha_2^*, \; \cdots, \; \alpha_N^* )
\] \vspace{3px}

\section{Infinite Dimensional Representations and Wave Functions}

On several occasions (infinite square well, harmonic oscillator) we have
expanded initial wave packets over stationary states that were discrete (like
above) but infinite dimensional (unlike above), obtaining an expression for the
wave function that could be used with the prime directive to determine the time
evolution of the wave packet. 

\[
\Psi(x, t = 0) = \sum_{i=1}^{\infty} c_i \psi_i(x) \quad \text{ where } \quad
c_i = \int_\Omega \psi_i^*(y) \Psi(y, t = 0) \, dy
\] \vspace{3px}

where $\Omega$ is the relevant domain). Let's assume we have a problem like
the harmonic oscillator where the wave functions extend over all space
($\Omega \equiv -\infty < x < \infty$ ). In this case, 

\[
\Psi(x, t=0) = \sum_{i=1}^{\infty} c_i\psi(x) \quad \text{where} \quad c_i
\int_{-\infty}^{\infty} \psi_i^*(y) \Psi(y, t=0)\, dy
\] \vspace{3px}

where $\psi_i(y)$ are the stationary state wave functions. 

Can we extract this result from the more abstract Dirac-formulation described
just above? Consider

\begin{align} \label{formintro}
|\Psi(t = 0)\rangle = \sum_{i=1}^{\infty} |\psi_i\rangle \langle \psi_i
| \Psi(t=0) \rangle 
\end{align} \vspace{3px}

We introduce the complete set of normalized position states (postponing
additional discussion until later). As position is a continuous quantum number, 

\[
1 = \sum_{i}^{}  |\hat{e}_i\rangle \langle \hat{e}_i | \quad \Rightarrow \quad
  1 = \int_{-\infty}^{\infty} |y\rangle \langle y | \, dy
\] \vspace{3px}

Consequently, 

\begin{align} \label{formintro_2}
|\Psi(t = 0)\rangle = \int_{-\infty}^{\infty} |y\rangle \langle
y | \Psi(t=0)\rangle
\end{align} \vspace{3px}

Inserting this into Equation \ref{formintro} then yields 

\[
|\Psi(t = 0)\rangle = \sum_{i=1}^{\infty} |\psi_i\rangle
\int_{-\infty}^{\infty} \langle \psi_i | y \rangle \langle y | \Psi(t
= 0)\rangle \, dy
\] \vspace{3px}

Finally we take the scalar product with $\langle x |$, 

\begin{align} \label{formintro_3}
\langle x | \Psi(t = 0) \rangle = \sum_{i=1}^{\infty}  \langle
x | \psi_i\rangle \int_{-\infty}^{\infty} \langle \psi_i | y\rangle \langle
y | \Psi(t = 0)\rangle \, dy
\end{align} \vspace{3px}

Now if we define the wave function as the quantity 

\[
\Psi(x, t= 0) = \langle x | \Psi(t = 0)\rangle
\] \vspace{3px}

and likewise with the other quantities appearing in Equations \ref{formintro_2}
and \ref{formintro_3}, those equations become 

\[
  (\ref{formintro_2}') \quad |\Psi(t = 0)\rangle = \int_{-\infty}^{\infty}
  \Psi(y, t) | y \rangle\, dy \qquad (\ref{formintro_3}') \quad |\Psi(x,
  t=0)\rangle = \sum_{i=1}^{\infty} \psi_i(x) \int_{-\infty}^{\infty}
  \psi_i^*(y) \Psi(y, t=0) \, dy  
\] \vspace{3px}

Equation \ref{formintro_3}' is just the expression we have been using up to
this point, where we expand the initial wave packet in terms of
stationary-state wave functions, prior to employing the prime directive.
Equation \ref{formintro_2}' identifies the wave function $\psi(x)$ as the
coefficients of the expansion of the state $| \psi \rangle$ in terms of the
position basis functions $|x\rangle$. In other words, wave functions are
expansion coefficients, analogous to the $c_i$ that arose in the discrete case
we described first. 

The notion of expansions and wave functions lead to an interpretation of the
wave function as a particular representation of a more basic and more object,
abstract state -- Dirac's ``ket", which he denoted as $|\Psi\rangle$. 

\subsection{Summary}

We now recognize wave functions are just a particular representation of a more
fundamental quantity, the state vector $|\Psi\rangle$, that exists independent
of the representation, just as an ordinary 3D vector exists independent of any
coordinate basis. However, one always has the freedom to select a basis, to
represent both state vectors and their adjoins. In Dirac's notation, 

\begin{mainbox}{Dirac Notation}
  \[
  \text{Dirac's Ket: } |\Psi\rangle  \quad \text{ representation in a basis
  $i$: } |\Psi\rangle \equiv \sum_{i}^{} | i \rangle \langle i | \Psi \rangle
  \equiv \sum_{i}^{} |i\rangle c_i
  \] \[ \text{Dirac's Bra: } \langle \Psi | \quad \text{ representation in
  a basis $i$: } \langle \Psi | \equiv \sum_{i}^{} \langle\Psi | i\rangle
  \langle i | \equiv \sum_{i}^{} \langle i | c_i^*  \] \vspace{3px}

  If the basis is described by a continuous index rather than a discrete one
  like  $i$, the sum is replaced with an integral.
\end{mainbox}



Up to this point we have repeatedly dealt with states of infinite dimension,
representing them by taking an integral over the position eigenstates, 

\[
\sum_{i}^{} |i\rangle \langle i | \equiv 1 \quad \Rightarrow \quad \int
|x\rangle \langle x | \, dx = 1
\] \vspace{3px}

We used the position eigenstate representation and an inner
product defined over the vector space of functions, 

\begin{align*}
  \delta_{\beta a} = \langle \psi_\beta | \psi_\alpha \rangle \Rightarrow \int
  \langle \psi_\beta | x \rangle \langle x | \psi_\alpha \rangle \, dx = \int
  \langle x|\psi_\beta\rangle^* \langle x | \psi_\alpha \rangle \, dx = \int
  \psi_\beta^* (x) \psi_\alpha (x) \, dx 
\end{align*} \vspace{3px}


\section{Continuous Spectra, Momentum, and Position Wave Functions}

The \textit{spectrum} of an operator is defined as the set of possible outcomes
after measurement. 

If the spectrum of a Hamiltonian, or any other quantum mechanical operator is
continuous, the eigenfunctions are not normalizable -- not part of the
``Hilbert Space" we will describe soon. For example, in the infinite square
well and harmonic oscillator, we found the resultant eigenfunctions / wave
functions to be discrete in energy, following 

\[
\text{Infinite Square Well: } E_n = \frac{\hbar^2n^2\pi^2}{2ma^2} \qquad
\text{Harmonic Oscillator: } E_n = \left( n + \frac{1}{2} \right) \hbar\omega   
\] \vspace{3px}

If you refer back to Figures \ref{graphsoln} and \ref{fig:infsquarewell} you
will see the spectrum of the Hamiltonian is discrete -- only certain wave
functions are allowed at specific energies. Because the set of wave functions,
the spectrum, is not continuous, the wave functions are normalizable. 

\subsection{Momentum-Position Symmetry} 

Now that we understand the Dirac formulation of Quantum mechanics, let's return
to our question at the beginning of this chapter -- why the momentum operator
$\hat{p}$ requires a derivative, yet the position operator $\hat{x} \rightarrow
x$ just returns $x$. I answered that the reason for this is our choice of
representing our quantum states with the position basis -- $\psi(x)$. Let's
try to understand this more closely. 


In the \textit{position basis}, the position operator $\hat{x}$ acting on a position
eigenstate $|x\rangle$ yields 

\[
\hat{x}|x\rangle = x |x\rangle
\] \vspace{3px}

Because  $p \leftrightarrow x$ are manifestly symmetric, in the
\textit{momentum basis}, the momentum operator $\hat{p}$ acting on a momentum
eigenstate $|p\rangle$ yields

\[
\hat{p}|p\rangle = p|p\rangle
\] \vspace{3px}

However, in the position basis, from our Dirac formulation, this equation can
also be written as 

\[
\hat{p} \langle x | p \rangle = p\langle x|p \rangle 
\] \vspace{3px}

Here, $\langle x|p \rangle $ is the wave function of the momentum eigenstate
$|p\rangle$, in the position basis. Substituting in the momentum operator
$\hat{p} = -i\hbar \frac{d }{d x} $ into the above equation, 

\[
-i\hbar \frac{d }{d x} \langle x|p \rangle = p\langle x|p \rangle 
\] \vspace{3px}

This is a first-order differential equation. We separate the variables, 

\begin{align}
  \frac{d }{d x} \langle x|p \rangle &= \frac{ip}{\hbar}\langle x |p \rangle
  \\
  \frac{1}{\langle x|p \rangle } \frac{d }{d x}  \langle x|p \rangle
                                     &= \frac{ip}{\hbar}
\end{align} \vspace{3px}

Integrating, 

\[
\int \frac{1}{\langle x|p \rangle } \frac{d \langle x|p \rangle }{d x} \, dx
= \int \frac{ip}{\hbar} \, dx 
\] \vspace{3px}

Finally, 

\begin{align} \label{}
  \ln \langle x|p \rangle &= \frac{ipx}{\hbar} + C \\ 
  \langle x|p \rangle &= Ae^{\frac{ipx}{\hbar}}
\end{align}\vspace{3px}


The expression $\langle x|p \rangle = Ae^{ipx / \hbar}$ thus represents the
wave function of the momentum eigenstate in the position basis. To solve for $A$,
we normalize, 

\[
\langle p'|p \rangle = \int_{-\infty}^{\infty} \langle p'|x \rangle \langle x|p
\rangle \, dx = A^2 \int_{-\infty}^{\infty} e^{i(p-p')x / \hbar} \, dx
= |A^2|2\pi\hbar \delta(p-p')
\] \vspace{3px}

Therefore to allow $\langle p' | p \rangle = \delta (p' - p)$, i.e. satisfying
orthonormality (both orthogonality and normalization), $A
= \frac{1}{\sqrt{2\pi\hbar}}$. Thus 

\begin{subbox}{}
  \[
  \langle x|p \rangle = \frac{1}{\sqrt{2\pi\hbar}} \quad \text{ then } \quad
  \langle p' | p \rangle = \delta(p' - p)
  \] 
\end{subbox}

Griffiths calls this \textit{Dirac-Delta Orthonormality}. These wave functions
form a complete basis that can expand any normalizable state vector. We note  

\begin{align} \label{}
  \int_{-\infty}^{\infty} | p \rangle \langle p | \, dp
  = \int_{-\infty}^{\infty} &dp \int_{-\infty}^{\infty}  dx
  \int_{-\infty}^{\infty} |x\rangle \langle x|p\rangle \langle p | x'\rangle
  \langle x'| \, dx ' = \frac{1}{2\pi\hbar} \int_{-\infty}^{\infty} dp
  \int_{-\infty}^{\infty} dx \int_{-\infty}^{\infty} \,|x\rangle
  \langle x'| e^{ip(x - x') / \hbar}\, dx' \\ 
                               &= \int_{-\infty}^{\infty} dx
                               \int_{-\infty}^{\infty} |x\rangle \langle
                               x'| \delta(x-x') \, dx' = \int_{-\infty}^{\infty}
                               |x\rangle \langle x| \, dx = 1
\end{align}\vspace{3px}

Thus we can use these momentum eigenstates just as we use the position
eigenstates. Given a state vector $|\alpha\rangle$, we can insert a complete
set to determine 

\[
|\alpha\rangle = \int_{-\infty}^{\infty} |p\rangle \langle p |\alpha\rangle \,
dp = \int_{-\infty}^{\infty} |p\rangle \psi_\alpha (p)
\] \vspace{3px}

Then taking the inner product with $\langle x |$, 

 \[
\psi_\alpha(x) \equiv \langle x | \alpha \rangle = \int_{-\infty}^{\infty}
\langle x | p\rangle \langle p |\alpha \rangle \, dp
= \frac{1}{\sqrt{2\pi\hbar}} \int_{-\infty}^{\infty} e^{ipx / \hbar}
\psi_\alpha(p) \, dp
\] \vspace{3px}

And conversely, 

\[
\psi_\alpha(p) = \langle p | \alpha \rangle = \int_{-\infty}^{\infty}  \langle
p|x\rangle \langle x|\alpha \rangle \, dx = \frac{1}{\sqrt{2\pi\hbar}}
\int_{-\infty}^{\infty} e^{-ipx / \hbar} \psi_\alpha(x) \, dx
\] \vspace{3px}

The momentum wave function is the Fourier transform of the position wave
function! Hence the symmetry. 


This raises a question we have not yet answered -- what is the \textit{wave
function} of a position state? That is, we have a position ket $| x_0 \rangle$
with 

\[
\hat{x} |x_0\rangle = x_0|x_0\rangle
\] \vspace{3px}

So inserting a complete set of position states, 

\[
|x_0\rangle = \int_{-\infty}^{\infty}  |x\rangle \langle x|x_0 \rangle \, dx
\equiv \int_{-\infty}^{\infty} |x\rangle \psi_{x_0}(x) \, dx \quad \Rightarrow
\quad \psi_{x_0} = \delta(x - x_0)
\] \vspace{3px}

\subsection{Example Problem --  Schr\"odinger equation to the wave equation} 

We have learned that the Schr\"odinger equation wave functions that we have
focused on thus far of are a particular representation -- in position space --
of a \textit{more fundamental} equation that can be written out in terms of an
abstract state vector and operators acting on it. Let's try to go from the
abstract equation to the wave equation. We start with an elementary Hamiltonian
energy eigenvalue equation: 

\[
  \left[ -\frac{\hat{p}^2}{2m} + \hat{V}(\hat{x}) \right] | \alpha \rangle
  = E|\alpha\rangle 
\] \vspace{3px}

We take the inner product with $\langle x |$, 

\[
  \langle x | \left[ -\frac{\hat{p}^2}{2m} + \hat{V}(\hat{x}) \right] |\alpha
  \rangle = E \langle x|\alpha \rangle = E\psi_\alpha (x) 
\] \vspace{3px}


So far so good, we have the wave function on the right. Now  

\[
\langle x | \hat{V}(x) | \alpha \rangle = \int_{-\infty}^{\infty} \langle
x | \hat{V}(\hat{x}) | x' \rangle \langle x' | \alpha \rangle \, dx'
= \int_{-\infty}^{\infty} V(\hat{x'}) \langle x | x' \rangle \langle x'
| \alpha \rangle \, dx' = \int_{-\infty}^{\infty}
V(x')\delta(x-x')\psi_\alpha(x') \, dx' = V(x) \psi_\alpha(x)
\] \vspace{3px}

In the next section, where we discuss \textit{Hermitian} operators, we will be
able to obtain the same result more efficiently. $\hat{V}$ is Hermitian, and
such operators can be moved from the ket to the bra in the following way, 

\[
\langle x | \hat{V}(\hat{x})| \alpha \rangle = \langle \hat{V}(\hat{x})x
| \alpha \rangle = V(x) \langle x|\alpha \rangle  = V(x) \psi_\alpha(x)
\] \vspace{3px}

obtaining the same result as above. And for $\hat{p}^2$, 

\begin{align} \label{}
  \langle x | \hat{p}^2 | \alpha \rangle &= \int_{-\infty}^{\infty} \langle
  x | \hat{p}^2 | p \rangle \langle p | \alpha \rangle \, dp
  = \int_{-\infty}^{\infty}  p^2 \langle x | p\rangle \langle p | \alpha
  \rangle\, dp = \frac{1}{\sqrt{2\pi\hbar}} \int_{-\infty}^{\infty}
  p^2e^{-ipx/\hbar} \langle p | \alpha \rangle \, dp \\
  &= \frac{1}{\sqrt{2\pi\hbar}} \int_{-\infty}^{\infty} \left( -\hbar^2
  \frac{d^2 }{d x^2}  \right) e^{ipx/\hbar} \langle p |\alpha \rangle \, dp
  = \int_{-\infty}^{\infty} \left( -\hbar^2 \frac{d^2 }{d x^2}  \right) \langle
  x | p \rangle \langle p |\alpha \rangle \, dp = \left( -\hbar^2 \frac{d^2 }{d
x^2}  \right) \langle x | \alpha \rangle \\ &= \left( -\hbar^2 \frac{d^2 }{d
x^2}  \right)  \psi_\alpha(x)
\end{align}\vspace{3px}

Putting the pieces together, we get our familiar position-space wave function
equation

\[
  \left[ -\frac{\hbar^2}{2m} \frac{d^2 }{d x^2}  + V(x) \right] \psi_\alpha(x)
  = E\psi_\alpha(x)
\] \vspace{3px}


\section{Hilbert Spaces, Hermitian Operators}


Here, we review the properties of vector spaces and scalar products, connecting
them to function spaces and integration. These concepts allow us to extract
scalars -- representing amplitudes and observables -- from our
infinite-dimensional vectors (functions).

\subsection{Hilbert Spaces} 

The Hilbert spaces in which quantum mechanical state vectors and operators
reside are vector spaces to which an inner product has been added, one with
specific properties connecting to our quantum mechanical requirements that
$|\psi(x)|^2$ be normalizable. 

\subsubsection{Inner Product Spaces} 

Inner product spaces are vector spaces with an inner product defined on them.
The inner product of two vectors $| \alpha \rangle$ and $| \beta \rangle$ is
a mapping from these vectors to a scalar in the scalar field of the vector
space, $\langle u|v \rangle = a$. In our applications this is a complex scalar.
The inner product (or scalar product) has the following properties: 

\begin{itemize}
  \item[1.] $\langle \gamma | \alpha + \beta \rangle = \langle \gamma | \alpha
    \rangle + \langle \gamma | \beta \rangle$ 
  \item[2.]  $\langle \beta | a \alpha \rangle = a \langle \beta | \alpha
    \rangle$
  \item[3.] $\langle \alpha | \alpha \rangle \geq 0$ 
  \item[4.] $\langle \alpha | \beta \rangle^* = \langle \beta | \alpha \rangle$
\end{itemize}

Note that 

\[ \langle \alpha + \beta | \gamma \rangle = \langle \gamma | \alpha + \beta
  \rangle^* = \langle\gamma | \alpha\rangle^* + \langle \gamma | \beta\rangle^*
= \langle \alpha | \gamma \rangle + \langle \beta | \gamma \rangle \] 
\[ \langle a \alpha | \beta \rangle = \langle \beta | a \alpha \rangle^* = (a
\langle \beta | \alpha \rangle)^* = a^* \langle \alpha | \beta\rangle \] 
\[ | \langle \alpha | \beta \rangle | ^2 \leq \langle \alpha | \alpha \rangle
\langle \beta | \beta \rangle \]


The last result, the triangle inequality, is familiar from ordinary vectors in
geometry, 

\[
|\vec{a} \cdot \vec{b}| = |ab \cos \theta| \leq ab \quad \Rightarrow | \langle
a | b\rangle | ^2 \leq \langle a | a \rangle \langle  b | b \rangle
\] \vspace{3px}

The triangle inequality takes the form of the Schwarz inequality the
infinite-dimensional vector spaces use in quantum mechanics 

\[
|\langle \alpha | \beta \rangle |^2 \leq \langle \alpha | \alpha \rangle
\langle \beta | \beta \rangle \rightarrow \Bigg| \int_{a}^{b} \psi_2^*(x)
\psi_1(x)  \, dx \Bigg|^2 \leq \int_{a}^{b} |\psi_1(x)|^2 \, dx \int_{a}^{b}
|\psi_2(x)|^2 \, dx
\] \vspace{3px}


The normalizable stationary states we have derived form complete basis in an
infinite dimensional complex vector space with these properties. Such spaces
are called \textit{Hilbert spaces}. Specifically, our vector space of states
includes only those whose wave functions are \textit{square integrable}. Thus
this choice of functions (our vectors describing physical states) along with
the complex numbers (our scalar field) and the inner product defined as above,
constitutes our Hilbert space. 


\subsection{Observables as expectations of Hermitian operators}

We want to define the properties of operators whose expectation values
correspond to possible quantum mechanical measurements. We note operators with
their ``hats" $\hat{Q}$. Operators act on states to produce new states: 

\[
\hat{Q}| \alpha \rangle = | \hat{Q} \alpha \rangle \equiv | \beta \rangle
\] \vspace{3px}

Here an operator $\hat{Q}$ acts on a state $| \alpha \rangle$ to produce a new
state $| \hat{Q}\alpha \rangle$. If we want we can give this new state
a different name -- we call it $| \beta \rangle$. 

We have discussed how abstract states can be represented in a basis that we
select, by using completeness of this basis, $\sum_i | i \rangle \langle
i | = 1$

\begin{align} \label{vecbasis}
|\alpha \rangle = \sum_{i}^{} | i \rangle \langle i | \alpha \rangle \equiv
\sum_{i}^{} |i\rangle \alpha_i \quad \text{ e.g. } \quad |\alpha \rangle = \int
|x\rangle \langle x | \alpha \rangle \, dx \equiv \int |x\rangle \psi_\alpha(x)
\, dx 
\end{align} \vspace{3px}

The coefficients of the basis vectors $|i\rangle$ in the expansion are numbers
generated from the inner product for our vector space -- $\langle i | \alpha
\rangle$. If the basis we adopt is the position eigenstates, then $\langle
| \alpha \rangle = \psi_\alpha(x)$, the wave function value at that $x$. 

In exactly the same way, we can pick a basis in which to represent operators 

\[
\hat{Q} = \sum_{i}^{} \sum_{j}^{} |i\rangle \langle i | \hat{Q} | j \rangle
\langle j | \equiv \sum_{i}^{} \sum_{j}^{} |i\rangle \langle j | Q_{ij}
\] \vspace{3px}

The $Q_{ij} \equiv \langle i | \hat{Q} | j\rangle$ are just numbers generated
from taking the inner product of vectors 

\[
  Q_{ij} = \langle i | \hat{Q} | j\rangle = \langle i | \hat{Q}j \rangle 
\] \vspace{3px}

and are the expansion coefficients of our operator $\hat{Q}$ in terms of the
basis operators $|i\rangle \langle j |$, exactly in the way that $\alpha_i
\equiv \langle i | \alpha \rangle$ in Equation \ref{vecbasis} are the expansion
coefficients of the state vector $|\alpha \rangle$ in terms of the basis
vectors $|i\rangle$. The quantity $\langle i | \hat{Q} | j\rangle $ is called
the \textit{matrix element} of $\hat{Q}$ in the basis introduced above. 


Frequently in quantum mechanics we are asked to solve an eigenvalue/eigenvector
problem, 

\begin{align} \label{eigenvalproblem}
  \hat{Q}|\alpha\rangle = Q_\alpha |\alpha\rangle \quad \text{ e.g. } \quad
  \hat{H}_{SE}|\alpha\rangle = \left[ \frac{\hat{p}^2}{2m}
  + \hat{V}(\hat{x})\right] |\alpha\rangle = E_\alpha |\alpha\rangle
\end{align}\vspace{3px}

Up to this point, $\hat{Q}$ has been our Schr\"odinger Hamiltonian
$\hat{H}_{SE}$ for which we were asked to find the eigen-energies and
eigenvector wave functions. We generally did this working in a specific
representation -- the position basis. But as we noted with the harmonic
oscillator, underlying this is the abstract, basis-independent form of the
Schr\"odinger equation of Equation \ref{eigenvalproblem}. 

While we solved the harmonic oscillator in its abstract form, via raising and
lowering operators, in all other cases we solved the eigenvalue/eigenfunction
problems using a basis. Let's go through the steps of representing the operator
Equation \ref{eigenvalproblem} above in terms of a basis. Since this reduces an
abstract problem into one that involves only numbers, we can solve using
differential equations, or a computer. We assume a finite Hilbert space of
dimension $N$ to make visualization easier. 

\begin{align} \label{}
  \hat{Q}|\alpha \rangle &= Q_\alpha |\alpha \rangle \; \Rightarrow \\ 
  \sum_{i}^{}  \sum_{j}^{} |i\rangle \langle i | \hat{Q} | j \rangle \langle
  j | \alpha \rangle &= Q_\alpha \sum_{k}^{} |k\rangle \langle k | \alpha
  \rangle \; \Rightarrow \\ 
  \sum_{i}^{} \sum_{j}^{} \langle m | i \rangle \langle i | \hat{Q} | j \rangle
  \langle j | \alpha \rangle &= Q_\alpha \sum_{k}^{} \langle m | k \rangle
  \langle k | \alpha \rangle \; \Rightarrow \\ 
  \sum_{j}^{} \langle m | \hat{Q} | j \rangle \langle j | \alpha \rangle &=
  Q_\alpha \langle m | \alpha \rangle \label{operatorbasis}
\end{align}\vspace{3px}

In the third line above we took the inner product with a specific adjoint basis
state $\langle m |, m \in \{1, 2, \hdots, N\}$ so we have  $N$ choices for that
state. Everything appearing in these equations is a number. We can write out
these $N$ equations

\[
\begin{pmatrix}
  Q_{11} & Q_{12} & \cdots Q_{1N} \\
  Q_{21} & Q_{22} & \cdots & Q_{2N} \\ 
  \vdots & \vdots & \vdots & \vdots \\ 
  Q_{N1} & Q_{N2} & \cdots & Q_{NN}  
\end{pmatrix} \begin{pmatrix}
  \alpha_1 \\ \alpha_2 \\ \vdots \\ \alpha_N 
\end{pmatrix} = Q_\alpha \begin{pmatrix}
  \alpha_1 \\ \alpha_2 \\ \vdots \alpha_N
\end{pmatrix} 
\] \vspace{3px}

This is the familiar matrix form of an eigenvalue problem. 

In other cases, we are simply acting on a state with an operator, not
necessarily looking for the operator's eigenvalues. We represented this
abstractly by 

\[
\hat{Q}|\alpha \rangle = |\hat{Q}\alpha \rangle \equiv |\beta\rangle
\] \vspace{3px}

The new vector generated, $| \beta \rangle$ in general would not be
proportional to $|\alpha\rangle$. If we now adopt a representation, this
becomes 

\[
\sum_{i}^{}  \sum_{j}^{} |i\rangle \langle i | \hat{Q} | j \rangle \langle
j | \alpha \rangle = \sum_{k}^{} |k\rangle \langle k | \beta \rangle
\] \vspace{3px}

and taking the inner product with $\langle m |, m \in \{1, 2, \hdots, N\}$
yields $N$ equations of the form 

\begin{align}
\sum_{j}^{} \langle m | \hat{Q} | j \rangle \langle j | \alpha \rangle
= \langle m | \beta \rangle 
\end{align} \vspace{3px}

The matrix/vector form of this is 


\[
\hat{Q}| \alpha \rangle \leftrightarrow \begin{pmatrix}
  Q_{11} & Q_{12} & \cdots & Q_{1N} \\ Q_{21} & Q_{22} & \cdots & Q_{2N} \\
  \vdots & \vdots & \vdots & \vdots \\ Q_{N1} & Q_{N2} & \cdots & Q_{NN}
\end{pmatrix} \begin{pmatrix}
  a_1 \\ a_2 \\ \vdots \\ a_N
\end{pmatrix} = \begin{pmatrix}
  b_1 \\ b_2 \\ \vdots \\ b_N
\end{pmatrix}  \leftrightarrow |\hat{Q} \alpha\rangle \quad \text{where} \quad
b_i = \sum_{j=1}^{N} Q_{ij}a_j
\] \vspace{3px}

\subsubsection{Infinite Dimensional Spaces}

Although we used a finite-dimensional vector space to illustrate the operator
algebra, the results extend to the infinite dimensional case. For example, if
we specialize Equation \ref{operatorbasis} to the Hamiltonian and use the
position basis, 

\begin{align} \label{infdimeq}
  \sum_{j}^{} \langle m | \hat{Q} | j\rangle \langle j | \alpha \rangle
  = Q_\alpha \langle m | \alpha \rangle \Rightarrow \int \langle x | \left[
  \frac{\hat{p}^2}{2m} + \hat{V}(\hat{x}) \right] |x'\rangle \psi_\alpha(x')
  = E_\alpha \psi_\alpha(x)
\end{align}\vspace{3px}

Now we showed previously that 

\[
\langle x | \hat{V}(\hat{x}) | x'\rangle = V(x')\delta(x-x')
\] \vspace{3px}

while 

\begin{align} \label{}
  \langle x | \hat{p}^2 | x' \rangle &= \int_{-\infty}^{\infty} \langle
  x | \hat{p}^2 | p \rangle \langle p | x' \rangle \, dp
  = \int_{-\infty}^{\infty} p^2 \langle x | p \rangle \langle p | x' \rangle \,
  dp = \int_{-\infty}^{\infty} p^2 \frac{1}{\sqrt{2\pi\hbar}}e^{ipx / \hbar}
  \langle p | x'\rangle \, dp \\ 
  &= -\hbar^2 \frac{d^2}{d x^2} \int_{-\infty}^{\infty}
  \frac{1}{\sqrt{2\pi\hbar}} e^{ipx / \hbar} \langle p | x' \rangle \, dp
  = -\hbar^2 \frac{d^2 }{d x^2} \int_{-\infty}^{\infty} \langle x|p \rangle
  \langle p|x' \rangle \, dp = -\hbar^2 \frac{d^2 }{d x^2} \langle x|x' \rangle
  \\ &= -\hbar^2 \frac{d^2 }{d x^2} \delta(x-x')
\end{align}\vspace{3px}


Plugging in these two results into Equation \ref{infdimeq} and using the delta
function to do the integral over $x'$ yields the familiar result: 

\[
  \left[ -\frac{\hbar^2}{2m} \frac{d^2 }{d x^2} + V(x) \right] \psi_\alpha(x)
  = E_\alpha \psi_\alpha(x)
\] \vspace{3px}

\subsubsection{Adjoint of matrices and Hermitian Operators} 

The adjoint of a matrix is obtained by taking is transpose, then complex
conjugating. So in a representation, 

\[
  [\hat{Q}^\dagger]_{ij} = [\hat{Q}]_{ji}^* = Q_{ji}^*
\] \vspace{3px}


A vector $| \alpha \rangle$ in a representation is just a matrix with one
column, with components $[|\alpha\rangle]_{i1}$. So the adjoint vector $\langle
\alpha |$ has components of 

\[
  [\langle \alpha | ]_{1i} \leftrightarrow \langle \alpha | i\rangle = \langle
  i | \alpha \rangle^* = a_i^* \leftrightarrow [|\alpha\rangle ]^*_{i1}
\] \vspace{3px}

That is, vectors and their adjoints in a representation are related by 

\[
|\alpha \rangle: \qquad \begin{pmatrix}
  a_1 \\ a_2 \\ \vdots \\ a_N
\end{pmatrix}  \leftrightarrow (a_1^*, \; a_2^*\, \hdots, \; a_N^*) \qquad
:\langle \alpha |
\] \vspace{3px}

Now if we take the conjugate of Equation \ref{operatorbasis}, 

\[
  \sum_{j=1}^{N} Q^*_{mj} a^*_j = b^*_m \quad \sum_{j=1}^{N} [\langle \alpha
  | ]_{1j}[Q^\dagger ]_{jm} = [\langle \beta | ]_{1m}  
\] \vspace{3px}

So we see 

\[
  [\hat{Q}|\alpha \rangle ]^\dagger = |\hat{Q}\alpha \rangle^\dagger
  = \langle \hat{Q} \alpha | = \langle a | Q^\dagger 
\] \vspace{3px}

In quantum mechanics the expectation of a measurement should be real, as it
must correspond to an observable. Thus we should require 

\[
\langle \alpha | \hat{Q} | \alpha \rangle = \langle \alpha | \hat{Q} | \alpha
\rangle^* \quad \Leftrightarrow \quad \langle \alpha | \hat{Q} | \alpha \rangle
= \langle \alpha | \hat{Q}^\dagger | \alpha \rangle \quad \Leftrightarrow \quad
\langle \alpha | \hat{Q} \alpha \rangle = \langle \hat{Q} \alpha
| \alpha\rangle
\] \vspace{3px}

In a representation, Hermitian operators correspond to matrices for which
$[\hat{Q}^\dagger]_{ij} = [\hat{Q}]_{ij}$ -- the matrix is equal to its
adjoint. Operators that satisfy the above conditions for all $|\alpha\rangle$
are called Hermitian. 


\begin{subbox}{Hermitian Operators}
  \[ \hat{Q} = \hat{Q}^\dagger \] \vspace{3px} 

  Observables are represented by Hermitian operators -- real, possible
  measurements: $\langle \alpha | \hat{Q} \alpha \rangle = \langle \hat{Q}
  \alpha | \alpha \rangle$
\end{subbox}


For example, the momentum operator is Hermitian, which we show via partial
integration, 

\begin{align} \label{}
  \langle \psi | \hat{p}\psi \rangle &= \int \langle \psi|x \rangle \langle
  x | \hat{p}\psi \rangle \, dx = \int \psi^*(x) \frac{\hbar}{i} \frac{d }{d x}
  \psi(x) \, dx = \int \left[ -\frac{\hbar}{i} \frac{d }{d x} \psi^*(x) \right]
  \psi(x)\, dx \\ 
                                     & = \int \left[\frac{\hbar}{i} \frac{d }{d
                                     x} \psi(x) \right]^*\psi(x) \, dx = \int
                                     [\langle x | \hat{p} \psi \rangle ]^*
                                     \langle x|\Psi \rangle \, dx = \int
                                     \langle \hat{p} \psi | x \rangle
                                     \langle x | \Psi\rangle \, dx = \langle
                                     \hat{p}\psi | \psi \rangle
\end{align}\vspace{3px}

But the operator $ \frac{d }{d x} $ would not be. 




\subsection{Determinate States of Hermitian Operators} 

The Hamiltonians we have studied thus far have stationary states of definite
energy. If you measure the energy of a stationary state you always get back the
same value. $|\alpha\rangle$ is said to be a determinate state of an operator
$\hat{Q}$ if every measurement 

\[
\langle \alpha | \hat{Q} | \alpha \rangle  = Q
\] \vspace{3px}

returns the same value $Q$. In this case, 

\[
0 = \sigma^2 = \langle \alpha | \hat{Q}^2 | \alpha \rangle - \langle \alpha
  | \hat{Q} | \alpha \rangle ^2 = \langle \alpha | (\hat{Q} - Q)^2 | \alpha
  \rangle = \langle (\hat{Q} - Q)\alpha | (\hat{Q} - Q)\alpha \rangle  
\] \vspace{3px}

But this can only be if 

\begin{mainbox}{}
  \[
  \hat{Q} \alpha = Q |\alpha\rangle
  \] \vspace{3px}
  
  Determinate states of an operator $\hat{Q}$ are eigenfunctions of $\hat{Q}$.
  This means the eigenvalue $Q$  can be used as a wave function label! -- we'll
  get to that later. 
\end{mainbox}

The eigenvalue $Q$ of a Hermitian operator is a real number -- I will
distinguish operators from their eigenvalues by denoting the former with
a hat, $\hat{Q}$. Thus the stationary states $|\alpha_i\rangle$ of the
Hamiltonians we have considered 

\[
\hat{H}|\alpha_i\rangle = E_i |\alpha_i\rangle
\] \vspace{3px}

are the determinate states of $\hat{H}$. The collection of all eigenvalues of
an operator $ \hat{Q}$ is the \textit{spectrum} of $\hat{Q}$. States with the
same eigenvalue are said to be \textit{degenerate}.


\subsection{Discrete Spectra}

We consider a Hermitian operator with a discrete spectrum -- the eigenvalues
correspond to eigenfunctions that are normalizable and thus proper states. We
note 

\begin{subbox}{}
  The eigenvalues of the normalizable eigenfunctions of a Hermitian operator
  are \textit{real} and eigenfunctions belonging to distinct eigenvalues are
  \textit{orthogonal}. 
\end{subbox}

The eigenvalues are real because 

\[
Q_\alpha = \langle \alpha | \hat{Q} \alpha \rangle = \langle \hat{Q}\alpha
| \alpha\rangle = \left[ | \hat{Q}\alpha \rangle \right]^\dagger |\alpha
\rangle = [Q_\alpha|\alpha\rangle]^\dagger |\alpha \rangle = Q_\alpha^* \langle
\alpha | \alpha \rangle = Q_\alpha^*
\] \vspace{3px}
Eigenfunctions belonging to distinct eigenvalues are orthogonal because, if
$\hat{Q}|\alpha\rangle = Q_\alpha |\alpha \rangle$ and $\hat{Q}|\beta\rangle
= Q_\beta |\beta\rangle$, with $Q_\alpha \neq Q_\beta$, then 

\[
  Q_\beta \langle \alpha | \beta \rangle = \textcolor{red}{\langle \alpha | \hat{Q} \beta \rangle
  = \langle \hat{Q} \alpha | \beta \rangle} = Q_\alpha \langle \alpha | \beta
\rangle \quad \Rightarrow \quad (Q\beta - Q_\alpha) \langle \alpha | \beta
\rangle  = 0 \quad \Rightarrow \quad \langle \alpha | \beta \rangle  = 0 
\] \vspace{3px}

If a subset of eigenfunctions are degenerate, we have previously described how
these states can be \textit{made} ortogonal/orthonormal via Gram-Schmidt. Thus
we can assume that the eigenfunctions of a Hermitian operator form an
orthonormal basis. 

A third property of the eigenfunctions of a Hermitian operator is 

\begin{mainbox}{}
  The eigenfunctions of an operator for any observable are complete. Any
  state in the Hilbert space can be expanded in terms of this basis.
\end{mainbox}

We have generally used the eigenstates of the Hamiltonian $\hat{H}$ as our
basis, expanding initial wave packets in terms of these stationary states, then
exploiting the prime directive to determine the time evolution of the wave
packet. While this is a special basis because of the prime directive, we
certainly have the freedom to use as a basis the eigenfunctions of other
operators. 


\subsection{Projection Operators} 

We have previously expressed the completeness of our Hilbert space, for
simplicity assumed discrete here, by the identity 

\[
I = \sum_{i=1}^{\infty} |i\rangle \langle i | 
\] \vspace{3px}

The projection operator onto the state $i$ is defined by 

\[
\hat{P}_i = |i\rangle \langle i | \quad \textit{ so equivalently } \quad
I = \sum_{i=1}^{\infty} \hat{P}_i 
\] \vspace{3px}

Projection operators have the following properties, which follow immediately
from the assumed orthonormality of the basis, 

\[
  \hat{P}_i \hat{P}_j = \delta_{ij} \hat{P}_j \quad \text{so in particular}
  \quad \hat{P}_i \hat{P}_i = \hat{P}_i
\] \vspace{3px}


Projection operators in quantum mechanics are connected with measurement, and thus are very
useful. For example, if a measurement is done on a wave packet formed as a sum
over many stationary states, there may be many possible outcomes of that
measurement. However, a specific
measurement will yield one result. The act of measurement is often said to
``collapse" the wave function. To be specific, suppose a measurement is done on
a system described by a wave packet to determine the energy. Assuming the
stationary states are non-degenerate, the result obtained will correspond to
the energy of one of the stationary states $i$. The measurement will thus have
collapsed the wave packet to that state. The state of the system immediately
after measurement will then be given by the projection operator $\hat{P}_i$
acting on the original wave packet (up to normalization, which in this case
would be adjusted to one).

In the case of measurement, the basis you use and thus the projection operator
employed depends on what one measures -- energy, position, momentum, spin,
etc. But each such observable
corresponds to a Hermitian operator for which there is a complete set of eigenfunctions, as noted
above. The appropriate projection operator thus should use the basis
appropriate to the measure ment. If the operator is not the Hamiltonian,
measurement will again collapse the wave function,
but the projection will generally not be to a single stationary state.


\subsection{Generalized Statistical Interpretation of Measurements} 


We can generalize early discussions about operators and measurements to include
any Hermitian operator, which all correspond to possible observables. The
normalizable eigenfunctions of a Hermitian operator $\hat{Q}$ are real and the
normalized eigenfunctions form an orthonormal basis. Consequently for any
Hermitian operator with a discrete spectrum, 

\[
\langle \alpha | \hat{Q} | \alpha \rangle = \sum_{i=1}^{\infty}
\sum_{j=1}^{\infty} \langle \alpha | i\rangle \langle i | \hat{Q} | j\rangle
\langle j | \alpha \rangle = \sum_{i=1}^{\infty} \sum_{j=1}^{\infty} \langle
\alpha | i \rangle Q_j \langle i | j \rangle \langle j | \alpha \rangle
= \sum_{i=1}^{\infty} Q_i | \langle i | \alpha \rangle|^2 \equiv
\sum_{i=1}^{\infty} Q_i |c_i^\alpha |^2  
\] \vspace{3px}

Here it is understood that $| \alpha \rangle$ is the state of a system at some
time of measurement $t$, and that the sum is over all discrete eigenstates of
$\hat{Q}$ which here is taken to be infinite, but could also very well be
finite. 

The Quantum Mechanical interpretation is that if a series of identical
experiments are done, with $|\alpha\rangle$ prepared identically each time,
then the possible outcomes $Q_i$ will be found with probability $|\langle
i | \alpha \rangle | ^2 = |c_i^\alpha|^2$. This is a probability because 

\[
1 = \langle \alpha | \alpha \rangle = \sum_{i=1}^{\infty} \langle \alpha
  | i\rangle \langle i | \alpha \rangle  = \sum_{i=1}^{\infty} |c_i^\alpha |^2 
\] \vspace{3px}

Thus any outcome $Q_i$ is possible in a given measurement -- that is, any
outcome for which the amplitude $\langle i | \alpha \rangle$ is nonzero. Only
the expectation -- the average of the outcomes after many repetitions of an
identical experiment -- is certain in Quantum Mechanics, 

\[
\langle \hat{Q} \rangle = \sum_{i=1}^{\infty}  Q_i |c_i^\alpha | ^2
\] \vspace{3px}

This interpretation generalizes to the continuous case

\begin{align} \label{}
  \langle \alpha | \hat{Q} | \alpha \rangle &= \int_{-\infty}^{\infty} dz
  \int_{-\infty}^{\infty} \langle \alpha | z \rangle \langle
  z | \hat{Q}(\hat{z'})| z' \rangle \langle z' | \alpha \rangle \, dz'
  = \int_{-\infty}^{\infty}  dz \int_{-\infty}^{\infty}  \langle \alpha
  | z \rangle Q(z') \langle z | z' \rangle \langle z' | \alpha \rangle \, dz'
  \\
  &= \int_{-\infty}^{\infty} dz \int_{-\infty}^{\infty} \langle \alpha
  | z \rangle Q(z') \delta(z- z') \langle z' | \alpha \rangle \, dz'
  = \int_{-\infty}^{\infty} Q(z) | \langle z | \alpha \rangle |^2 \, dz
\end{align}\vspace{3px}

where $z$ represents the appropriate continuous eigenspectrum, for example, $x$
if we are finding the expectation values for some potential $\hat{V}(\hat{x})$
or $ p$ if we were evaluating the operator $-\frac{\hbar^2}{2m}\hat{p}^2$. So
we can summarize 



\begin{subbox}{}
  A Hermitian $\hat{Q}$ has a discrete spectrum and eigenstates \[ \{Q_i,
  |i\rangle\} \Rightarrow \langle \alpha | \hat{Q} | \alpha \rangle 
  = \sum_{i=1}^{\infty}  Q_i |c_i^\alpha|^2\] \[ \text{Probability of outcome
    } \quad Q_i
  = |c_i^\alpha|^2 = |\langle i | \alpha \rangle |^2. \] 

  A Hermitian $\hat{Q}$ has a continuous spectrum and eigenstates \[\{Q(z),
  |z\rangle\} \Rightarrow \langle \alpha | \hat{Q} | \alpha \rangle
= \int_{-\infty}^{\infty} Q(z)|\langle z | \alpha \rangle |^2 \, dz\] 
  \[ \text{Probability of outcome } \quad  dQ(z) = |\langle z | \alpha \rangle
  |^2 \, dz \]
\end{subbox}

Here we define $dQ(z)$ via a Taylor series expansion. If $z_0$ is the middle of
the region $\delta z$, \[ \text{outcome between } Q(z_0 + \delta z / 2) \text{
  and }
  Q(z_0 - \delta z / 2) \text{ has the probability } |\langle z_0 | \alpha \rangle |^2
  \delta z\] \[ \text{ outcome between } Q(z_0) \pm \frac{1}{2}
  \frac{dQ(z)}{dz}\Bigg|_{z_0} \, dz \text{ has the probability } |\langle z_0 | \alpha
\rangle |^2 \, dz\] 



\subsection{Simultaneous Measurements \& Commuting Operators} 




Now we come to a key point that will influence much of the rest of this book,
as we move onto problems in 3D with more degrees of freedom. 

Suppose we have two Hermitian operators $\hat{A}$ and $\hat{B}$ with a common,
complete set of eigenvectors. That is, 

\[
  \hat{A} \psi_{A_i, B_i} = A_i \psi_{A_i \, B_i} \qquad \hat{B}\psi_{A_i \,
  B_i} = B_i \psi_{A_i \, B_i}
\] \vspace{3px}

Let's first pause here to explain the notation. I've assumed a basis exists
that can be enumerated by $i$, with each basis state carrying to quantum
numbers $A_i$ and $B_i$ that are the eigenvalues obtained when the operators
$\hat{A} $ and $\hat{B}$ act on the basis state. One can view $(A_i, B_i)$ as
the index of the various basis states -- one has to specify two basis state
eigenvalues to identify the basis state. Our basis states are simultaneously
eigenvectors of both  $\hat{A}$ and $\hat{B}$. 


It follows that 

\begin{align} \label{}
  \hat{A}[\hat{B}\psi_{A_i \, B_i}] &= \hat{A}[B_i \psi_{A_i \, B_i}] = A_i B_i
  \psi_{A_i\, B_i} \\ \hat{B}[\hat{A} \psi_{A_i \, B_i} ] &= \hat{B} [ A_i
  \psi_{A_i \, B_i } ] = A_i B_i \psi_{A_i \, B_i }
\end{align}\vspace{3px}

Thus $(\hat{A}\hat{B} - \hat{B}\hat{A}) \psi_{A_i \, B_i} = 0$ for every state
in the Hilbert space. We conclude 

\begin{mainbox}{}
  If two observables are simultaneously measurable, 
  \[
    [\hat{A}, \hat{B}] = 0
  \] \vspace{3px}
  Conversely one can show if $[\hat{A}, \hat{B}] = 0$, a common set of
  eigenfunctions for the operators can be found. 
\end{mainbox}

\subsection{Generalized Uncertainty Principle} 

Most Hermitian do not commute ($[\hat{A} , \hat{B}] = 0 \rightarrow
\hat{A}\hat{B} = \hat{B}\hat{A}$). If $\hat{A} = \hat{p}$ and $\hat{B}
= \hat{x}$, then working in position space, 

\[
  [\hat{A}, \hat{B}]\psi(x) = \left[ \frac{\hbar}{i} \frac{d }{d x}
  x - x \frac{\hbar}{i} \frac{d }{d x} \right] \psi(x) = \frac{\hbar}{i}
  \psi(x) \quad [\hat{A}, \hat{B}] = \frac{\hbar}{i}
\] \vspace{3px}

So one can not label states simultaneously by their position and momentum. 

If $\hat{A}$ and $\hat{B}$ are Hermitian, then their product can be written as
the sum of two Hermitian operators, 

\[
 \hat{A}\hat{B} = \frac{1}{2} \left[ (\hat{A}\hat{B} +  \hat{B}\hat{A})
+ i \left( \frac{1}{i} [\hat{A}, \hat{B}]\right) \right] \equiv \frac{1}{2}
\left[ \hat{H}_1 + i \hat{H}_2\right]
\] \vspace{3px}

as 
\[
  \hat{H}_1^\dagger \equiv \left( \hat{A}\hat{B}
  + \hat{B}\hat{A}\right)^\dagger = \hat{B}\hat{A} + \hat{A}\hat{B}
= \hat{H}_1 
\]
\[ \hat{H}_2^\dagger \equiv \left( \frac{1}{i} [\hat{A},
  \hat{B}]\right)^\dagger = -\frac{1}{i} (\hat{A}\hat{B}
  - \hat{B}\hat{A})^\dagger = -\frac{1}{i}(\hat{B}\hat{A} - \hat{A}\hat{B})
  = \hat{H}_2
\]


We have previously defined uncertainty for a (Hermitian) operator $\hat{A}$ in
terms of the variance 

\[
 \langle (\Delta \hat{A})^2 \rangle \equiv \langle \psi | (\hat{A} - \langle
 \hat{A} \rangle )^2 | \psi\rangle = \langle \Delta \hat{A} \psi\rangle \equiv
 \langle f | f \rangle 
\] \vspace{3px}

so also 

\[
\langle ( \Delta \hat{B} )^2 \rangle \equiv \langle \Delta \hat{B}\psi | \Delta
\hat{B} \psi \rangle \equiv \langle g | g \rangle
\] \vspace{3px}

Previously we have used various forms of the Schwarz inequality, which comes
from 

\begin{align} \label{}
  &\int (\langle g |g\rangle f(x) - \langle g | f \rangle g(x))^* (\langle
  g | g \rangle f(x) - \langle g | f \rangle g(x) ) \, dx \geq 0 \\
  &\Rightarrow [\langle g | g \rangle^2 \langle f | f \rangle - \langle
  g | g \rangle \langle f | g \rangle \langle g | f \rangle - \langle
  g | g \rangle \langle g | f \rangle^* \langle g | f \rangle + \langle
  g | f\rangle^* \langle g | f\rangle \langle g | g \rangle ] \geq 0 \\
  &\Rightarrow \langle g | g \rangle \langle f | f \rangle \geq | \langle
  f | g \rangle |^2
\end{align}\vspace{3px}

So applying this to the $f$ and $g$ defined above, 

\begin{align} \label{}
  \langle \Delta \hat{A} \psi | \Delta \hat{A} \psi \rangle \langle \Delta
  \hat{B} \psi | \Delta \hat{B} \psi \rangle &\geq \Bigg| \langle \Delta
  \hat{A} \psi | \Delta \hat{B} \psi \rangle \Bigg|^2 \\
  \Rightarrow \langle \psi | (\Delta \hat{A})^2 |\psi\rangle \langle \psi
  | (\Delta \hat{B})^2 | \psi \rangle &\geq \textcolor{red}{\Bigg| \langle \psi
  | \Delta \hat{A} \Delta \hat{B} |\psi\rangle \Bigg|^2}
\end{align}\vspace{3px}

Now we utilize our theorem above, 

\[
\Delta \hat{A} \Delta \hat{B} = \frac{1}{2} \left[ (\Delta \hat{A} \Delta
  \hat{B} + \Delta \hat{B} \Delta \hat{A}) + i\left( \frac{1}{i} [\Delta
\hat{A}, \Delta \hat{B}] \right) \right] \equiv \frac{1}{2}
\left[\hat{H}_1 + i \hat{H}_2 \right]
\] \vspace{3px}


So applying this to the RHS above and remembering expectation values of
Hermitian operators are real numbers (observables), 

\[
  \textcolor{red}{\Bigg| \langle \psi | \Delta \hat{A} \Delta \hat{B} | \psi
  \rangle \Bigg|^2 } = \Bigg| \langle \psi | \frac{1}{2} \left[ \hat{H}_1
  + i \hat{H}_2 \right] |\psi\rangle \Bigg|^2 = \langle \psi | \frac{1}{2}
  \hat{H}_1 | \psi \rangle^2 + \langle \psi | \frac{1}{2} \hat{H}_2
  |\psi\rangle ^2 \geq \langle \psi | \frac{1}{2} \hat{H}_2 |\psi\rangle^2
\] \vspace{3px}

Thus we obtain the generalized uncertainty principle: 

\begin{subbox}{Generalized Uncertainty Principle}
  \[ \textcolor{red}{\langle (\Delta \hat{A})^2 \rangle \langle (\Delta
    \hat{B})^2 \rangle}\; \geq\;
    \langle \frac{1}{2i} \left[ \Delta \hat{A}, \Delta \hat{B}\right] \rangle^2
    = \textcolor{red}{\langle \frac{1}{2i} \left[\hat{A}, \hat{B}\right]
  \rangle^2} \]
    \[ \text{or } \quad \sigma_A\sigma_B \; \geq\; \Bigg| \langle \frac{1}{2i} \left[ \hat{A},
    \hat{B} \right] \rangle \Bigg| \] 
\end{subbox}


Consequently if $\left[ \hat{A}, \hat{B}\right]$ commute, then it is possible
that $\sigma_A^2 \sigma_B^2 = 0$, in which case we could find a basis in which
the state has \textit{definite} eigenvalues $A$ and $B$. But if $\left[\hat{A},
\hat{B}\right] \neq 0 $, like position and momentum, we can not label wave
functions by the eigenvalues of both operators. These observables are
incompatible in this sense, and we can not construct a complete set of common
eigenfunctions. It follows 

\begin{mainbox}{}
  There is an uncertainty principle for every pair of observables whose
  operators do not commute.
\end{mainbox}


It is enormously helpful to identify \textit{maximal sets of commuting
operators} such as $\hat{H}, \hat{P}, \hat{\vec{L}}^2, \hat{L_z}, \hdots$. Once
identified, they provide instructions for identifying complete orthonormal
bases that simply our mechanics. Such bases break up the Hilbert space
into blocks that can often be treated separately. 

But we also learn that uncertainties are inherent in Quantum Mechanics when we
ask multiple questions of systems. It is not just $\hat{x}$ and $\hat{p}$ that
can not be determined simultaneously with unlimited precision, but most
observables. 



\subsection{Energy-Time Uncertainty Principle} 

We consider the time variation of the expectation of observable $\hat{Q}$ 

\[
\frac{d }{d t} \langle \psi |\psi\rangle = \langle \frac{\partial
\psi}{\partial t} | \hat{Q} | \psi \rangle + \langle \psi | \frac{d \hat{Q}}{d
t} | \psi \rangle + \langle \psi | \hat{Q} | \frac{\partial \psi}{\partial t} 
\] \vspace{3px}

where we allow for $\hat{Q}$ itself having explicit time dependence. Now if
$|\psi\rangle$ is a solution of the Schr\"odinger equation, 

\[
i\hbar \frac{\partial }{\partial t} |\psi \rangle = \hat{H}|\psi\rangle \quad
\Rightarrow \frac{\partial }{\partial t} |\psi\rangle = -\frac{i}{\hbar}
\hat{H} |\psi \rangle
\] \vspace{3px}

This yields

\begin{subbox}{Generalized Ehrenfest Theorem}
  \[
  \frac{d }{d t} \langle \psi | \hat{Q} | \psi \rangle = \frac{i}{\hbar}
  \langle \psi | [\hat{H}, \hat{Q}]| \psi \rangle + \langle \psi
  | \frac{\partial \hat{Q}}{\partial t} | \psi \rangle
  \] \vspace{3px}
\end{subbox}


Thus for operators that have no explicit dependence on time, the rate of change
of the expectation value of an operator is determined by the commutator of
$\hat{Q}$ with $\hat{H}$. If $\hat{Q}$ commutes with $\hat{H}$, its expectation
value is constant in time. 

We can obtain a related result from the generalized uncertainty principle by
the substitutions $\hat{A} \rightarrow \hat{H}$ and $\hat{B} \rightarrow
\hat{Q}$, yielding in a case for an operator that has no explicit time
dependence 

\[
\sigma^2_H\sigma_Q^2 \; \geq \; \langle \psi | \frac{1}{2i} \left[ \hat{H},
\hat{Q}\right]| \psi \rangle^2 = \left[ -\frac{\hbar}{2} \frac{d }{d t} \langle
\psi | \hat{Q} | \psi \rangle \right]^2 \quad \Rightarrow \quad \sigma_H
\sigma_Q \; \geq \; \frac{\hbar}{2} \Bigg| \frac{d }{d t} \langle \psi
| \hat{Q} | \psi \rangle \Bigg|
\] \vspace{3px}


If we abbreviate the expectation value of the Hamiltonian, $\langle \psi
| \hat{H} | \psi \rangle$ as $\bar{E}$, 

\[
  \sigma_H^2 = \langle \psi | (\hat{H} - \bar{E})^2 | \psi \rangle
  = \sum_{i,j}^{} \langle \psi | E_j \rangle \langle E_j | (\hat{H}
  - \bar{E})^2 | E_i \rangle \langle E_i | \psi \rangle = \sum_{i}^{} |c_i|^2
  (E_i - \bar{E})^2 \equiv (\Delta E)^2
\] \vspace{3px}

Given a time $t_0$ of a measurement, we can also define a time interval $\Delta
t$ where the expectation value of $\hat{Q}$ will change by one standard
deviation, 

\[
\sigma_Q = \Bigg| \frac{d \langle \psi | \hat{Q} | \psi \rangle}{d t_0}\Bigg| \,
\Delta t_Q
\] \vspace{3px}

Plugging these into our result above, we obtain 

\begin{mainbox}{Energy-Time Uncertainty Principle}
  \[
  \Delta E \Delta t_Q \; \geq \; \frac{\hbar}{2}
  \] \vspace{3px}
  
  where \[ \Delta E = \sqrt{\langle | \hat{H}^2 | \psi \rangle - \langle \psi
  | \hat{H} | \psi \rangle^2} \qquad \sigma_Q = \Bigg| \frac{d \langle \psi
| \hat{Q}| \psi \rangle}{d t_0} \Bigg|\, \Delta t_Q \]
\end{mainbox}




Thus if a wave packet is a stationary state, $\Delta E \rightarrow 0$,
requiring $\Big| \frac{d \langle \psi | \hat{Q} | \psi \rangle}{d t_0} \Big|
\, \Delta t_Q \rightarrow 0$. Thus any observable that is itself not explicitly
dependent on time will not vary. 





















