
\section{Addition of Angular Momentum}

Much of the interesting physics of the hydrogen atom and other atomic systems
derives from interactions that involve both orbital motion and the electron
spin (as well as the nuclear total angular momentum). This problem leads us to
describe systems with more than one angular momentum operator. We describe here
the coupling of two commuting angular momentum operators $\hat{J}_1$ and
$\hat{J}_2$. An understanding of this two-angular-momenta system will allow us
to proceed to other problems of much greater complexity -- by successively
coupling angular momenta in pairs. 

The orthonormal eigenstates of $\hat{J_1}^2$ and $J_{1z}$ we will denote by
$|j_1m_1\rangle$: $\hat{J_2}$ will have no effect on these states. Similarly
$\hat{J_2}$ and $J_{2z}$ will have the eigenstates $|j_2m_2\rangle$ and
$\hat{J_1}$ will have no effect on them. That is, these operators act in
a direct product space 

\[
  \Sigma = \Sigma_{j_1} \otimes \Sigma_{j_2}
\] \vspace{3px}

corresponding to the state vectors 

\[
|j_1m_1; j_2m_2 \rangle \equiv |j_1m_1\rangle |j_2m_2\rangle
\] \vspace{3px}

The wave function labels come from the full set of four commuting operators
$\hat{J_1}^2, J_{1z}, \hat{J_2}^2,$ and $J_{2z}$. The Hilbert space of
a physical problem may involve other degrees of freedom. For example, if one is
describing a particle according to its location in 3D space and its spin, then
its wave function could be represented as 

\[
  | \vec{r_1} s_1 = \frac{1}{2}m_{s_1} \rangle \rightarrow |n_1\ell 1 m_1 s_1
  = \frac{1}{2} m_{s_1} \rangle 
\] \vspace{3px}

Or one could have two electrons, with a possible set of labels for the full
Hilbert space being 

\[
  | \vec{r_1} s_1 = \frac{1}{2} m_{s_1} \rangle | \vec{r_2} s_2
  = \frac{1}{2}m_{s_2} \equiv \begin{cases}
    |\vec{r_1}s_1m_{s_1}; \vec{r_2}s_2m_{s_2} \\ |\vec{r_1}\rangle |
    s_1m_{s_1}\rangle |\vec{r_2}\rangle |s_2m_{s_2}\rangle
  \end{cases} 
\] \vspace{3px}


That is, we can think of this as a single Hilbert space for the problem, or
alternatively as a direct product space involving the kets for particle 1 and
particle 2, or alternatively, the ket for each particle can be viewed as
a \textit{product} of kets describing the particle's spatial and spin degrees
of freedom. 

Another example would be the case where $\hat{J_1}$ and $\hat{J_2}$ might
represent the orbital and spin angular momentum carried by a single particle.
This would correspond to one of the electrons described above, but where the
nature of the problem (e.g., motion in a central field) allows us to further
decompose the state vector $|\vec{r_1}\rangle$. In this case the full set of
labels for our Hilbert space could be taken to be 

\[
  |\vec{r_1}s_1m_{s_1} \rangle \rightarrow | n_1\ell_1 m_{\ell_1} s_1m_1\rangle
\] \vspace{3px}


Both of the examples described above involve Hilbert spaces that are infinite,
if all degrees of freedom are considered.  In many cases we may want to focus only on the angular momentum
quantum numbers – and often only the magnetic quantum numbers, as we may be working in
subspaces with fixed $j_1$ and $j_2$. In that case we may put aside the other quantum numbers that
may be carried along (implicitly) and focus on just the angular momentum degrees of freedom. In
the following we thus will suppress the accompanying quantum numbers. 

If we have commuting angular momenta $\hat{J_1}$ and $\hat{J_2}$, then when we
form $\hat{J} = \hat{J_1} + \hat{J_2}$, 

\[
  [J_{1i}, J_{1j}] = i\epsilon_{ijk} J_{1k} \quad [J_{2i}, J_{2j}]
  = i\epsilon_{ijk} J_{2k} \quad \Rightarrow \quad [J_i, J_j] = [J_{1i},
  J_{1j}] + [J_{2i}, J_{2j}] = i\epsilon_{ijk} (J_{1k} + J_{2k})
  = \epsilon_{ijk} J_k
\] \vspace{3px}

we have another angular momentum. The ``coupled representation" corresponds to
a direct sum of subspaces having definite $j$

\[
  |(j_1j_2) jm \rangle \quad \Leftrightarrow \quad \Sigma = \Sigma_{|j_1
  - j_2|} \oplus \cdots \oplus \Sigma_{j_1+j_2} \qquad |j_1-j_2| \leq j \leq
  j_1 + j_2
\] \vspace{3px}


The wave function has been labeled by the eigenvalues of another set of four
commuting operators $\hat{J_1}^2,  \hat{J_2}^2, \hat{J^2}$, and $J_z$. Thus the
eigenvalues $j_1$ and $j_2$ are held in common in the coupled and uncoupled
representations. Previously we counted the number of distinct magnetic
substates in the uncoupled representation -- 

\[
  (2j_1 + 1 )(2j_2+1)
\] \vspace{3px}

and of course we have the same number of states in the coupled representation, 

\[
\sum_{j=|j_1 - j_2|}^{j_1+j_2} (2j+1) = (j_1 + 1)(j_2+1) 
\] \vspace{3px}


Which basis should we use? It depends on one's problem. For the simple Coulomb
hydrogen atom problem, the interaction had no dependence on spin, so we worked
in the uncoupled representation -- and could forget entirely about the spin
degree of freedom. If we include spin, all we have to remember is that each
state is actually two, one with spin up, and one with spin down. But they have
the same energy. 

However, had we considered corrections to the hydrogen atom Hamiltonian
associated with the electron's velocity, we would have encountered new
contributions to the Hamiltonian, such as an interaction proportional to
$\vec{\ell} \cdot \vec{s} $ that couples the electron's angular momentum to its
spin. The eigenstates can no longer be written as single uncoupled states, but
instead take on the coupled form, 

\[
|(\ell s) jm_j\rangle
\] \vspace{3px}


where $j$ is the total angular momentum we get by coupling $\ell$ to $s
= \frac{1}{2}$, so $j = \ell \pm \frac{1}{2}$. Thus the six uncoupled $2p$
states in hydrogen: 

\[
2p: \quad |n=2, \ell=1, m \rangle |s = \frac{1}{2}m_s\rangle \rightarrow
\begin{cases}
  |n \left( \ell\frac{1}{2} \right) j = \frac{3}{2} m\rangle \\ 
  |n\left( \ell\frac{1}{2} \right) j = \frac{1}{2}m\rangle 
\end{cases} 
\] \vspace{3px}




are reshuffled to produce two subsets of states that transform as $j
= \frac{1}{2}$ and $\frac{3}{2}$ amplitudes. These states -- not the uncoupled
ones -- are the stationary state basis Nature chooses and thus we must also.
The problem is rotationally invariant, and Nature knows that. In this sense,
our ability to solve the hydrogen atom in the uncoupled basis was the result of
an ``accidental" degeneracy. ``Accident" is in quotes because the accident was
on we created -- by ignoring the fine-structure interactions that break the
degeneracy of these states.


