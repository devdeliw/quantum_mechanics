\section{Angular Momentum}
\subsection{Addition of Angular Momentum}

Much of the interesting physics of the hydrogen atom and other atomic systems
derives from interactions that involve both orbital motion and the electron
spin (as well as the nuclear total angular momentum). This problem leads us to
describe systems with more than one angular momentum operator. We describe here
the coupling of two commuting angular momentum operators $\hat{J}_1$ and
$\hat{J}_2$. An understanding of this two-angular-momenta system will allow us
to proceed to other problems of much greater complexity -- by successively
coupling angular momenta in pairs. 

The orthonormal eigenstates of $\hat{J_1}^2$ and $J_{1z}$ we will denote by
$|j_1m_1\rangle$: $\hat{J_2}$ will have no effect on these states. Similarly
$\hat{J_2}$ and $J_{2z}$ will have the eigenstates $|j_2m_2\rangle$ and
$\hat{J_1}$ will have no effect on them. That is, these operators act in
a direct product space 

\[
  \Sigma = \Sigma_{j_1} \otimes \Sigma_{j_2}
\] \vspace{3px}

corresponding to the state vectors 

\[
|j_1m_1; j_2m_2 \rangle \equiv |j_1m_1\rangle |j_2m_2\rangle
\] \vspace{3px}

The wave function labels come from the full set of four commuting operators
$\hat{J_1}^2, J_{1z}, \hat{J_2}^2,$ and $J_{2z}$. The Hilbert space of
a physical problem may involve other degrees of freedom. For example, if one is
describing a particle according to its location in 3D space and its spin, then
its wave function could be represented as 

\[
  | \vec{r_1} s_1 = \frac{1}{2}m_{s_1} \rangle \rightarrow |n_1\ell 1 m_1 s_1
  = \frac{1}{2} m_{s_1} \rangle 
\] \vspace{3px}

Or one could have two electrons, with a possible set of labels for the full
Hilbert space being 

\[
  | \vec{r_1} s_1 = \frac{1}{2} m_{s_1} \rangle | \vec{r_2} s_2
  = \frac{1}{2}m_{s_2} \equiv \begin{cases}
    |\vec{r_1}s_1m_{s_1}; \vec{r_2}s_2m_{s_2} \\ |\vec{r_1}\rangle |
    s_1m_{s_1}\rangle |\vec{r_2}\rangle |s_2m_{s_2}\rangle
  \end{cases} 
\] \vspace{3px}


That is, we can think of this as a single Hilbert space for the problem, or
alternatively as a direct product space involving the kets for particle 1 and
particle 2, or alternatively, the ket for each particle can be viewed as
a \textit{product} of kets describing the particle's spatial and spin degrees
of freedom. 

Another example would be the case where $\hat{J_1}$ and $\hat{J_2}$ might
represent the orbital and spin angular momentum carried by a single particle.
This would correspond to one of the electrons described above, but where the
nature of the problem (e.g., motion in a central field) allows us to further
decompose the state vector $|\vec{r_1}\rangle$. In this case the full set of
labels for our Hilbert space could be taken to be 

\[
  |\vec{r_1}s_1m_{s_1} \rangle \rightarrow | n_1\ell_1 m_{\ell_1} s_1m_1\rangle
\] \vspace{3px}


Both of the examples described above involve Hilbert spaces that are infinite,
if all degrees of freedom are considered.  In many cases we may want to focus only on the angular momentum
quantum numbers – and often only the magnetic quantum numbers, as we may be working in
subspaces with fixed $j_1$ and $j_2$. In that case we may put aside the other quantum numbers that
may be carried along (implicitly) and focus on just the angular momentum degrees of freedom. In
the following we thus will suppress the accompanying quantum numbers. 

If we have commuting angular momenta $\hat{J_1}$ and $\hat{J_2}$, then when we
form $\hat{J} = \hat{J_1} + \hat{J_2}$, 

\[
  [J_{1i}, J_{1j}] = i\epsilon_{ijk} J_{1k} \quad [J_{2i}, J_{2j}]
  = i\epsilon_{ijk} J_{2k} \quad \Rightarrow \quad [J_i, J_j] = [J_{1i},
  J_{1j}] + [J_{2i}, J_{2j}] = i\epsilon_{ijk} (J_{1k} + J_{2k})
  = \epsilon_{ijk} J_k
\] \vspace{3px}

we have another angular momentum. The ``coupled representation" corresponds to
a direct sum of subspaces having definite $j$

\[
  |(j_1j_2) jm \rangle \quad \Leftrightarrow \quad \Sigma = \Sigma_{|j_1
  - j_2|} \oplus \cdots \oplus \Sigma_{j_1+j_2} \qquad |j_1-j_2| \leq j \leq
  j_1 + j_2
\] \vspace{3px}


The wave function has been labeled by the eigenvalues of another set of four
commuting operators $\hat{J_1}^2,  \hat{J_2}^2, \hat{J^2}$, and $J_z$. Thus the
eigenvalues $j_1$ and $j_2$ are held in common in the coupled and uncoupled
representations. Previously we counted the number of distinct magnetic
substates in the uncoupled representation -- 

\[
  (2j_1 + 1 )(2j_2+1)
\] \vspace{3px}

and of course we have the same number of states in the coupled representation, 

\[
\sum_{j=|j_1 - j_2|}^{j_1+j_2} (2j+1) = (j_1 + 1)(j_2+1) 
\] \vspace{3px}


Which basis should we use? It depends on one's problem. For the simple Coulomb
hydrogen atom problem, the interaction had no dependence on spin, so we worked
in the uncoupled representation -- and could forget entirely about the spin
degree of freedom. If we include spin, all we have to remember is that each
state is actually two, one with spin up, and one with spin down. But they have
the same energy. 

However, had we considered corrections to the hydrogen atom Hamiltonian
associated with the electron's velocity, we would have encountered new
contributions to the Hamiltonian, such as an interaction proportional to
$\vec{\ell} \cdot \vec{s} $ that couples the electron's angular momentum to its
spin. The eigenstates can no longer be written as single uncoupled states, but
instead take on the coupled form, 

\[
|(\ell s) jm_j\rangle
\] \vspace{3px}


where $j$ is the total angular momentum we get by coupling $\ell$ to $s
= \frac{1}{2}$, so $j = \ell \pm \frac{1}{2}$. Thus the six uncoupled $2p$
states in hydrogen: 

\[
2p: \quad |n=2, \ell=1, m \rangle |s = \frac{1}{2}m_s\rangle \rightarrow
\begin{cases}
  |n \left( \ell\frac{1}{2} \right) j = \frac{3}{2} m\rangle \\ 
  |n\left( \ell\frac{1}{2} \right) j = \frac{1}{2}m\rangle 
\end{cases} 
\] \vspace{3px}




are reshuffled to produce two subsets of states that transform as $j
= \frac{1}{2}$ and $\frac{3}{2}$ amplitudes. These states -- not the uncoupled
ones -- are the stationary state basis Nature chooses and thus we must also.
The problem is rotationally invariant, and Nature knows that. In this sense,
our ability to solve the hydrogen atom in the uncoupled basis was the result of
an ``accidental" degeneracy. ``Accident" is in quotes because the accident was
on we created -- by ignoring the fine-structure interactions that break the
degeneracy of these states.

\subsection{Transformations Between Coupled and Uncoupled Bases}

Because the coupled and uncoupled bases of states span equivalent spaces, we
can expand any state in one basis relative to the other. 

\begin{mainbox}{Clebsch-Gordan Coefficients}
  Unitary transformations from the uncoupled to the coupled representations are
  accomplished with \textcolor{red}{Clebsch-Gordan coefficients} 

  \[ |(\ell_1\ell_2) jm_j \rangle = \sum_{m_1 = -\ell_1}^{\ell_1} \sum_{m_2
    = -\ell_2}^{\ell_2} |\ell_1m_1\ell_2m_2\rangle \textcolor{red}{\langle
  \ell_1m_1\ell_2m_2|(\ell_1\ell_2) jm_j\rangle} \]
\end{mainbox}

The transformation coefficients in red -- numbers that conventionally are
real, as we will see -- are the Clebsch-Gordan coefficients. They are nonzero
only if $m_j = m_1 + m_2$. We stress that the result in the box above is simply
a unitary transformation that does not alter the portion of the Hilbert space
being spanned. But the coupled representation breaks the Hilbert space into
a block-diagonal form, with each block labeled by one of the allowed values of
$j, |\ell_1 - \ell_2| \leq \ell_1+\ell_2$.

The inverse transformation is 

\[
|\ell_1 m_1 \ell_2 m_2\rangle = \sum_{j=|\ell_1 - \ell_2|}^{\ell_1+\ell_2}
\sum_{m_j = -j}^{j} |(\ell_1\ell_2) jm_j \rangle \langle
(\ell_1\ell_2)jm_j|\ell_1m_1\ell_2m_2\rangle 
\] \vspace{3px}

The statements of unitary follow from the orthogonality of the states in either
basis, namely 

\[ \delta_{m_1'm_1} \delta_{m_2'm_2} = \langle \ell_1 m_1' \ell_2m_2'
| \ell_1m_1\ell_2m_2 \rangle = \sum_{j=|\ell_1-\ell_2}^{\ell_1+\ell_2}
\sum_{m_j = -j}^{j} \langle \ell_1m_1' \ell_2m_2' | (\ell_1\ell_2)jm_j \rangle
\langle (\ell_1\ell_2)jm_j | \ell_1 m_1 \ell_2 m_2 \rangle\] 

\[
  \delta_{jj'} \delta_{m_j'm_j} = \langle (\ell_1\ell_2)j'm_j' |(\ell_1\ell_2)
  jm_j \rangle = \sum_{m_1 = -\ell_1}^{\ell_1} \sum_{m_2 = -\ell_2}^{\ell_2}
  \langle (\ell_1\ell_2) j'm_j' | \ell_1m_1\ell_2m_2\rangle \langle
  \ell_1m_1\ell_2m_2 | (\ell_1\ell_2)jm_j \rangle
\] \vspace{3px}

It is understood in the last relation that $j$ satisfies the triangle
condition, $|\ell_1-\ell_2| \leq j \leq \ell_1+\ell_2$. 

\subsection{Angular Momentum \& Rotations of States}

Let’s now try to make the connections to rotations
more explicit.

We used rotational symmetry to simplify our treatment of the general central force problem, but in
a hidden way, a consequence of our separation of variable in spherical coordinates. We found that
all 3D central-force problems can be reduced to 1D radial equations, with the angular behavior
encoded in wave functions $|\ell m\rangle$ whose position representations are the
$Y_{\ell m}$s.

\[
  \langle \theta, \phi | \ell m\rangle = Y_{\ell m}(\theta, \phi)
\] 

\begin{itemize}
  \item[1.] The angular solutions are universal, valid for any central-force
    problems we decide to do.
  \item[2.] Infinite-dimensional physics is factored into finite-dimensional
    subspaces, blocks labeled by $\ell$ containing $2\ell+1$ magnetic substates
    labeled by $m$. This revealed an energy degeneracy associated with $m$. 
  \item[3.] $\ell$ and $m$ are quantum labels of our orthonormal stationary
    states.
\end{itemize}


The $Y_{\ell m}$s are angular momentum eigenstates for our coordinate system --
the magnetic number $m$ is defined with respect to the $z$ axis we picked.

Because these eigenfunctions are associated with a particular $z$-axis -- yet
clearly the physics can not depend on the choice of coordinate system -- it is
natural to ask what happens were we to rotate to a different coordinate system.
Under an arbitrary rotation $\hat{U}$, the state $|\ell m\rangle$ will
\textit{remain in the subspace} defined by $\{|\ell m\rangle, m = -\ell,
\cdots, \ell\}$. That is, 

\[
\hat{U}(\hat{n}, \phi)|\ell m\rangle = \sum_{m' = -\ell}^{\ell} |\ell m'\rangle
\langle \ell m' | \hat{U} | \ell m \rangle \equiv \sum_{m' = -\ell}^{\ell}
|\ell m'\rangle D_{m'm}^\ell (\hat{n}, \phi )
\] \vspace{3px}

The rotation operator here has been defined by an axis $\hat{n}$ about which we
rotate and an angle $\phi$ specifying the number of radians in the rotation
(with the direction of rotation conventionally defined by right hand rule). The
unitary operator that accomplishes this is 

\[
\hat{U}(\hat{n}, \phi) = \text{exp} \left[ -\frac{i}{\hbar} \phi  \hat{n} \cdot
\hat{L} \right] \rightarrow \text{exp} \left[ -\frac{i}{\hbar} \phi \hat{n}
\cdot \hat{J}\right]
\] \vspace{3px}

where here we replaced $\hat{L}$ with $\hat{J}$ because we intend the
discussion to apply to \textit{any} angular momentum operator. What is
important about this? The numerical coefficients describing the rotation, the
$D_{m'm}^j(\hat{n}, \phi)$ depend on the rotation and on $j$, but not on
details of the state being rotated. It could be a Coulomb state, a 3D harmonic
oscillator state, etc. 

Given a Hilbert space and an operator acting in that Hilbert space, if there exists a subset of
the Hilbert space in which the operator, acting on that subspace, generates states that are also
in that Hilbert subspace, we say that this subspace is invariant under that operator. Thus under
rotations, the states of definite $\ell, m = -\ell, \cdots , \ell$ are invariant subspaces.

We can do a very simple example of a rotation. Suppose we have picked a set of axes and made
measurements, but someone else has chosen different $x, y$ axes, but hasn't
changed the $z$ axis. What
would our states look like in the other person’s coordinate system? The solution in our coordinate
system can be transformed to the other coordinate system by a rotation by some
angle $\phi$ around our
mutual $z$ axis, to align the $x, y$ axes. As an exponentiated operator is defined by the corresponding
power series, for this case


\[
\hat{U}(\hat{n}, \phi) = \text{exp} \left[ -\frac{i}{\hbar}\phi \hat{n} \cdot
\hat{J}\right] = \text{exp} \left[ -\frac{i}{\hbar} \phi \hat{J}_z\right] 
\] \vspace{3px}


But if we apply $(\hat{J}_z)^n$ on $|jm\rangle$ it returns $m^n$. So on
re-exponentiating, 

\[
  D_{m'm}^j (\hat{n}, \phi) = \langle jm' | \hat{U}|jm\rangle = \langle jm'
  | \text{exp} [-i\phi m] | jm\rangle = \delta_{m'm} \text{exp}  [-i m\phi ]
\] \vspace{3px}

Because the two experimentalists agree on a common z axis, they also have
a common set of quantum numbers. The transformed solution differs from the
original only by a phase. This expression
is easily computed as a function of $\phi$.

How difficult is it to generalize this? 

\begin{itemize}
  \item[1.] What must we do to describe a rotation from one coordinate system
    to another that share the same origin but otherwise can differ in arbitrary
    ways?
  \item[2.] Once we figure out how to characterize the rotation, can we
    evaluate the expressions for $D_{m'm}^j (\hat{n}, \phi)$, as we did with
    our simple rotation above? 
\end{itemize}

Without derivation, the answers are: 

\begin{itemize}
  \item[1.] Using techniques you may have seen in classical mechanics, one
    coordinate system $(x,y,z)$ can be put into alignment with a second
    coordinate system $(x',y',z')$ via a series of three sequential rotations
    about the axes of the original system -- these are the Euler angle
    rotations. The rotations are 

    \begin{itemize}
      \item[-] A rotation of $\gamma$ about the $z$ axis 
      \item[-] A rotation of $\beta$ around the $y$ axis
      \item[-] A rotation of $\alpha$ around the $z$ axis
    \end{itemize}

    This means that the general $\hat{U}$ takes the form 

    \[
    \hat{U}(\alpha, \beta, \gamma) = \text{exp} \left[ -\frac{i}{\hbar} \alpha
    \hat{J_z}\right] \text{exp}  \left[ -\frac{i}{\hbar}\beta \hat{J_y}\right]
    \text{exp} \left[ -\frac{i}{\hbar} \gamma \hat{J_z}\right], \quad 0 \leq
    \alpha, \gamma \leq 2\pi, 0 \leq \beta \leq \pi
    \] \vspace{3px}
    
    and consequently 

    \[
      D_{m'm}^j (\alpha, \beta, \gamma) = \langle jm' | \hat{U}(\alpha, \beta,
      \gamma) | jm \rangle = \text{exp}  [ -i\alpha m] d_{m'm}^j (\beta)
      \text{exp} [-i\gamma m]
    \] \vspace{3px}
    
    where 

    \[
      d_{m'm}^j(\beta) = \langle jm' | \text{exp} \left[ -\frac{i}{\hbar} \beta
      \hat{J_y} \right] | jm\rangle 
    \] \vspace{3px}
    
  \item[2.] an we evaluate expressions like that directly above? As we will
    discuss soon, $\hat{J_y}$ can be
expressed in terms of raising and lowering operators that have a simple matrix element in our
angular momentum subspaces, which when we expand our exponentiated operator in a power
series, allows the power series to be evaluated term by term and summed. All this has been
done and tabulated for our use. So the answer is yes. For $j = \frac{1}{2}$
, for example, one can show

\[
  d_{m'm}^\frac{1}{2} (\beta) = \begin{pmatrix}
    \cos \beta /2 & -\sin \beta /2 \\ \sin \beta/2 & \cos \beta /2 
  \end{pmatrix} 
\] \vspace{3px}

\end{itemize} 

\subsection{Review So Far}


We described how the states of a system with two angular momenta be represented
either in the coupled or uncoupled representations within the same subspace of
the Hilbert space, but with the former more useful in systems where the total
angular momenta is conserved. 

\begin{subbox}{Clebsch-Gordan Coefficients}
  Unitary transformations from the uncoupled to the coupled representations are
  accomplished with \textcolor{red}{Clebsch-Gordan coefficients} 

  \[ |(\ell_1\ell_2) jm_j \rangle = \sum_{m_1 = -\ell_1}^{\ell_1} \sum_{m_2
    = -\ell_2}^{\ell_2} |\ell_1m_1\ell_2m_2\rangle \textcolor{red}{\langle
  \ell_1m_1\ell_2m_2|(\ell_1\ell_2) jm_j\rangle} \]
\end{subbox}

We also described how the rotational symmetry of the eigenstates we derived in
the central force problem, the position-space states $Y_{\ell m}$, remain in
the subspace $\{ |\ell m\rangle, m = - \ell, \cdots, \ell\}$ under rotations.
This is what it means to be an angular momentum eigenstate. So 

\begin{mainbox}{}
  \[ \hat{U}(\hat{n}, \phi) | \ell m \rangle = \sum_{m' = -\ell}^{\ell} |\ell
    m' \rangle \langle \ell m' | \hat{U} | \ell m \rangle \equiv \sum_{m'
  = -\ell}^{\ell}  |\ell m' \rangle D_{m'm}^\ell (\hat{n}, \phi) \] 
  \[ \hat{U}(\hat{n}, \phi) = \text{exp} \left[ -\frac{i}{\hbar} \phi \hat{n}
  \cdot \hat{L} \right] \] 
\end{mainbox}

We also noted that the most general rotation of axes can be described by three
Euler angles defining sequential rotations about the $\hat{z}, \hat{y}$ and
$\hat{z}$ axes of the initial axes, 

\begin{subbox}{}
   \[
    \hat{U}(\alpha, \beta, \gamma) = \text{exp} \left[ -\frac{i}{\hbar} \alpha
    \hat{J_z}\right] \text{exp}  \left[ -\frac{i}{\hbar}\beta \hat{J_y}\right]
    \text{exp} \left[ -\frac{i}{\hbar} \gamma \hat{J_z}\right], \quad 0 \leq
    \alpha, \gamma \leq 2\pi, 0 \leq \beta \leq \pi
    \] \vspace{3px}
    \[
      D_{m'm}^j (\alpha, \beta, \gamma) = \langle jm' | \hat{U}(\alpha, \beta,
      \gamma) | jm \rangle = \text{exp}  [ -i\alpha m] d_{m'm}^j (\beta)
      \text{exp} [-i\gamma m]
    \] \vspace{3px}
    \[
      d_{m'm}^j(\beta) = \langle jm' | \text{exp} \left[ -\frac{i}{\hbar} \beta
      \hat{J_y} \right] | jm\rangle
    \] 
\end{subbox}


\subsection{Constructing Spherical Tensor Operators}

The discussion so far shows that states of good angular momentum not only
factorize the Hilbert space into subspaces, but that this factorization is
preserved under rotations -- the subspaces are invariant. If we can take one
more step -- figure out how to form operators that behave in the same way -- we
will be able to greatly simplify our calculations of observables (matrix
elements). 

Why have we not previously talked about the angular momentum operator? -- how
to formulate such operators so they transform simply under rotations? The
simple reason is that our 3D work thus far has been on Hamiltonian, which are
scalars and thus do not change under rotations. For our central force problem 

\[
\langle \ell' m | \hat{H} | \ell m \rangle \rightarrow \langle \ell m | \hat{H}
| \ell m \rangle \rightarrow \langle \ell | \hat{H}|\ell\rangle 
\] \vspace{3px}

There is no dependence on $m$ and thus no dependence on the choice of
coordinate system. But we have need in Quantum Mechanics for many types of
operators, and most do carry nonzero angular momentum. It is clear we need to
understand how both state vectors \textit{and} operators transform under
rotations, if we are to treat matrix elements involving both states and
operators -- and observables correspond to matrix elements. This is clear
because 

\[
  \langle j' m | \hat{T}| jm \rangle = \langle j'm' | \hat{U}^{-1} \hat{U}
  \hat{T} \hat{U}^{-1} \hat{U} | jm \rangle
  \] \[ \text{ transformed operator: } \hat{T}' = \hat{U}\hat{T}\hat{U}^{-1} \] \vspace{3px}


What requirement should we place on this transformation? It is the following: 

\begin{mainbox}{Spherical Tensor Operators}
  An irreducible spherical tensor operator of rank  $k$, denoted $\hat{T}_q^k$,
  is a set of $2k+1$ operators, $q = -k, -k+1, \cdots, k-1, k$ that transforms
  according to 

  \begin{align} \hat{U}(\alpha, \beta, \gamma) \hat{T}_q^k \hat{U}^{-1}(\alpha, \beta,
    \gamma) = \sum_{q'}^{} T_{q'}^k D_{q'q}^k (\alpha, \beta,\gamma)
  \end{align}
\end{mainbox}

That is, components of spherical tensor operators transform under rotations of
the coordinate frame just like the components of the states $|jm_j\rangle$ we
previously discussed. 

As in the case of choosing the representation $|(\ell_1\ell_2) jm\rangle$ not
$|\ell_1m_1\ell_2m_2\rangle$, this is largely a matter of properly grouping the
components of our operators. For example, we can rearrange the components of
the rank-one operator $\hat{r}$ -- so the Cartesian components $(x, y, z)$
transform into the components of a spherical tensor: 

\[
\begin{pmatrix}
  r_1 \\ r_0\\ r_{-1} 
\end{pmatrix} = \begin{pmatrix}
  -\frac{1}{\sqrt{2}} (x + iy) \\ z \\ \frac{1}{\sqrt{2}} (x - iy)
\end{pmatrix} = r\sqrt{\frac{4\pi}{3}} \begin{pmatrix}
Y_{11}(\theta, \phi) \\ Y_{10} (\theta, \phi) \\ Y_{1-1} (\theta, \phi)
\end{pmatrix}   
\] \vspace{3px}

But what about more complicated operators -- quantum systems may have dipoles,
quadropoles, octupoles, etc. diagonal moments, and more important, these
operators govern the \textit{transitions} between quantum states. Is there some
general procedure for grouping components to create spherical tensors that
transform simply and link only invariant subspaces? 

Before answering this question, let's look at a few examples to get a better
idea of what we want to do. Suppose, for example, we had two factors of
$\vec{r}$, two rank-one operators we want to combine. From the Cartesian
components we  can construct six independent bilinears, 

\[
  (x^2, xy, xz, y^2, yz, z^2) 
\] \vspace{3px}

And the spherical regrouping? It is 

\[
  (x^2, xy, xz, y^2, yz, z^2) \quad \leftrightarrow \qquad \begin{cases}
    \qquad \qquad \qquad (r^2Y_{00}) &J = 0 \\ (r^2 Y_{22}, r^2 Y_{21}, r^2Y_{20}, r^2Y_{2-1},
    r^2Y_{2-2}) &J=2
  \end{cases} 
\] \vspace{3px}

There are six independent Cartesian bilinears on the left, and two spherical
operators on the right, one with a single component ($J=0$ ), and one with five
($J=2$). Note that when we couple  $\hat{r}$ to itself, only even-parity
spherical tensors are generated -- and only six Cartesian bilinears arise. If
we had coupled $\hat{r}$ to the Pauli matrices $\hat{\sigma}$ (we'll get to
Pauli later), nine Cartesian bilinears would result, and a $J=1$ spherical
tensor operator would also be generated. 

Let's try three. The Cartesian trilinears we can now form are these 10, 

\[
  (x^3, x^2y, x^2z, xy^2, xz^2, y^3, y^2z, yz^2, z^3) \quad \leftrightarrow
  \quad \begin{cases}
     r^3Y_{1m} &J=1, m = -1, 0, 1 \\ r^2Y_{3m} &J = 3, m= 3, 2, 1,
    0, -1, -2, -3
  \end{cases} 
\] \vspace{3px}


Again, the two representations agree on the number of trilinear operators. In
this case, the use of three powers of the same vector leads to only odd
spherical harmonics. 

These examples reflect a general procedure that can be used to generate more
complicated spherical tensor operators from elementary ones like $\hat{r},
\hat{p}$, and $\hat{\sigma}$. One can show 

\begin{subbox}{}
  If $\hat{A}_{\ell_1m_1}$ and $\hat{B}_{\ell_2m_2}$ are spherical tensor
  operators, so too is 

  \[ \hat{C}_{\ell m} = \sum_{m_1m_2}^{} \hat{A}_{\ell_1 m_1}
    \hat{B}_{\ell_2m_2} \langle \ell_1m_1 \ell_2m_2| (\ell_1 \ell_2 ) \ell
  m \rangle\] 
\end{subbox}

Our friend the Clebsch-Gordan coefficients show up again. 

\section{The Wigner-Eckart Theorem}

Finally, we come to the real payoff of using states and operators that
transform under rotations like our friends, the $Y_{\ell m}$s. The evaluation
of matrix elements via Wigner-Eckart describes a powerful factorization of
matrix elements into a piece that is operator dependent but frame independent,
and a piece that is specific to the quantization axis the experimentalist has
chosen, but independent of the operator apart from its angular momentum rank.
This second term, again given in terms of Clebsch-Gordan coefficients, encodes
all of the geometry of rotation. Two forms are given, 

\begin{mainbox}{Wigner-Eckart Theorem}
  \begin{align*} \langle n' j' m' | \hat{T}_{kq} | njm \rangle
    &= \frac{(-1)^{j-m}}{\sqrt{2k+1}} \langle j' m' j - m | (j'j) kq \rangle
  \langle nj' ||\hat{T}_k || nj \rangle \\ &=
  \frac{(-1)^{k-j+j'}}{\sqrt{2j'+1}} \langle kq jm | (kj) j'm' \rangle \langle
nj' ||\hat{T}_k || nj\rangle \end{align*}
\end{mainbox}

Although we will not derive this here, the results should look reasonable to
you, at least from a coupling perspective. Consider the second form: it states
that if one operators on a state of angular momenta $j, m$ with a spherical
operator carrying angular momenta $k, q$, the only way this is going to connect
to a final state characterized by $j'm'$ is if $j,m$ and $k,q$ are coupled to
$j'm'$. 

Why is this a powerful result? There are a total of  $(2j' + 1)(2j+1)(2k+1)$
matrix elements above. (The ones where $q + m - m' \neq 0$ are zero though). To
determine these, only one measurement or one calculation is needed. One
measurement for a specific $m', m$ and $q$ determines the ``reduced matrix
element" $\langle n j' || \hat{T}_k || nj \rangle$, a quantity independent of
magnetic quantum numbers. And once that quantity is known, all other matrix
elements are then determined. This makes sense. Suppose, for example, you make
a measurement with one coordinate system, and want to know what the answer is
in another. As bra, ket, and operators all transform under the rules of
rotation, you can envision rotating your axes to the new axes to determine the
answer. That is just geometry, not a function of operator physics. The
Wigner-Eckart theorem just tells one the geometry is really as simple as
a Clebsch-Gordan coefficient. 

\section{Orbital Angular Momenta \& Ladder Operators}

We previously discussed the commutation relations among the orbital angular
momentum operators

\[
  [\hat{L}_i, \hat{L}_j ] = i\hbar \epsilon_{ijk} \hat{L}_k \quad \text{ and
  }  \quad [\hat{L}^2, \hat{L}_i] = 0
\] \vspace{3px}

and then the convention choice to take $\hat{L}^2$ and $L_z$ as the maximal set
of commuting operators. We are then able to adopt a basis labeled by these
quantum numbers -- our stationary states are eigenfunctions of both operators. 

From the unused operators $\hat{L}_z, \hat{L}_y$ we form the linear
combinations 

\[
\hat{L}_\pm \equiv \hat{L}_x \pm iL_y
\] \vspace{3px}

and then we study their properties. 


\begin{itemize}
  \item[1.] $L_\pm$ \textit{is an eigenstate} of $\hat{L}^2$ -- As $[\hat{L}^2,
    L_i] = 0$, clearly $[\hat{L}^2, L_\pm] = 0$. Further we find 

    \[
      [\hat{L}_z, \hat{L}_\pm ] = [\hat{L}_z, \hat{L}_x ] \pm i [\hat{L}_z,
      \hat{L}_y] = i\hbar \hat{L}_y \pm \hbar \hat{L}_x = \pm \hbar(\hat{L}_x
      \pm i \hat{L}_y = \pm \hbar \hat{L}_\pm
    \] \vspace{3px}
    
    So suppose we act on on an eigenstate of $\hat{L}^2$ and $\hat{L}_z$,
    $|n\ell m\rangle$, by $\hat{L}_\pm$: 

    \[
    \hat{L}^2 \hat{L}_\pm | n\ell m \rangle = \hat{L}_\pm \hat{L}^2 | n \ell
    m\rangle = \hat{L}_\pm \hbar^2 \ell(\ell + 1) | n \ell m \rangle = \hbar^2
    \ell(\ell+1) \hat{L}_\pm | n \ell m \rangle 
    \] \vspace{3px}
    
    which states that $\hat{L}_\pm |n \ell m \rangle$ is an \textit{eigenstate}
    of $\hat{L}^2$. So  $\hat{L}_\pm$ produces a state within the selected
    invariant subspace. 

  \item[2.] $\hat{L}_{\pm}$ \textit{raises/lowers} $m$ by one unit; 

    \begin{align*} \label{}
      \langle n \ell m' | \hat{L}_z \hat{L}_\pm - \hat{L}_\pm \hat{L}_z | n\ell
      m\rangle = \pm \hbar \langle n \ell m' | \hat{L}_\pm | n \ell m \rangle
    \end{align*} \[ \Rightarrow \hbar(m' - m) \langle n \ell m' | \hat{L}_z
    | n \ell m \rangle = \pm \hbar \langle n \ell m | \hat{L}_\pm | n \ell
    m\rangle \quad \Rightarrow \quad  m' - m = \pm 1 \] 

    So $\hat{L}_+$ raises $m$ by one unit while $\hat{L}_-$ lowers $m$ by one
    unit. 

  \item[3.] The raising/lowering amplitudes \textit{are related}. 
    \[ \hat{L}_+ |n\ell m\rangle = x_m | n \ell m + 1 \rangle \quad \hat{L}_-
    |n\ell m \rangle = x_m' |n \ell m - 1 \rangle\] \vspace{3px}

    Therefore, $x_m'$

    \[ x_m' = \langle n \ell m - 1 | \hat{L}_- | n \ell m \rangle = \langle
    n \ell m - 1 | \hat{L}_x - i \hat{L}_y | n \ell m \rangle \] 
    \[ = \langle ( \hat{L}_x + i \hat{L}_y ) n \ell m - 1 | n \ell m \rangle
    = \langle \hat{L}_+ n \ell m - 1 | n \ell m \rangle = x_{m-1}^* \]
    \vspace{3px}
   
    Consequently, 

    \[
      x_{m+1}^{'} = x_m^* \quad \Rightarrow \quad \hat{L}_+ | n \ell m \rangle
    = x_m | n \ell m + 1 \rangle \quad \hat{L}_-|n \ell m \rangle
    = x_{m-1}^*|n \ell m -1 \rangle
    \] \vspace{3px}
    
    Note again that the statevector is defined by quantum numbers  $n, \ell$
    and  $m \pm 1$; it is not literally $n \ell m - 1$ as an algebraic
    expression. 

  \item[4.] From the commutation relations one can show $\hat{L}_+ \hat{L}_-
    - \hat{L}_- \hat{L}_+ = 2\hbar L_z$. And 

    \[
    \langle n \ell m | \hat{L}_+\hat{L}_- - \hat{L}_- \hat{L}_+ | n \ell
    m \rangle = x_{m-1}^* \langle n \ell m | \hat{L}_+ | n \ell m - 1\rangle
    - x_m \langle n \ell m | \hat{L}_- | n \ell m + 1 \rangle = |x_{m-1}|^2
  - |x_m|^2 
    \] \vspace{3px}
    
    And 

    \[
    \langle n \ell m | 2\hbar L_z | n \ell m \rangle = 2\hbar^2 m
    \] \vspace{3px}
    
    Therefore

    \[
      |x_m|^2 - |x_{m-1}|^2 = -2\hbar^2 m \quad \Rightarrow \quad |x_m|^2 = (C
      - m (m + 1))\hbar^2
    \] \vspace{3px}
    
    One can show $C = \ell(\ell+1)$ based on the fact that $\hat{L}_+$
    annihilates $| n \ell \ell \rangle$. Consequently, 

    \begin{subbox}{Unfixed Angular Momentum Ladder Operators}
      \begin{align} \label{}
        \hat{L}_+ | n \ell m \rangle &= e^{i\beta_m} \hbar \sqrt{\ell(\ell+1)
        - m(m+1)}|n \ell m + 1 \rangle \\ \hat{L}_-|n\ell m\rangle
                                     &= e^{-i\beta_{m-1}} \hbar \sqrt{\ell(\ell + 1 ) - m(m-1)} | n \ell
        m - 1\rangle 
      \end{align}
    \end{subbox}

  \item[5.] \textit{Phase choices of Condon and Shortley}: We adopt a phase
    convention for fixing the relative phases $e^{i\beta_{m}}$ and
    $e^{-i\beta_{m-1}}$. The relations above involve  $2\ell$ phases $\{
      \beta_{\ell - 1}, \cdots \beta_{-\ell}$ and we have $2\ell$ relative
      phases at our disposal. We can get rid of all the phases recursively,
      starting with $m = \ell$ and $m = \ell - 1$, where we have 

      \begin{align} \label{}
        \hat{L}_- |n \ell \ell \rangle &= e^{-i\beta_{\ell - 1}}\hbar
        \sqrt{\ell(\ell+1) - \ell(\ell-1)} | n \ell \ell - 1 \rangle \\
        \hat{L}_+|n \ell \ell - 1 \rangle &= e^{i\beta_{\ell - 1}}\hbar
        \sqrt{\ell ( \ell - 1 ) - \ell ( \ell - 1 )} | n \ell \ell \rangle
      \end{align}\vspace{3px}
    
      the relations involving these states would read 

      \[
      \hat{L}_- | n \ell \ell \rangle = \hbar \sqrt{\ell(\ell+1)
      - \ell(\ell-1)} | n \ell \ell - 1 \rangle' \quad \hat{L}_+ | n \ell \ell
      - 1 \rangle' = \hbar \sqrt{\ell(\ell+1) - \ell(\ell - 1)}|n \ell \ell
      \rangle
      \] \vspace{3px}
      
      as desired. The relation between states with $m = \ell - 1$ and $m = \ell
      - 2$ would then read 

      \begin{align} \label{}
        \hat{L}_- |n \ell \ell - 1\rangle' &= e^{-\beta(\beta_{\ell - 2}
        + \beta_{\ell - 1} )} \hbar \sqrt{\ell(\ell + 1) - (\ell - 1)(\ell
          - 2)}|n \ell \ell - 2 \rangle \\ \hat{L}_+|n \ell \ell - 2\rangle
                                           &= e^{i(\beta_{\ell - 2}
                                           + \beta_{\ell - 1})} \hbar \sqrt{\ell(\ell + 1)
        - (\ell - 1)(\ell - 2)}|n \ell \ell - 1 \rangle'
      \end{align}\vspace{3px}
      
    and we could absorb this shifted phase into $|n \ell \ell - 2\rangle$
    without affecting what we did above, 

    \[
      |n \ell \ell- 2\rangle' = e^{-i (\beta_{\ell - 2} + \beta_{\ell - 1})} |n
      \ell \ell - 2\rangle
    \] \vspace{3px}
    
    and so on. We adopt such a phase convention for fixing relative phases,
    leaving one overall phase still arbitrary, yielding the conventional
    fixed angular momentum operators

    \begin{mainbox}{Angular Momentum Operators}
      \begin{align} \label{}
        \hat{L}_+ |n\ell m \rangle &= \hbar \sqrt{\ell(\ell + 1) - m ( m +1)}
        | n\ell m + 1 \rangle \\ \hat{L}_-|n \ell m \rangle &= \hbar
        \sqrt{\ell(\ell+1) - m(m-1)} |n \ell m -1 \rangle
      \end{align}
    \end{mainbox}
    
    This choice is incorporated in definitions of the $Y_{\ell m}$s for
    example. Now with these definitions we see 

    \[
    \hat{L}_+|n \ell \ell \rangle = 0 \qquad \hat{L}_-|n \ell - \ell \rangle
    = 0
    \] \vspace{3px}
    
    like I mentioned above. 
    
  \item[6.] \textit{Matrix Elements of $\hat{L}_x$ and $\hat{L}_y$}. As 

    \[
    \hat{L}_x = \frac{1}{2} \left( \hat{L}_+ + \hat{L}_- \right) \quad \text{
    and }  \quad  \hat{L}_y = \frac{1}{2i} \left( \hat{L}_+ - \hat{L}_- \right)   
    \] \vspace{3px}

    one finds 

    \begin{align} \label{}
      \hat{L}_x|\ell m \rangle &= \frac{\hbar}{2} \sqrt{\ell(\ell+1) - m(m+1)}|\ell m +1
      \rangle + \frac{\hbar}{2} \sqrt{\ell(\ell+1) - m(m - 1)}|\ell m -1
      \rangle \\ \hat{L}_y |\ell m \rangle &= \frac{\hbar}{2i}
      \sqrt{\ell(\ell+1) - m(m+1)}|\ell m +1\rangle - \frac{\hbar}{2i}
      \sqrt{\ell(\ell+1) - m (m - 1)}|\ell m -1 \rangle 
    \end{align}\vspace{3px}
    
    
    So these are tri-diagonal Hermitian matrices but with zeros down the
    diagonal. For example, we have the following analogs of the $\ell
    = \frac{1}{2}$ Pauli matrices, 

    \[
    L_x = \frac{\hbar}{2} \begin{pmatrix}
      0 & \sqrt{2} & 0 \\ \sqrt{2} & 0 & \sqrt{2} \\ 0 & \sqrt{2} & 0 
    \end{pmatrix} \qquad L_x = \begin{pmatrix}
      0 & -i\sqrt{2} & 0 \\ i\sqrt{2} & 0 & -i\sqrt{2} \\ 0 & i\sqrt{2} & 0
    \end{pmatrix} 
    \] \vspace{3px}
    
    As the orbital angular momentum comes from $\hat{r} \times \hat{p}$, we can
    transform into spherical coordinates and derive the position-space
    representations of the various components

    \[ \hat{L}_z = \frac{\hbar}{i} \frac{\partial }{\partial \phi} \qquad
      \hat{L}_x = i\hbar \left( \sin \phi \frac{\partial }{\partial \theta}
      + \cos \phi \cot \theta \frac{\partial }{\partial \phi}  \right) \qquad
      \hat{L}_y = i\hbar \left( -\cos\phi \frac{\partial }{\partial \theta}
    + \sin\phi \cot \theta \frac{\partial }{\partial \phi}  \right) \] 
    \[
      \hat{L}_\pm + \pm \hbar e^{\pm i\phi} \left( \frac{\partial }{\partial
      \theta} \pm i \cot \theta \frac{\partial }{\partial \phi}  \right)  
    \] \vspace{3px}
    
    These acting on the $Y_{\ell m}$s for $\ell = 1$ will generate the same
    matrices deduced above. 
\end{itemize}

\section{Generating the Clebsch-Gordan Coefficients}

While I won't go through the derivation of Clebsch-Gordan coefficients in
detail, I do want to stress that the raising/lowering operator properties
derived above, including the Condon and Shortley phase convention, is all one
needs to derive numerical values for the Clebsch-Gordan coefficients while
showing that they are real. The procedure is basically an algorithm that can be
executed using the lowering operator. (For practical purposes just use Wolfram
Mathematica to calculate them for you). 

I will sketch the algorithm for the coupling of two angular momenta $\ell_1
= 1$ and $\ell_2 = 1$. One begin with the state of maximum total $L_z = 2$
which demands that $\ell_{12}$ also be maximum. 

\begin{align}\label{cg_1}
|(\ell = 1 \ell_2 = 1) \ell_{12} = 2 m_{12} = 2 \rangle \quad = \quad |\ell_1
= 1 m_1 = 1 \ell_2 m_2 = 1 \rangle 
\end{align} \vspace{3px}


These two states are identified with each other  and in each representation
they are the unique states with the $L_z$ eigenvalue of $2\hbar$. Consequently,
(with the sign in this part of the CS phase convention), 

\[
\langle ( \ell_1 = 1 \ell_2=1 ) \ell_{12} = 2 m_{12} = 2 | \ell_1 = 1 m_1
= 1 \ell_2 = 1 m_2 = 1 \rangle = 1
\] \vspace{3px}

Now we lower both sides of Equation \ref{cg_1}. $L_{12-} = L_{1-} + L_{2-}$,
and we use the left form on the left and right form on the right. This yields 

\begin{align} \label{cg_2}
  2\hbar |(\ell_1 = 1 \ell_2 = 1) \ell_{12} = 2 m_{12} = 1 \rangle
  = \sqrt{2}\hbar (|\ell_1 = 1 m_1 = 0 el_2 = 1 m_2 = 1\rangle + |\ell_1 m_1
  = 1\ell_2 = 1 m_2 = 0 \rangle )
\end{align}\vspace{3px}

If we contract both sides of Equation \ref{cg_2} with $\hat{\langle \ell_1
= 1 m_1 = 0 \ell_2 = 1 m_2 = 1|}$ we obtain 

\[
\langle \ell_1 = 1 m_1 = 0 \ell_2 = 1 m_2 = 1 | (\ell_1 = 1 \ell_2 = 1)
\ell_{12} = 2 m_{12} = 1 \rangle = \frac{1}{\sqrt{2}}
\] \vspace{3px}

and if we contract both sides of Equation \ref{cg_2} with $\langle \ell_1
= 1 m_1 = 1 \ell_2 = 1 m_2 = 0|$ we obtain 

\[
\langle \ell_1 m_1 = 1 \ell_2 = 1 m_2 = 0 | (\ell_1 = 1 \ell_2 = 1 ) \ell_{12}
= 2 m_{12} = 1 \rangle = \frac{1}{\sqrt{2}}
\] \vspace{3px}

But examining Equation \ref{cg_2} we see that there are two other normalized
states of $L_z = 1$ that do not appear, that are orthogonal to the states in
\ref{cg_2}, and thus these must be uniquely identified with each other. That
is, we find 

\begin{align} \label{cg_3}
|(\ell = 1 \ell_2 = 1)\ell_{12} = 1 m_{12} = 1 \rangle = \frac{1}{\sqrt{2}}
( - |\ell_1 = 1 m_1 = 0 \ell_2 = 1 m_2 = 1 \rangle + |\ell_1 = 1 m_1 = 1 \ell_2
= 1 m_2 = 0 \rangle ) 
\end{align} \vspace{3px}

from which we deduce

\begin{align} \label{}
  \langle \ell_1 = 1 m_1 = 0 \ell_2 = 1 m_2 = 1 | (\ell_1 = 1 \ell_2
  = 1 ) \ell_{12} = 1 m_{12} = 1 \rangle &= -\frac{1}{\sqrt{2}} \\ 
  \langle \ell_1 = 1 m_1 = 1 \ell_2 = 1 m_2 = 0 | (\ell_1 =1 \ell_2 = 1)
  \ell_{12} = 1 m_{12} = 1 \rangle &= \frac{1}{\sqrt{2}}
\end{align}\vspace{3px}

Notice that, even with the assumption that the Clebsch-Gordan coefficients are
real, the sign chosen on the RHS of Equation \ref{cg_3} appears arbitrary. This
is again part of the CS phase convention: the component with the maximum $m$ on
the RHS was taken to have the positive sign. 

The point here is not to have you calculate all of the CG coefficients -- that
is what Mathematica is for -- but instead for you to understand that they can
be derived using properties of the lowering operator combined with a convention
for fixing various signs/phases. 


\section{Spin} 

\subsection{Some History}

In the early 1920s experimentalists studied the splitting of the lines of
hydrogen and other atoms when the atoms were placed in a magnetic field. The
external field defines a direction, and thees breaks rotational symmetry, and
as a consequence, previously degenerate states split into their magnetic sub
components, $-\ell \leq m \leq \ell$. The number of distinct lines produced
exceeded the number that would be expected based on the description of the
hydrogen atom we have developed so far. 

\begin{figure}[H]
  \centering
    \includegraphics[width =7cm]{zeemansplit.pdf}
    \caption{Zeeman Splittings in Hydrogen}
    \label{zeemansplit}
\end{figure}

Alfred Lande, an experimentalist based at T\"ubingen University, had developed
an empirical model describing the Zeeman splittings, not based in theory. Pauli
had concluded the necessary theory was one that introduced an additional
electron quantum number that could take on only two values. A 20-year-old
student Kronig, visiting Lande at the time Pauli also came for discussions,
interpreted Pauli's idea as an electron having a ``spin" $s = \frac{1}{2}$.
Kronig shared his ideas with Pauli, whose objections included 

\begin{itemize}
  \item[1.] the discrepant values needed for $g_s$. 
  \item[2.] the physical requirements for generating the needed magnetic moment
    were unrealistic -- Pauli argued that the only radius one could construct
    for the electron, $r \sim \frac{e^2}{mc^2} \sim \alpha \frac{\hbar
    c}{mc^2}$ was so small that the electron's surface would need to rotate at
    100s of $c$ to generate the magnetic moment. 
\end{itemize}

Kronig's idea was also met with skepticism from Niels Bohr, and was never
published. 

Nine month's later the same idea was generated by two young theorists in
Holland, graduate students Uhlenbeck and Goudsmit, who described the notion of
spin to Ehrenfest. He told the two that the idea was ``either nonsense, or
something important" and urged them to write a paper. Uhlenbeck and Goudsmit
consulted Lorentz, the leading Dutch theorist of the time, who raised
objections similar to those of Pauli. They returned to Ehrenfest, asking him to
return their paper, but he had already submitted it, and advised his students
not to worry as they were young enough to be forgiven for stupidity. 

In 1926, Thomas identified a relativistic correction, a spin-orbit contribution
to the Hydrogen atom Hamiltonian that removed the need for a state-dependent
magnetic moment -- they key objection both Pauli and Lorentz raisd. That plus
the acceptance that the electron spin was not some analog of classical spin,
but a uniquely quantum phenomena of the electron, resolved earlier objections. 


\subsection{Quantum Mechanical Spin} 

Spin in Quantum Mechanics is defined by an algebra borrowed from orbital
angular momentum: 

\[
  [\hat{S}_i, \hat{S}_j ]  = i\hbar \epsilon_{ijk} S_k \quad \Rightarrow \quad
  [\hat{S}_x, \hat{S}_y ] = i\hbar \hat{S}_z \quad [\hat{S}_y, \hat{S}_z]
  = i\hbar \hat{S}_x \quad [\hat{S}_z, \hat{S}_x ] = i\hbar \hat{S}_y 
\] \vspace{3px}

which leads to the familiar relations we have derived 

\[
\hat{S}^2 |sm_s\rangle = \hbar^2 s(s+1)|sm_s\rangle \quad \hat{S}_z |s m_s
\rangle = \hbar m_s | sm_s \rangle \quad  \hat{S}_\pm | sm_s \rangle = \hbar
\sqrt{s(s+1) - m_s (m_s \pm 1 )} |sm_s \pm 1 \rangle
\] \vspace{3px}

Elementary particles in the standard model come with an intrinsic spin. All
particles that make up physical matter have spin $s = \frac{1}{2}$ and are
called \textit{fermions}. They include charged \textit{leptons} (which only
have electromagnetic weak interactions)

\[
  e^- \quad e^+ \quad \mu^- \quad \mu^+ \quad \tau^- \quad \tau^+ 
\] \vspace{3px}

and their neutral partners, the \textit{neutrinos}

\[
  \nu_e \quad \bar{\nu}_e \quad \nu_\mu \quad \bar{\nu}_\mu \quad \nu_\tau
  \quad \bar{\nu}_\tau 
\] \vspace{3px}

and the \textit{quarks}, which have strong interactions and their antiparticles 

\[
  u \quad d \quad s \quad c \quad t \quad b \qquad \qquad \bar{u} \quad
  \bar{d} \quad \bar{s} \quad \bar{c} \quad \bar{t} \quad \bar{b} 
\] \vspace{3px}

We also have the particles that mediate the forces, which carry spin $s = 1$, 

\[
\text{weak interactions: } W^\pm, \; Z \qquad \text{ electromagnetic
interactions: } \gamma \qquad \text{ strong interactions: } g 
\] \vspace{3px}

and finally we have the \textit{Higgs boson} which generates the masses of the charged
fermions, which is the standard model's only spinless $s = 0$ particle. 

The nucleon is a ``composite fermion" -- like an elementary fermion, only one
composite fermion at a time can occupy a given quantum state. Its spin is made
up of the spin of its elementary quark components, the angular momentum of the
quarks, and the ``glue" (gluons) that hold the nucleon together. 


\subsection{Spin $\frac{1}{2}$}

The $s = \frac{1}{2}$ case is the most interesting for us as this is the
intrinsic spin of the electron and all other elementary fermions, and because
of its simplicity. First envision an arbitrary spin state, which we might
denote $|\vec{s}\rangle$. Now we select a basis by first introducing
a coordinate system, and choosing $\hat{s}$ and $s_z$ as our eigenstate labels.
Then, with the basis, our state (assumed normalized) can be represented 

\[
|\vec{s}\rangle = \sum_{m_s}^{} | \frac{1}{2} m_s \rangle \langle \frac{1}{2}
m_s | \vec{s}\rangle
\] \vspace{3px}

There are two $\hat{s}_z$ basis states can be expressed in terms of what are
called Pauli spinors: 

\[
|\frac{1}{2}m_s = \frac{1}{2}\rangle \leftrightarrow \begin{pmatrix}
  1 \\ 0
\end{pmatrix} \equiv \chi_+ \qquad | \frac{1}{2} m_s = -\frac{1}{2}\rangle
\leftrightarrow \begin{pmatrix}
  0 \\ 1
\end{pmatrix} \equiv \chi_-
\] \vspace{3px}


Here $\chi_+$ and $\chi_-$ are the two special states where the spin points up
or down, respectively with respect to the chosen $z$ axis. The general
normalized spin-1/2 state can be written 

\[
  \chi_{\alpha, \beta} = \begin{pmatrix}
    \alpha \\ \beta
  \end{pmatrix} = \alpha\chi_+ + \beta\chi_- \quad \text{ with } |\alpha|^2
  + |\beta|^2 = 1
\] \vspace{3px}

That is, in the first equation above, the expansion coefficients in this basis
are 

\[
\langle \frac{1}{2} m_s = \frac{1}{2} \rangle \vec{s} \rangle = \alpha \qquad
\langle \frac{1}{2} m_s = -\frac{1}{2} | \vec{s} \rangle = \beta
\] \vspace{3px}

From the general expression of spin, one has for the case of $s=\frac{1}{2}$, 

\begin{align} \label{}
  \hat{S}^2 |s m_s \rangle = \hbar^2 s(s+1)|s m_s \rangle \quad &\Rightarrow
  \quad \hat{S}^2 \chi_+ = \frac{3\hbar^2}{4}\chi_+ \quad \hat{S}^2 \chi_-
  = \frac{3\hbar^2}{4}\chi_- \\ 
  \hat{S}_z|s m_s \rangle = \hbar m_s | s m_s \rangle \quad &\Rightarrow \quad
  \hat{S}_z \chi_+ = \frac{\hbar}{2}\chi_+ \quad \hat{S}_z \chi_-
  = -\frac{\hbar}{2}\chi_-
\end{align}\vspace{3px}

so in the spin-1/2 representations,  $\hat{S}^2$ and $\hat{S}_z$ correspond to
the matrices 

\[
\hat{S}^2 = \frac{3\hbar^2}{4} \begin{pmatrix}
  1 & 0 \\ 0 & 1 
\end{pmatrix} \qquad \hat{S}_z = \frac{\hbar}{2} \begin{pmatrix}
  1 & 0 \\ 0 & -1 
\end{pmatrix} 
\] \vspace{3px}


Similarly, 

\begin{align} \label{}
  \hat{S}_+|s m_s \rangle = \hbar \sqrt{s(s+1) - m_s (m_s + 1)} | s m_s
  + 1 \rangle \quad &\Rightarrow \quad \hat{S}_+ \chi_+ = 0 \quad \hat{S}_+
  \chi_- = \hbar \chi_+ \\ 
  \hat{S}_-|s m_s \rangle = \hbar \sqrt{s(s+1) - m_s(m_s - 1)}|s m_s
  + 1 \rangle \quad &\Rightarrow \quad \hat{S}_-\chi_+ = \hbar \chi_- \quad
  \hat{S}_-\chi_- = 0
\end{align}\vspace{3px}


Therefore, 

\[
\hat{S}_+ = \hbar \begin{pmatrix}
  0 & 1 & 0 & 0 
\end{pmatrix} \qquad \hat{S}_- = \hbar \begin{pmatrix}
  0 & 0& 1 & 0
\end{pmatrix} 
\] \vspace{3px}

And as $\hat{S}_\pm = \hat{S}_x \pm i \hat{S}_y$, then $\hat{S}_x = \frac{1}{2}
( \hat{S}_+ + \hat{S}_- )$ and $\hat{S}_y = \frac{1}{2i} (\hat{S}_+
- \hat{S}_-)$, or 

\[
\hat{S}_x = \frac{\hbar}{2} \begin{pmatrix}
  0 & 1 & 1 & 0
\end{pmatrix} \qquad \hat{S}_y = \frac{\hbar}{2} \begin{pmatrix}
  0 & -i & i & 0
\end{pmatrix} 
\] \vspace{3px}

Conventionally, $\hat{S}_x, \hat{S}_y,$ and $\hat{S}_z$ are expressed in terms
of the Pauli matrices $\vec{\sigma}$ as 

\begin{subbox}{Pauli Spin Matrices}
  \[
  \hat{S} = \frac{\hbar}{2} \vec{\sigma} \qquad \qquad \sigma_x = \begin{pmatrix}
    0 & 1 \\ 1 & 0 
  \end{pmatrix} \qquad \sigma_y = \begin{pmatrix}
    0 & -i \\ i & 0 
  \end{pmatrix} \qquad \sigma_z = \begin{pmatrix}
    1 & 0 \\ 0 & -1
  \end{pmatrix} 
  \] 
\end{subbox}

These are Hermitian operators, as are the matrices representing observables
$\hat{S}^2$ and $\hat{S}_z$.


\subsection{Spin Measurements} 

We noted above that the most general normalized statevector can be expressed as 

\[
  \chi_{\alpha, \beta} \equiv \alpha \chi_+ + \beta \chi_- = \begin{pmatrix}
    \alpha \\ \beta
  \end{pmatrix} , \qquad |\alpha|^2 + |\beta|^2 = 1
\] \vspace{3px}

We will be interested in measuring spin relative to some direction, e.g., the
$x, y$ or $z$ axis. And when we do any such measurement, the answer is binary
-- either $+\frac{\hbar}{2}$ or $-\frac{\hbar}{2}$, with respective
probabilities that will depend on the state we are in. The $z$ axis is
particularly simple because its the axis of our angular momentum quantization.
If we calculate the expectation value of $\hat{S}_z$ in the general state
described above,
 
\[
  \langle \hat{S}_z \rangle \equiv \chi_{\alpha, \beta}^\dagger \hat{S}_z
  \chi_{\alpha, \beta} = (\alpha^* \; \; \beta^*) \frac{\hbar}{2} \begin{pmatrix}
    1 & 0 \\ 0 & -1
  \end{pmatrix} \begin{pmatrix}
    \alpha \\ \beta
  \end{pmatrix} = \frac{\hbar}{2} (|\alpha|^2 - |\beta|^2 )
\] \vspace{3px}

From this result, we can see choices corresponding to states of definite spin
along $z$ -- the case where every measurement will give the same result, either
always $\hbar / 2$ or always $-\hbar / 2$ -- are 

\[
\alpha =1, \; \beta = 0 \quad \text{ corresponds to spin aligned with $z $:
} \quad \chi_+ =  \begin{pmatrix}
  1\\0
\end{pmatrix} 
\] \[ \alpha = 0, \; \beta = 1 \quad \text{ corresponds to spin anti-aligned
with $z$: } \quad \chi_- = \begin{pmatrix}
  0 \\ 1
\end{pmatrix} \] \vspace{3px}


(up to an overall phase). Of course, this was obvious from the start. 


But now we can do the same thing with $x$ to find the states aligned with or
against $x$. Taking the expectation value of $\hat{S}_x$, 

\[
  \langle \hat{S}_x \rangle \equiv \chi_{\alpha, \beta}^\dagger \hat{S}_x
  \chi_{\alpha, \beta} = (\alpha^* \; \; \beta^* ) \frac{\hbar}{2}\begin{pmatrix}
    0 & 1 \\ 1 & 0 
  \end{pmatrix} \begin{pmatrix}
    \alpha \\ \beta
  \end{pmatrix} = \frac{\hbar}{2} (\alpha^* \beta + \alpha \beta^* ) = \hbar
  \text{Re}[\alpha \beta^*] 
\] \vspace{3px}

So the state with $\alpha = \beta = \frac{1}{\sqrt{2}}$ has $\langle \hat{S}_x
\rangle = \frac{\hbar}{2}$ and the state with $\alpha = -\beta
= \frac{1}{\sqrt{2}}$ has $\langle \hat{S}_x\rangle = -\frac{\hbar}{2}$. Thus

\[
\chi_+^x = \begin{pmatrix}
  \frac{1}{\sqrt{2}} \\ \frac{1}{\sqrt{2}} \end{pmatrix} \quad \chi_-^x = \begin{pmatrix}
    \frac{1}{\sqrt{2}} \\ - \frac{1}{\sqrt{2}}
  \end{pmatrix} \quad \Rightarrow \quad \chi_+ = \frac{1}{\sqrt{2}} (\chi_+^x
  + \chi_-^x) \quad \chi_- = \frac{1}{\sqrt{2}} (\chi_+^x - \chi_-^x)
\] \vspace{3px}

For a system quantizing along $z$, $\chi_+^x$ and $\chi_-^x$ are states
pointing up and down along $x$. And we can do the exercise a third time, taking
the expectation value of $\hat{S}_y$ in the general state: 

\[
  \langle \hat{S}_y \rangle \equiv \chi_{\alpha, \beta}^\dagger \hat{S}_y
  \chi_{\alpha, \beta} = (\alpha^* \;\; \beta^*) \frac{\hbar}{2}\begin{pmatrix}
    0 & -i \\ i & 0 
  \end{pmatrix} \begin{pmatrix}
    \alpha \\ \beta
  \end{pmatrix} = \frac{\hbar}{2} (-i\alpha^*\beta + i\alpha \beta^*) = \hbar
  \text{Re}[i\alpha \beta^*]
\] \vspace{3px}

So the state with $\alpha = \frac{1}{\sqrt{2}}, \beta = \frac{i}{\sqrt{2}}$ has
$\langle \hat{S}_y \rangle = \frac{\hbar}{2}$ and the state with $\alpha
= \frac{1}{\sqrt{2}}, \beta = -\frac{i}{\sqrt{2}}$ has $\langle \hat{S}_y
\rangle = -\frac{\hbar}{2}$. Therefore 

\[
\chi_+^y = \begin{pmatrix}
  \frac{1}{\sqrt{2}} \\ \frac{i}{\sqrt{2}} 
\end{pmatrix} \quad \chi_-^y = \begin{pmatrix}
  \frac{1}{\sqrt{2}} \\ - \frac{i}{\sqrt{2}}
\end{pmatrix} \quad \Rightarrow \quad \chi_+ = \frac{1}{\sqrt{2}} (\chi_+^y
+ \chi_-^y ) \quad \chi_- = \frac{i}{\sqrt{2}} (\chi_-^y - \chi_+^y)
\] \vspace{3px}

So our general state can be expressed in several equivalent ways

\begin{mainbox}{General Quantum Statevector}
  \[ \chi_{\alpha, \beta} \equiv \alpha \chi_+ + \beta \chi_- = \left(
    \frac{\alpha + \beta}{\sqrt{2}} \right) \chi_+^x + \left( \frac{\alpha
  - \beta}{\sqrt{2}} \right) \chi_-^x = \left( \frac{\alpha - i\beta}{\sqrt{2}}
\right) \chi_+^y + \left( \frac{\alpha + i\beta}{\sqrt{2}} \right) \chi_-^y \]     
\end{mainbox}


\subsection{Example -- Wave Packet Collapse}

These results allow one to quickly compute various examples, including the
collapse of a wave packet. 

\begin{itemize}
  \item[1.] Suppose the initial state is pointed along the positive $z$ axis --
    that is, the initial state is $|\chi_+\rangle$. Then a measurement of
    $\hat{S}_x$ yields $+\frac{\hbar}{2}$ and $-\frac{\hbar}{2}$ with
    probabilities 

    \[
    |\langle \chi_+^x | \chi_+\rangle |^2 = | \frac{1}{\sqrt{2}}(\chi_+
    + \chi_-)| \chi_+ \rangle|^2 = \frac{1}{2} \quad \text{and} \quad |\langle
    \chi_-^x | \chi_+\rangle |^2 = | \langle \frac{1}{\sqrt{2}} (\chi_+
    - \chi_-)|\chi_+\rangle |^2 = \frac{1}{2}
    \] \vspace{3px}
  \item[2.] If the initial state is $\chi_+$, then a measurement of $\hat{S}_y$ 
    yields $+\frac{\hbar}{2}$ and $-\frac{\hbar}{2}$ with probabilities 

    \[
    |\langle \chi_+^y | \chi_+ \rangle |^2 = | \langle \frac{1}{\sqrt{2}}
    (\chi_+ + i\chi_-) | \chi_+ \rangle |^2 = \frac{1}{2} \quad \text{and}
    \quad |\langle \chi_-^y | \chi_+ \rangle |^2 = |\langle \frac{1}{\sqrt{2}}
    (\chi_+ - i\chi_- )|\chi_+\rangle |^2 = \frac{1}{2}
    \] \vspace{3px}
    
  \item[3.] And collapse: the state is initially $|\chi_+\rangle$, but
    a measurement of $\hat{S}_x$ is done yielding the answer
    $|\chi_+^x\rangle$. If another measurement is done to see if the spin is
    aligned along $z$, the answer will be 

    \[
      |\langle \chi_+ | \chi_+^x \rangle |^2 = | \langle \frac{1}{\sqrt{2}}
      (\chi_+^x + \chi_-^x )|\chi_+^x \rangle |^2 = \frac{1}{2}
    \] \vspace{3px}
    
    So half the time the electron will be anti-aligned with $z$, even though it
    started out aligned.
\end{itemize}

\section{Larmor Precession \& Prime Directive}

A charged particle with spin generates a magnetic dipole moment $\mu$
proportional to its spin 

\[
\mu = \gamma \vec{S}
\] \vspace{3px}

where $\gamma$ is the geomagnetic ratio (nearly exactly $e/m$ for the
electron). This magnetic momentum responds to an applied magnetic field
$\vec{B}$, yielding 

\[
\hat{H} = -\mu \cdot \vec{B} = -\gamma \vec{B} \cdot \hat{\vec{S}}
\] \vspace{3px}

If we define our $z$-axis along $\vec{B} = B \hat{z}$, this becomes 

\[
\hat{H} = -\gamma B \hat{S}_z = -\frac{\gamma \hbar B}{2} \begin{pmatrix}
  1 & 0 \\ 0 & -1
\end{pmatrix} 
\] \vspace{3px}

yielding the stationary states and eigenvalues 

\[
\chi_+, \quad E_+ = -\frac{\gamma B \hbar}{2}\qquad \chi_-, E_- = \frac{\gamma
B \hbar}{2}
\] \vspace{3px}

so $\chi_+$, with its spin pointing along the  $z$-axis, in the direction of
$\vec{B}$, has the lower energy. At time $t=0$, if we are given an arbitrary
wave packet 

\[
\chi(0) = \alpha \chi_+ + \beta\chi_- \equiv \cos \left( \frac{\theta}{2}
\right) \chi_+ + \sin \left( \frac{\theta}{2} \right) \chi_-  
\] \vspace{3px}

where we have used the constraint $|\alpha|^2 + |\beta|^2 = 1$ and the ability
to absorb phases into the basis states to write this in terms of a ``mixing
angle". This angle is the one between $\vec{B}$ and the spin. The prime
directive then gives us the general solution for how this wave packet evolves
in time in the presence of $\vec{B}$, with the $z$-axis pointing along
$\vec{B}$: 

\[
\chi(t) = \alpha \chi_+ + \beta \chi_- \equiv \cos \left( \frac{\theta}{2}
\right) e^{i\gamma B t / 2} + \sin \left( \frac{\theta}{2} \right) \chi_-
e^{-i\gamma Bt / 2} = \begin{pmatrix}
  \cos \left( \frac{\theta}{2} \right) e^{i\gamma Bt / 2} \\ \sin \left(
  \frac{\theta}{2} \right) e^{-i\gamma Bt/2}  
\end{pmatrix}     
\] \vspace{3px}

Just as we did previously, we can calculate the expectation value of $\hat{S}_x$

\[
\langle \hat{S}_x(t)\rangle = \chi(t)^\dagger \frac{\hbar}{2} \begin{pmatrix}
  0 & 1 \\ 1 & 0
\end{pmatrix} \chi(t) = \frac{\hbar}{2}\sin \theta \cos \gamma B t
\] \vspace{3px}

and also $\hat{S}_y$, 

\[
\langle \hat{S}_y(t) \rangle = \chi(t)^\dagger \frac{\hbar}{2} \begin{pmatrix}
  0 & -i \\ i & 0 
\end{pmatrix} \chi(t) = -\frac{\hbar}{2}\sin\theta \sin \gamma Bt
\] \vspace{3px}

and also $\hat{S}_z$, 

\[
\langle \hat{S}_z(t) \rangle = \chi(t)^\dagger \frac{\hbar}{2} \begin{pmatrix}
  1 & 0 \\ 0 & -1
\end{pmatrix} \chi(t) = \frac{\hbar}{2} \cos \theta
\] \vspace{3px}

These are the formulae for a classical precessing gyroscope. The angle $\theta$
and thus the projection on the $z$-axis remains constant while the projection
onto the $x-y$ plane precesses with an angular frequency $\gamma B$, the Larmor
frequency. 

\section{Applications of the Addition of Angular Momenta}

An interesting example of what we discussed previously about the addition of
angular momenta is having two particles each with a spin-1/2. In general (not
only spin-1/2), states in the uncoupled representation are eigenstates of four
operators: 


\begin{align*}
(\hat{S}^{(1)})^2 \left| s_1 m_1 s_2 m_2 \right\rangle &= s_1(s_1 + 1)\hbar^2 \left| s_1 m_1; s_2 m_2 \right\rangle \\
(\hat{S}^{(2)})^2 \left| s_1 m_1 s_2 m_2 \right\rangle &= s_2(s_2 + 1)\hbar^2 \left| s_1 m_1; s_2 m_2 \right\rangle \\
\hat{S}^{(1)}_z \left| s_1 m_1 s_2 m_2 \right\rangle &= m_1\hbar \left| s_1 m_1; s_2 m_2 \right\rangle \\
\hat{S}^{(2)}_z \left| s_1 m_1 s_2 m_2 \right\rangle &= m_2\hbar \left| s_1 m_1; s_2 m_2 \right\rangle
\end{align*}


while in the coupled representation, 

\[
  (\hat{S}^{(1)})^2 |(s_1s_2) sm_s \rangle = s_1 (s_1 + 1)\hbar^2 |(s_1 s_2)
sm_s \rangle \]
\[ (\hat{S}^{(2)})^2 |(s_1s_2)sm_s \rangle = s_2 (s_2 + 1 ) \hbar^2
|(s_1s_2)sm_2 s \rangle \] 
\[
\hat{S}^2 \left| (s_1 s_2) s m_s \right\rangle = s(s + 1) \hbar^2 \left| (s_1
s_2) s m_s \right\rangle \] \[
\hat{S}_z \left| (s_1 s_2) s m_s \right\rangle = m_s \hbar \left| (s_1 s_2) s m_s \right\rangle
\] \vspace{3px}

where $\hat{S} = \hat{S}^{(1)} + \hat{S}^{(2)}$. 


Specializing to spin-1/2, a case that will be useful when we discuss the Helium
atom later, we can form either $s = 1$ (spin symmetric) triplet states of an $s
= 0$ (spin antisymmetric) singlet state. Using our Clebsch-Gordan tech, we
write these states in terms of their uncoupled equivalents. For the triplet
case, 

\[
  |\left( \frac{1}{2}\frac{1}{2} \right) s=1 m_s \rangle = \sum_{m_1m_2}
  \langle \frac{1}{2} m_1 \frac{1}{2} m_2 | \left( \frac{1}{2}\frac{1}{2}
  \right) sm_s \rangle | \frac{1}{2} m_1 \frac{1}{2} m_2 \rangle = \begin{cases}
    |\frac{1}{2}\frac{1}{2}\frac{1}{2}\frac{1}{2}\rangle m_s &= 1 \\
    \frac{1}{\sqrt{2}}|\frac{1}{2}\frac{1}{2}\frac{1}{2}-\frac{1}{2} \rangle
    + \frac{1}{\sqrt{2}} |\frac{1}{2}-\frac{1}{2}\frac{1}{2}\frac{1}{2} \rangle
    m_s &= 0 \\
    |\frac{1}{2} -\frac{1}{2}\frac{1}{2}-\frac{1}{2}\rangle m_s &= -1
  \end{cases}   
\] \vspace{3px}

While for the singlet case, 

\[
|\left( \frac{1}{2}\frac{1}{2} \right) s = 0 m_s = 0\rangle = \sum_{m_1m_2}^{}
\langle \frac{1}{2}m_1 \frac{1}{2}m_2 | \left( \frac{1}{2}\frac{1}{2} \right)
00\rangle | \frac{1}{2} m_1 \frac{1}{2}m_2 \rangle = \frac{1}{\sqrt{2}}
| \frac{1}{2}\frac{1}{2}\frac{1}{2}-\frac{1}{2} \rangle - \frac{1}{\sqrt{2}}
| \frac{1}{2} -\frac{1}{2} \frac{1}{2}\frac{1}{2} \rangle
\] \vspace{3px}

In summary 

\begin{subbox}{Symmetric Spin Triplet $(s = 1)$ States}
  \[
\left|\uparrow; \uparrow\right\rangle = \left| \frac{1}{2} \frac{1}{2}; \frac{1}{2} \frac{1}{2} \right\rangle = \left| \left( \frac{1}{2} \frac{1}{2} \right) s = 1 \, m_s = 1 \right\rangle \quad m_s = 1
\]

\[
\frac{1}{\sqrt{2}} \left( \left| \uparrow; \downarrow \right\rangle + \left| \downarrow; \uparrow \right\rangle \right) = \frac{1}{\sqrt{2}} \left( \left| \frac{1}{2} \frac{1}{2}; \frac{1}{2} - \frac{1}{2} \right\rangle + \left| \frac{1}{2} - \frac{1}{2}; \frac{1}{2} \frac{1}{2} \right\rangle \right) \equiv \left| \left( \frac{1}{2} \frac{1}{2} \right) s = 1 \, m_s = 0 \right\rangle \quad m_s = 0
\]

\[
\left| \downarrow; \downarrow \right\rangle = \left| \frac{1}{2} - \frac{1}{2}; \frac{1}{2} - \frac{1}{2} \right\rangle = \left| \left( \frac{1}{2} \frac{1}{2} \right) s = 1 \, m_s = -1 \right\rangle \quad m_s = -1
\]
\end{subbox}

\begin{mainbox}{Antisymmetric Spin Singlet $(s =0)$ State}
  \[
\frac{1}{\sqrt{2}} \left( \left| \uparrow; \downarrow \right\rangle - \left| \downarrow; \uparrow \right\rangle \right) = \left( \left| \frac{1}{2} \frac{1}{2}; \frac{1}{2} - \frac{1}{2} \right\rangle - \left| \frac{1}{2} - \frac{1}{2}; \frac{1}{2} \frac{1}{2} \right\rangle \right) \equiv \left| \left( \frac{1}{2} \frac{1}{2} \right) s = 0 \, m = 0 \right\rangle \quad m_s = 0
\]
\end{mainbox}


\section{Review}

\subsection{Clebsch-Gordan Coefficients}

Now that we have introduced quantum mechanical spin, we can ``close the loop"
on our previous discussion of Clebsch-Gordan coefficients, using
spin-$\frac{1}{2}$ as an example. The two cases we will address are the
coupling of two spins (very relevant to our future discussion of the ground
state of Helium with its two electrons) and the coupling of orbital angular and
spin momentum for a single electron (important for our subsequent treatment of
spin-orbit interactions in Helium). We will also summarize here the key
relationships you might need when utilizing Clebsch-Gordan coefficients. This
summary might be particularly helpful for those of you following the discussion
in Griffiths. Griffiths has a rather intimidating table of Clebsch-Gordan
coefficients -- that is, intimidating if your eyes are keen enough to read
print that small. Griffiths also uses a notation that doesn't help with
transparency. He writes 

\[
  |sm\rangle = \sum_{m_1+m_2=m}^{} C_{m_1m_2m}^{(s_1s_2)s} |s_1s_2m_1m_2\rangle
\] \vspace{3px}

A more common notation that I feel better captures the meaning of the
coefficients are the expansion coefficients of the states of the orthonormal
coupled basis in terms of the states of the orthonormal uncoupled basis - 

\[
  |(s_1 s_2 ) sm \rangle = \sum_{m_1, m_2, m_1 + m_2 = m} |s_1m_1s_2m_2\rangle
  \langle s_1m_1s_2m_2|(s_1s_2)sm\rangle
\] \vspace{3px}

Here $\vec{S} = \vec{S}^{(1)} + \hat{S}^{(2)}$, the basis states are
eigenstates of 

\[ \text{coupled -- basis: } \hat{S}^{(1)}, \hat{S}^{(2)}, \hat{S}, \hat{S}_z
\] 
\[
  \text{uncoupled basis: } \hat{S}^{(1)}, \hat{S}_z^{(1)}, \hat{S}^{(2)},
  \hat{S}_z^{(2)}
\] \vspace{3px}

As $\hat{S}_z = \hat{S}_z^{(1)} + \hat{S}_z^{(2)}$, we have the restriction in
the sum that $m = m_1 + m_2$. And the expansion coefficients -- analogous to
the $c_i$s when we expanded a wave packet in terms of a complete set of
stationary states -- are the Clebsch-Gordan coefficients. 

\[
\langle s_1m_1 s_2 m_2 | (s_1 s_2) sm \rangle
\] \vspace{3px}

We have previously described some of the properties of Clebsch-Gordan
coefficients. As 

\begin{align} \label{}
  \langle (s_1 s_2 ) s' m' | (s_1 s_2) sm \rangle &= \delta_{s's} \delta_{m'm}
  \quad \Rightarrow \quad \delta_{s's}\delta_{m'm}\\ &\Rightarrow
  \sum_{m_1,m_2, \;\; m_1+m_2
  = m}^{} |\langle s_1 m_1 s_2 m_2 | (s_1 s_2) sm \rangle |^2 = 1
\end{align}\vspace{3px}


But in fact the coefficients are defined as real, which is possible because
phases can be absorbed into the states. So 

\[
\langle s_1 m_1 s_2 m_2 | (s_1 s_2) sm \rangle^* = \langle s_1 m_1 s_2 m_2
| (s_1s_2) sm\rangle
\] \vspace{3px}

Also, we can expand the uncoupled states in terms of the coupled states, the
reverse of what we did above. 

\[
|s_1m_1s_2m_2\rangle = \sum_{s,m, \; \; m = m_1 + m_2}^{} |(s_1s_2) sm\rangle
\langle (s_1 s_2) sm | s_1m_1s_2m_2\rangle 
\] \vspace{3px}

Some further properties of Clebsch-Gordan coefficients that we have not
explicitly derived, but follow from the same discussion include 

\[ \langle j_1 m_1 j_2 m_2 | (j_1j_2) jm \rangle = (-1)^{j_1+j_2-j} \langle
j_2m_2 j_1 m_1 | (j_2 j_1 ) jm \rangle \] 
\[ \langle j_1 m_1 j_2 m_2 | (j_1j_2) jm \rangle = (-1)^{j_1+j_2-j} \langle j_1
  - m_1 j_2 - m_2 | (j_1 j_2) j - m\rangle \] \vspace{3px}


\section{Addition of Momenta}

\subsection{Addition of Two Spins with Spin $\frac{1}{2}$}

We have stressed that the Clebsch-Gordan coefficients describe the orthogonal
transformation between the uncoupled and coupled bases. We can write this out
explicitly f or the case of two spin-1/2 particles. 

i\[
\left(
\begin{array}{c}
\left| \left(\frac{1}{2} \frac{1}{2}\right) 1 \, 1 \right\rangle \\[3mm]
\left| \left(\frac{1}{2} \frac{1}{2}\right) 1 \, 0 \right\rangle \\[3mm]
\left| \left(\frac{1}{2} \frac{1}{2}\right) 0 \, 0 \right\rangle \\[3mm]
\left| \left(\frac{1}{2} \frac{1}{2}\right) 1 \, -1 \right\rangle
\end{array}
\right)
=
\left(
\begin{array}{cccc}
  1 & 0 & 0 & 0 \\[3mm]
0 & \sqrt{\frac{1}{2}} & \sqrt{\frac{1}{2}} & 0 \\[3mm]
0 & -\sqrt{\frac{1}{2}} & \sqrt{\frac{1}{2}} & 0 \\[3mm]
0 & 0 & 0 & 1
\end{array}
\right)
\left(
\begin{array}{c}
\left| \frac{1}{2} \frac{1}{2} \frac{1}{2} \frac{1}{2} \right\rangle \\[3mm]
\left| \frac{1}{2} \frac{1}{2} \frac{1}{2} -\frac{1}{2} \right\rangle \\[3mm]
\left| \frac{1}{2} -\frac{1}{2} \frac{1}{2} \frac{1}{2} \right\rangle \\[3mm]
\left| \frac{1}{2} -\frac{1}{2} \frac{1}{2} -\frac{1}{2} \right\rangle
\end{array}
\right)
\] \vspace{3px} 


This matrix is the matrix of corresponding Clebsch-Gordan coefficients

\[
\begin{pmatrix}
  \left\langle \frac{1}{2}\frac{1}{2}\frac{1}{2}\frac{1}{2}| \left(
  \frac{1}{2}\frac{1}{2} \right) 11 \right\rangle & 0 & 0 & 0 \\[3mm]
    0 & \left\langle \frac{1}{2}\frac{1}{2}\frac{1}{2}-\frac{1}{2}| \left(
    \frac{1}{2}\frac{1}{2} \right) 10\right\rangle & \left\langle
    \frac{1}{2}-\frac{1}{2}\frac{1}{2}\frac{1}{2} | \left(
\frac{1}{2}\frac{1}{2} \right) 1 0 \right\rangle & 0 \\[3mm] 
0 & \left\langle \frac{1}{2}\frac{1}{2}\frac{1}{2}-\frac{1}{2} | \left(
\frac{1}{2}\frac{1}{2} \right)  00 \right\rangle & \left\langle \frac{1}{2}-
\frac{1}{2}\frac{1}{2}\frac{1}{2} | \left( \frac{1}{2}\frac{1}{2} \right) 00
\right\rangle & 0 \\[3mm]
0 & 0 & 0 & \left\langle \frac{1}{2}-\frac{1}{2}\frac{1}{2}-\frac{1}{2}
  | \left( \frac{1}{2}\frac{1}{2} \right) 1 -1 \right\rangle  
\end{pmatrix}
\] \vspace{3px}

\subsection{Addition of Orbital Angular Momentum and Spin}

Here we consider the application to a single particle that has both spin and
angular momentum. We can add these to form a total angular momentum 

\[
\hat{j} = \hat{\ell} + \hat{s}
\] \vspace{3px}

For the spin-$\frac{1}{2}$ electron with definite $\ell$ there would be
$(2\ell+1)(2)$ possibilities for $j$. If $\ell = 0$ there is a single
possibility: 

\[
|(\ell = 0s) j = \frac{1}{2} m_j = \pm \frac{1}{2} \rangle
\] \vspace{3px}

But for $\ell > 0$, two multiplets are formed -- spin aligned and spin
antialigned. From the $2(2\ell+1)$ states, 

\[ (\ell s ) j = \ell + \frac{1}{2} m_j \rangle \quad 2\ell+2 \text{ states
} \] 
\[ |(\ell s) j = \ell - \frac{1}{2} m_j \rangle \quad 2\ell \text{ states } \]
\vspace{3px}


Thus in a hydrogen atom, once spin is considered, the states of good angular
momenta are 

\[ 1s_{1/2}: |n=1 \left(\ell = 0 s = \frac{1}{2}\right) j = \frac{1}{2}m_j
\rangle \] 
\[ 2s_{1/2}: |n=2 \left( \ell = 0 s = \frac{1}{2} \right) j = \frac{1}{2} m_j
  \rangle \quad 2p_{1/2}: |n = 2 \left( \ell=1 s = \frac{1}{2} \right)
  j = \frac{1}{2} m_j \rangle \quad 2p_{3/2}: |n = 2 \left( \ell
= 1 s = \frac{1}{2} \right) j = \frac{3}{2}m_j \rangle \] 
\[ 3s_{1/2}: |n = 2 \left(\ell = 0 s = \frac{1}{2} \right) j = \frac{1}{2}m_j
  \rangle \quad 3p_{1/2}: |n = 2 \left( \ell = 1 s = \frac{1}{2} \right)
  j = \frac{1}{2} m_j \rangle \quad 3p_{3/2}: |n=2 \left(\ell
= 1 s = \frac{1}{2} \right) j = \frac{3}{2}m_j \rangle \] 
\[ 3d_{3/2}: |n=2 \left( \ell = 2 s = \frac{1}{2} \right) j = \frac{3}{2}m_j
  \rangle \quad 3d_{5/2}: |n = 2 \left(\ell = 2s=\frac{1}{2}\right)
j = \frac{5}{2} m_j \rangle \] \vspace{3px} 

