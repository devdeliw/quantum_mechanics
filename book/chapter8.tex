
So far we have studied quantum mechanical problems that are either finite, or
involving functions in only one-dimension, described by the coordinate $x$. Our experience with 1D prepares us to solve a huge variety of 3D problems, many
of which can be reduced to effectively 1D properties by exploiting symmetries
like rotational invariance. 

\section{Schr\"odinger's Equation in 3D}

The generalization of the Schr\"odinger equation 

\[
\hat{H} | \Psi \rangle = i\hbar \frac{\partial \Psi}{\partial t} 
\] \vspace{3px}


for 3D is obtained by the replacement 

\[
\hat{H} = \frac{1}{2m} \hat{p}_x^2 + \hat{V}(\hat{x}) \rightarrow \frac{1}{2m}
\hat{\vec{p}}^2 + \hat{V}(\hat{x}, \hat{y}, \hat{z}) = \frac{1}{2m}
(\hat{p}_x^2 + \hat{p}_y^2 + \hat{p}_z^2) + \hat{V}(\hat{x}, \hat{y}, \hat{z})
\] \vspace{3px}

The operators $\hat{p}_x, \; \hat{p}_y, \; \hat{p}_z^2$ in position space are
just the natural generalization of the 1D case, 

\[
\hat{p}_x \rightarrow \frac{1}{i} \frac{\partial }{\partial x} \quad \hat{p}_y
\rightarrow \frac{\hbar}{i} \frac{\partial }{\partial y} \quad \hat{p}_z
\rightarrow \frac{\hbar}{i} \frac{\partial }{\partial z} 
\] \vspace{3px}

This can be expressed more compactly as the replacement

\[
\hat{p}_x \rightarrow \hat{\vec{p}} = \frac{\hbar}{i} \left( \hat{e}_x
\frac{\partial }{\partial x} , \, \hat{e}_y \frac{\partial }{\partial y} , \,
\hat{e}_z \frac{\partial }{\partial z}  \right)  = \frac{\hbar}{i} \nabla
\] \vspace{3px}

Here $\hat{e}_x, \, \hat{e}_y, \, \hat{e}_z$ are the unit vectors along
someone's favorite coordinate system, while $\nabla$ expresses the same thing,
without reference to a coordinate system. Thus the Schr\"odinger equation can
be written compactly as  

\begin{subbox}{3D Schr\"odinger's Equation}
  \[\left[ -\frac{\hbar^2}{2m} \nabla^2 + V(\vec{r}) \right] \Psi(\vec{r}, t)
  = i\hbar \frac{\partial }{\partial t} \Psi(\vec{r}, t) \] 
\end{subbox}

with  $r = (x, y, z)$ with our usual assumption that $V$ is not explicitly time
dependent, and with the Laplacian defined as 

\[
 \nabla^2 = \frac{\partial^2 }{\partial x^2} + \frac{\partial^2 }{\partial y^2}
 + \frac{\partial^2 }{\partial z^2} 
\] \vspace{3px}

\subsection{Normalization and the Prime Directive} 

We require that our wave functions in 3D represent a probability distribution
and thus require that 

\[
\int |\Psi|^2 \, d^3r = \int |\Psi|^2 \, dxdydz = \int |\Psi|^2 r^2 \, dr
d\Omega = 1, \quad \text{ where } d\Omega = \sin\theta \, d\theta d\phi
\] \vspace{3px}

Our strategy for solving problems in 3D will continue to be based on finding
the stationary states so we can implement the prime directive. 

\[
  \Psi(\vec{r}, t) \rightarrow \psi_n(\vec{r}) e^{-iE_n t / \hbar} \qquad
  \left[ -\frac{\hbar^2}{2m} \nabla^2 + V(\vec{r}) \right] \psi_n(\vec{r})
  = E_n \psi_n(\vec{r})
\] \vspace{3px}

Given a wave packet $\Psi(\vec{r}, 0)$, the prime directive generates
a solution of the Schr\"odinger equation with this initial condition 

\[
|\Psi(t)\rangle = \sum_{n}^{} |\psi_n\rangle \langle \psi_n| \Psi(0) \rangle
e^{-iE_n t /\hbar}
\] \vspace{3px}

which in position space becomes 

\[
\langle \vec{r}, \psi(t) \rangle = \sum_{n}^{} \langle \vec{r} | \psi_n\rangle
\langle \psi_n | \Psi(0) \rangle e^{-iE_nt/\hbar} \quad \Rightarrow \quad
\Psi(\vec{r}, t) = \sum_{n}^{} \psi_n(\vec{r}) \langle \psi_n | \Psi(0) \rangle
e^{-iE_n t / \hbar}
\] \vspace{3px}

\section{Infinite Cubical Box}

An interesting first problem is the infinite cubical box, the analog of the 1D
infinite square well, where 

\[
V(x, y, z) = \begin{cases}
  0 \quad &\frac{a}{2} < x, \,y,\, z < \frac{a}{2} \\
  \infty &\quad\;\;\text{otherwise} 
\end{cases} 
\] \vspace{3px}
The time-independent Schr\"odinger equation reads 

\[
-\frac{\hbar^2}{2m} \left[ \frac{\partial^2 }{\partial x^2}  + \frac{\partial^2
}{\partial y^2}  + \frac{\partial^2 }{\partial z^2} \right] \psi(x, y, z)
= E\psi(x, y, z)
\] \vspace{3px}

We write it in this form because the box is Cartesian, so the boundary
conditions will be easier to implement if we exploit that symmetry. We look for
a separable solution $\psi_{n_x, n_y, n_z} = \psi_{n_x} (x) \psi_{n_y} (y)
\psi_{n_z}(z)$. Substituting this in, the dividing by $\psi_{n_x} \psi_{n_y}
\psi_{n_z}(z)$ yields 

\[
  \frac{1}{\psi_{n_x}} \frac{d^2 \psi_{n_x}}{d^2 x^2} + \frac{1}{\psi_{n_y}}
  \frac{d^2 \psi_{n_y}}{d y^2} + \frac{1}{\psi_{n_z}} \frac{d^2 \psi_{n_z}}{d
  z^2} = -\frac{2m}{\hbar^2} E_{n_x, n_y, n_z} 
\] \vspace{3px}

Since each term on the left is independent and can be varied separately from
the others, while the term on the right is constant, it must be that each term
on the left is constant, that is, 

\[
  \frac{1}{\psi_{n_x}} \frac{d^2 \psi_{n_x}}{d x^2} = -k_{n_x}^2 \qquad
  \frac{1}{\psi_{n_y}} \frac{d^2 \psi_{n_y}}{d y^2} = -k_{n_y}^2 \qquad
  \frac{1}{\psi_{n_z}} \frac{d^2 \psi_{n_z}}{d z^2} = -k_{n_z}^2 \qquad
  E = \frac{\hbar^2}{2m} \left( k_{n_x}^2 + k_{n_y}^2 + k_{n_z}^2 \right)  
\] \vspace{3px}


But these are three 1D problems for which we know the solutions from our 1D
infinite square well. 

\[
\psi_n(\xi) = \begin{cases}
  &\sqrt{\frac{2}{a}} \cos \frac{\pi n \xi}{a} \quad n = 1,3,5, \hdots \text{
  even} \\ &\sqrt{\frac{2}{a}} \sin \frac{\pi n \xi}{a} \quad n = 2,4,6,\hdots
  \text{ odd}  
\end{cases}  \qquad \xi \in \{x, \, y, \, z\}
\] \[ k_n^2 = \frac{n^2\pi^2}{a^2} \] \vspace{3px}


  Consequently, 

  \[
    \psi_{n_x, n_y, n_z} (x, y, z) = \psi_{n_x} (x) \psi_{n_y}(y) \psi_{n_z}(z)
    \qquad E_{n_x, n_y, n_z} = \frac{\hbar^2 \pi^2}{2ma^2} ( n_x^2 + n_y^2
    + n_z^2)
  \] \vspace{3px}
  
  
  Thus to enumerate all the states and their energies, we need to enumerate all
  ordered triplets $(n_x, n_y, n_z)$. So there are many more possible states in
  3D! Note that (1, 1, 2) is distinct from (1, 2, 1) -- the orthonormality
  condition is given below. 


  One sees that this solution in 3D is labeled by quantum numbers from
  \textit{three commuting Hermitian operators} corresponding to three
  observables that can be measured simultaneously, 

  \[
  \hat{p}_x = \hbar \hat{k}_x \qquad \hat{p}_y = \hbar \hat{k}_y \qquad
  \hat{p}_z = \hbar \hat{k}_z
  \] \vspace{3px}
  
  This is an important theme in all of the 3D problems we do -- finding a maximal set of independent
but commuting operators whose eigenvalues then act as the wave function labels,
or quantum numbers. We showed previously that given two commuting Hermitian operators, a basis exists in
which the basis states are simultaneously eigenfunctions of both operators.
Note that in this case considered above, $\hat{H}$ is not an independent
operator, as $E$ is determined if one knows $p_x, p_y, p_z$. This contrasts
with the 1D case, where wave functions typically carry one label, which in most
cases is the energy eigenvalue. 

A couple of observations 

\begin{itemize}
  \item[1.] The solution is properly normalized

    \[
      \int |\psi_{n_x, n_y, n_z} (x, y, z) |^2 \, d\vec{r} = \left[
      |\psi_{n_x}(x)|^2 \, dx \right] \left[ \int |\psi_{n_y} (y) |^2 \right]
      \left[ \int | \psi_{n_z}(z) |^2 \right] = 1
    \] \vspace{3px}
    \item[2.] The solutions are orthonormal 
  
  \[
    \int \psi^*_{n_x', n_y', n_z'} (x, y, z) \psi_{n_x, n_y, n_z} (x, y, z) \,
    d\vec{r} = 
    \] \[ \left[ \psi_{n_x'}^* (x) \psi_{n_x}(x) \right] \left[ \psi_{n_y'}^*
    (y) \psi_{n_y}(y) \right]
    \left[ \psi_{n_z'}^* (z) \psi_{n_z}(z) \right] = \delta_{n_x', n_x}
    \delta_{n_y', n_y} \delta_{n_z', n_z} \] \vspace{3px}

  \item[3.] Unlike in 1D, there are energy degeneracies -- solutions are no
    longer solely identified by their energies. For example the states labeled 

    \[
      (n_x, n_y, n_z) = \{ (1, 1, 2),\, (1, 2, 1),\, (2, 1, 1)\}
    \] \vspace{3px}
    
    all have energy $E = \frac{3\hbar^2\pi^2}{ma^2}$. In 3D we generally will
    need to find additional quantum labels for our wave functions, as $E$ does
    not suffice. 
  \item[4.] The solutions provide a complete basis for representing any wave
    packet satisfying the same boundary conditions. This property plus 1. and
    2. allows us to implement the prime directive. 
  \item[5.] We made good use of parity in 1D and we have already exploited it
    here, solving the infinite cubic well. In 1D, the square well parity is
    $(-1)^{n+1}$. 

     \[
    \text{1D} : \quad \text{under } x \rightarrow -x, \; \psi_n(x) \rightarrow
    \psi_n(-x) = (-1)^{n+1}\psi_n(x)
    \] \vspace{3px}
    
    but in 3D, the cubic well parity is $(-1)^{n_x + n_y + n_z + 1}$. 

     \[
       \text{3D} : \quad \text{under } \{ x, y, z\} \rightarrow \{-x, -y, -z\},
       \; \psi_{n_x}(x) \psi_{n_y}(y) \psi_{n_z}(z) \rightarrow \psi_{n_x}(-x)
       \psi_{n_y}(-y) \psi_{n_z}(-z)
       \] \[ = (-1)^{n_x+ n_y+n_z+1} \psi_{n_x}(x)
     \psi_{n_y}(y) \psi_{n_z}(z)\]
\end{itemize}

These questions of normalization, orthonormality, completeness, and wave
function labeling will come up in every 3D problem we tackle. 






