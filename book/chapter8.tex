
So far we have studied quantum mechanical problems that are either finite, or
involving functions in only one-dimension, described by the coordinate $x$. Our experience with 1D prepares us to solve a huge variety of 3D problems, many
of which can be reduced to effectively 1D properties by exploiting symmetries
like rotational invariance. 

\section{Schr\"odinger's Equation in 3D}

The generalization of the Schr\"odinger equation 

\[
\hat{H} | \Psi \rangle = i\hbar \frac{\partial \Psi}{\partial t} 
\] \vspace{3px}


for 3D is obtained by the replacement 

\[
\hat{H} = \frac{1}{2m} \hat{p}_x^2 + \hat{V}(\hat{x}) \rightarrow \frac{1}{2m}
\hat{\vec{p}}^2 + \hat{V}(\hat{x}, \hat{y}, \hat{z}) = \frac{1}{2m}
(\hat{p}_x^2 + \hat{p}_y^2 + \hat{p}_z^2) + \hat{V}(\hat{x}, \hat{y}, \hat{z})
\] \vspace{3px}

The operators $\hat{p}_x, \; \hat{p}_y, \; \hat{p}_z^2$ in position space are
just the natural generalization of the 1D case, 

\[
\hat{p}_x \rightarrow \frac{1}{i} \frac{\partial }{\partial x} \quad \hat{p}_y
\rightarrow \frac{\hbar}{i} \frac{\partial }{\partial y} \quad \hat{p}_z
\rightarrow \frac{\hbar}{i} \frac{\partial }{\partial z} 
\] \vspace{3px}

This can be expressed more compactly as the replacement

\[
\hat{p}_x \rightarrow \hat{\vec{p}} = \frac{\hbar}{i} \left( \hat{e}_x
\frac{\partial }{\partial x} , \, \hat{e}_y \frac{\partial }{\partial y} , \,
\hat{e}_z \frac{\partial }{\partial z}  \right)  = \frac{\hbar}{i} \nabla
\] \vspace{3px}

Here $\hat{e}_x, \, \hat{e}_y, \, \hat{e}_z$ are the unit vectors along
someone's favorite coordinate system, while $\nabla$ expresses the same thing,
without reference to a coordinate system. Thus the Schr\"odinger equation can
be written compactly as  

\begin{subbox}{3D Schr\"odinger's Equation}
  \[\left[ -\frac{\hbar^2}{2m} \nabla^2 + V(\vec{r}) \right] \Psi(\vec{r}, t)
  = i\hbar \frac{\partial }{\partial t} \Psi(\vec{r}, t) \] 
\end{subbox}

with  $r = (x, y, z)$ with our usual assumption that $V$ is not explicitly time
dependent, and with the Laplacian defined as 

\[
 \nabla^2 = \frac{\partial^2 }{\partial x^2} + \frac{\partial^2 }{\partial y^2}
 + \frac{\partial^2 }{\partial z^2} 
\] \vspace{3px}

\subsection{Normalization and the Prime Directive} 

We require that our wave functions in 3D represent a probability distribution
and thus require that 

\[
\int |\Psi|^2 \, d^3r = \int |\Psi|^2 \, dxdydz = \int |\Psi|^2 r^2 \, dr
d\Omega = 1, \quad \text{ where } d\Omega = \sin\theta \, d\theta d\phi
\] \vspace{3px}

Our strategy for solving problems in 3D will continue to be based on finding
the stationary states so we can implement the prime directive. 

\[
  \Psi(\vec{r}, t) \rightarrow \psi_n(\vec{r}) e^{-iE_n t / \hbar} \qquad
  \left[ -\frac{\hbar^2}{2m} \nabla^2 + V(\vec{r}) \right] \psi_n(\vec{r})
  = E_n \psi_n(\vec{r})
\] \vspace{3px}

Given a wave packet $\Psi(\vec{r}, 0)$, the prime directive generates
a solution of the Schr\"odinger equation with this initial condition 

\[
|\Psi(t)\rangle = \sum_{n}^{} |\psi_n\rangle \langle \psi_n| \Psi(0) \rangle
e^{-iE_n t /\hbar}
\] \vspace{3px}

which in position space becomes 

\[
\langle \vec{r}, \psi(t) \rangle = \sum_{n}^{} \langle \vec{r} | \psi_n\rangle
\langle \psi_n | \Psi(0) \rangle e^{-iE_nt/\hbar} \quad \Rightarrow \quad
\Psi(\vec{r}, t) = \sum_{n}^{} \psi_n(\vec{r}) \langle \psi_n | \Psi(0) \rangle
e^{-iE_n t / \hbar}
\] \vspace{3px}

\section{Infinite Cubical Box}

An interesting first problem is the infinite cubical box, the analog of the 1D
infinite square well, where 

\[
V(x, y, z) = \begin{cases}
  0 \quad &\frac{a}{2} < x, \,y,\, z < \frac{a}{2} \\
  \infty &\quad\;\;\text{otherwise} 
\end{cases} 
\] \vspace{3px}
The time-independent Schr\"odinger equation reads 

\[
-\frac{\hbar^2}{2m} \left[ \frac{\partial^2 }{\partial x^2}  + \frac{\partial^2
}{\partial y^2}  + \frac{\partial^2 }{\partial z^2} \right] \psi(x, y, z)
= E\psi(x, y, z)
\] \vspace{3px}

We write it in this form because the box is Cartesian, so the boundary
conditions will be easier to implement if we exploit that symmetry. We look for
a separable solution $\psi_{n_x, n_y, n_z} = \psi_{n_x} (x) \psi_{n_y} (y)
\psi_{n_z}(z)$. Substituting this in, the dividing by $\psi_{n_x} \psi_{n_y}
\psi_{n_z}(z)$ yields 

\[
  \frac{1}{\psi_{n_x}} \frac{d^2 \psi_{n_x}}{d^2 x^2} + \frac{1}{\psi_{n_y}}
  \frac{d^2 \psi_{n_y}}{d y^2} + \frac{1}{\psi_{n_z}} \frac{d^2 \psi_{n_z}}{d
  z^2} = -\frac{2m}{\hbar^2} E_{n_x, n_y, n_z} 
\] \vspace{3px}

Since each term on the left is independent and can be varied separately from
the others, while the term on the right is constant, it must be that each term
on the left is constant, that is, 

\[
  \frac{1}{\psi_{n_x}} \frac{d^2 \psi_{n_x}}{d x^2} = -k_{n_x}^2 \qquad
  \frac{1}{\psi_{n_y}} \frac{d^2 \psi_{n_y}}{d y^2} = -k_{n_y}^2 \qquad
  \frac{1}{\psi_{n_z}} \frac{d^2 \psi_{n_z}}{d z^2} = -k_{n_z}^2 \qquad
  E = \frac{\hbar^2}{2m} \left( k_{n_x}^2 + k_{n_y}^2 + k_{n_z}^2 \right)  
\] \vspace{3px}


But these are three 1D problems for which we know the solutions from our 1D
infinite square well. 

\[
\psi_n(\xi) = \begin{cases}
  &\sqrt{\frac{2}{a}} \cos \frac{\pi n \xi}{a} \quad n = 1,3,5, \hdots \text{
  even} \\ &\sqrt{\frac{2}{a}} \sin \frac{\pi n \xi}{a} \quad n = 2,4,6,\hdots
  \text{ odd}  
\end{cases}  \qquad \xi \in \{x, \, y, \, z\}
\] \[ k_n^2 = \frac{n^2\pi^2}{a^2} \] \vspace{3px}


  Consequently, 

  \[
    \psi_{n_x, n_y, n_z} (x, y, z) = \psi_{n_x} (x) \psi_{n_y}(y) \psi_{n_z}(z)
    \qquad E_{n_x, n_y, n_z} = \frac{\hbar^2 \pi^2}{2ma^2} ( n_x^2 + n_y^2
    + n_z^2)
  \] \vspace{3px}
  
  
  Thus to enumerate all the states and their energies, we need to enumerate all
  ordered triplets $(n_x, n_y, n_z)$. So there are many more possible states in
  3D! Note that (1, 1, 2) is distinct from (1, 2, 1) -- the orthonormality
  condition is given below. 


  One sees that this solution in 3D is labeled by quantum numbers from
  \textit{three commuting Hermitian operators} corresponding to three
  observables that can be measured simultaneously, 

  \[
  \hat{p}_x = \hbar \hat{k}_x \qquad \hat{p}_y = \hbar \hat{k}_y \qquad
  \hat{p}_z = \hbar \hat{k}_z
  \] \vspace{3px}
  
  This is an important theme in all of the 3D problems we do -- finding a maximal set of independent
but commuting operators whose eigenvalues then act as the wave function labels,
or quantum numbers. We showed previously that given two commuting Hermitian operators, a basis exists in
which the basis states are simultaneously eigenfunctions of both operators.
Note that in this case considered above, $\hat{H}$ is not an independent
operator, as $E$ is determined if one knows $p_x, p_y, p_z$. This contrasts
with the 1D case, where wave functions typically carry one label, which in most
cases is the energy eigenvalue. 

A couple of observations 

\begin{itemize}
  \item[1.] The solution is properly normalized

    \[
      \int |\psi_{n_x, n_y, n_z} (x, y, z) |^2 \, d\vec{r} = \left[
      |\psi_{n_x}(x)|^2 \, dx \right] \left[ \int |\psi_{n_y} (y) |^2 \right]
      \left[ \int | \psi_{n_z}(z) |^2 \right] = 1
    \] \vspace{3px}
    \item[2.] The solutions are orthonormal 
  
  \[
    \int \psi^*_{n_x', n_y', n_z'} (x, y, z) \psi_{n_x, n_y, n_z} (x, y, z) \,
    d\vec{r} = 
    \] \[ \left[ \psi_{n_x'}^* (x) \psi_{n_x}(x) \right] \left[ \psi_{n_y'}^*
    (y) \psi_{n_y}(y) \right]
    \left[ \psi_{n_z'}^* (z) \psi_{n_z}(z) \right] = \delta_{n_x', n_x}
    \delta_{n_y', n_y} \delta_{n_z', n_z} \] \vspace{3px}

  \item[3.] Unlike in 1D, there are energy degeneracies -- solutions are no
    longer solely identified by their energies. For example the states labeled 

    \[
      (n_x, n_y, n_z) = \{ (1, 1, 2),\, (1, 2, 1),\, (2, 1, 1)\}
    \] \vspace{3px}
    
    all have energy $E = \frac{3\hbar^2\pi^2}{ma^2}$. In 3D we generally will
    need to find additional quantum labels for our wave functions, as $E$ does
    not suffice. 
  \item[4.] The solutions provide a complete basis for representing any wave
    packet satisfying the same boundary conditions. This property plus 1. and
    2. allows us to implement the prime directive. 
  \item[5.] We made good use of parity in 1D and we have already exploited it
    here, solving the infinite cubic well. In 1D, the square well parity is
    $(-1)^{n+1}$. 

     \[
    \text{1D} : \quad \text{under } x \rightarrow -x, \; \psi_n(x) \rightarrow
    \psi_n(-x) = (-1)^{n+1}\psi_n(x)
    \] \vspace{3px}
    
    but in 3D, the cubic well parity is $(-1)^{n_x + n_y + n_z + 1}$. 

     \[
       \text{3D} : \quad \text{under } \{ x, y, z\} \rightarrow \{-x, -y, -z\},
       \; \psi_{n_x}(x) \psi_{n_y}(y) \psi_{n_z}(z) \rightarrow \psi_{n_x}(-x)
       \psi_{n_y}(-y) \psi_{n_z}(-z)
       \] \[ = (-1)^{n_x+ n_y+n_z+1} \psi_{n_x}(x)
     \psi_{n_y}(y) \psi_{n_z}(z)\]
\end{itemize}

These questions of normalization, orthonormality, completeness, and wave
function labeling will come up in every 3D problem we tackle. 

\section{Quantum Mechanics with Spherical Symmetry}

\subsection{Symmetry and Spherical Coordinates}

Many of the potentials we deal with in physics are spherically symmetric,
requiring a different procedure for separating variables. In cases where the
Schr\"odinger equation takes the form 

\[
  \left[ -\frac{\hbar^2}{2m} \nabla^2 + V(r) \right] \psi(\vec{r})
  = E\psi(\vec{r}),
\] \vspace{3px}

instead of using Cartesian coordinates ($x, y, z$), one uses coordinates better
matched to the rotational symmetry of the problem -- spherical coordinates $(r,
\theta, \phi)$, where 

\[
x = r\sin\theta\cos\phi \quad y = r\sin\theta\sin\phi \quad z = r\cos\theta
\quad 0 \leq \theta \leq \pi \quad 0 \leq 0 \leq 2\pi 
\] \vspace{3px}

We separate the equation in spherical coordinates as follows

\[
  \left[ \left( \left\{ -\frac{\hbar^2}{2m} \nabla^2\right\}_r + V(r) \right)
  + \left\{ -\frac{\hbar^2}{2m} \nabla^2 \right\}_{\theta, \phi} \right]
  \psi(\vec{r}) = E\psi(\vec{r})
\] \vspace{3px}

and look for a solution of the form 

\[
\psi(\vec{r}) = R(r) Y(\theta, \phi)
\] \vspace{3px}

The angular solution $Y(\theta, \phi)$ will be a solution of Laplace's equation
restricted to the 2D unit sphere -- the second term in the Hamiltonian above --
and will be valid for any problem where the potential is spherically symmetric,
only depending on $r$. In contrast, the radial solution $R(r)$, corresponding
to the first term in the Hamiltonian above, will depend on the potential.
However, as an equation just in $r$, in general this solution is no more
complex than those we encountered in 1D Quantum Mechanics. So although the
separation takes a bit of algebra, the task we are about to undertake is
conceptually simple. 

The Laplacian in spherical coordinates is 

\[
\nabla^2 = \frac{1}{r^2} \frac{\partial }{\partial r}  \left( rse
\frac{\partial }{\partial r}  \right) + \frac{1}{r^2\sin\theta} \frac{\partial
}{\partial \theta} \left( \sin\theta \frac{\partial }{\partial \theta}  \right)
+ \frac{1}{r^2\sin^2\theta} \left( \frac{\partial^2 }{\partial \phi^2}  \right)  
\] \vspace{3px}

from which obtain the explicit separation of the Hamiltonian, 

\[
\left[ \left\{ -\frac{\hbar^2}{2m} \frac{1}{r^2} \frac{\partial }{\partial r}
  \left( r^2 \frac{\partial }{\partial r}  \right) + V(r) \right\}
  - \frac{\hbar^2}{2m} \left\{ \frac{1}{r^2\sin\theta} \frac{\partial }{\partial
    \theta}  \left( \sin\theta \frac{\partial }{\partial \theta}  \right)
    + \frac{1}{r^2\sin^2 \theta} \left( \frac{\partial^2 }{\partial \phi^2}
\right) \right\} \right] \psi(\vec{r}) = E\psi(\vec{r})   
\] \vspace{3px}

We substitute $\psi(\vec{r}) = R(r)Y(\theta, \phi)$, after which we multiply on
the left by $-mr^2 / (\hbar^2R(r) Y(\theta, \phi))$, similar to what we did
with the infinite cubical box. This yields 

\[
\left\{ \frac{1}{R} \frac{d }{d r} \left( r^2 \frac{d R}{d r}  \right)
- \frac{2mr^2}{\hbar^2} (V(r) - E) \right\} + \frac{1}{Y} \left\{
\frac{1}{\sin\theta} \frac{\partial }{\partial \theta} \left( \sin\theta
\frac{\partial Y}{\partial \theta}  \right) + \frac{1}{\sin^2\theta}\left(
\frac{\partial^2 Y}{\partial \psi^2}  \right) \right\} = 0   
\] \vspace{3px}


As the radial and angular terms are separate and can be varied independently,
a solution requires both terms to be a constant. It proves convenient to use as
that constant, $\ell ( \ell + 1)$. We then obtain 

\begin{align} \label{}
  \frac{1}{R} \frac{d }{d r} \left( r^2 \frac{d R}{d r}  \right)
  - \frac{2mr^2}{\hbar^2} (V(r) - E ) &= \ell(\ell + 1) \\  
  \frac{1}{Y} \left\{ \frac{1}{\sin\theta} \frac{\partial }{\partial \theta}
    \left( \sin\theta \frac{\partial Y}{\partial \theta}  \right)
    + \frac{1}{\sin^2\theta} \left( \frac{\partial^2 Y}{\partial \phi^2}
\right) \right\} &= -\ell(\ell +1)  
\end{align}\vspace{3px}


\subsection{The Spherical Harmonics} 

Before going into the details of finding the solutions $Y(\theta, \phi)$,
I should stress that conceptually we are doing something simple -- finding the
solutions of Laplace's equation restricted to the surface of a sphere. Once we
have those solutions, they will be common to all 3D QM problems in which the
potential depends only on $r$. 

We rewrite the angular equation as 

\[
\left\{ \sin\theta \frac{\partial }{\partial \theta}  \left( \sin\theta
  \frac{\partial }{\partial \theta}  \right) + \left( \frac{\partial^2
}{\partial \phi^2}  \right) \right\} Y(\theta, \phi) = -\ell(\ell+1) \sin^2
\theta Y(\theta, \phi)  
\] \vspace{3px}

then seek a separated solution 

\[
Y(\theta, \phi) = \Theta(\theta) \Phi(\phi)
\] \vspace{3px}

Plugging this in and dividing the left by $Y(\theta, \phi)$ yields 

\[
\frac{1}{\Theta(\theta)} \left\{ \sin \theta \frac{\partial }{\partial \theta}
  \left( \sin\theta \frac{\partial }{\partial \theta}  \right) + \ell(\ell+1)
\sin^2 \theta \right\} \Theta(\theta) + \frac{1}{\Phi(\phi)}\left(
\frac{\partial^2 }{\partial \phi^2}  \right) \Phi(\phi) = 0  
\] \vspace{3px}

Like we have seen before, each term can be varied independently, and so we must
require that each is a constant, which we will call $m^2$. This yields 

\[
\frac{1}{\Phi(\phi)} \left( \frac{d^2 }{d \phi^2}  \right)  \Phi(\phi) = -m^2
\qquad \frac{1}{\Theta(\theta)} \left\{ \sin\theta \frac{d }{d \theta} \left(
\sin\theta \frac{d }{d \theta}  \right) + \ell(\ell+1) \sin^2 \theta \right\}
\Theta(\theta) = m^2 
\] \vspace{3px}

The first equation requires 

\[
\frac{d^2 \Phi(\phi)}{d \phi^2} = -m^2 \Phi(\phi) \qquad \Rightarrow \qquad
\Phi(\phi) = e^{im\phi}
\] \vspace{3px}


There is also a boundary condition -- our wave function must be continuous, so
$\Phi(0) = \Phi(2\pi)$, so $m$ must be an integer, 

\[
m = 0, \pm 1, \pm 2,\pm 3, \hdots
\] \vspace{3px}

In the second equation, noting that 

\[
\sin\theta \frac{d }{d \theta}  = \sin\theta \frac{d \cos\theta}{d \theta}
\frac{d }{d \cos\theta} = -\sin^2 \theta \frac{d }{d \cos\theta}
= -(1-\cos^2\theta) \frac{d }{d \cos\theta}  
\] \vspace{3px}

we find 

\[
\left\{ (1 - \cos^2 \theta) \frac{d}{d \cos \theta} \left( (1 - \cos^2 \theta)
\frac{d}{d \cos \theta} \right) + \ell(\ell + 1)(1 - \cos^2 \theta) \right\}
\Theta(\cos \theta) = m^2 \Theta(\cos \theta) \Rightarrow
\]
\[
\left\{ (1 - \cos^2 \theta) \frac{d^2}{d \cos^2 \theta} - 2 \cos \theta \frac{d}{d \cos \theta} + \left( \ell(\ell + 1) - \frac{m^2}{1 - \cos^2 \theta} \right) \right\} \Theta(\cos \theta) = 0
\]




This is a famous equation whose solutions are single valued functions on the
interval $-1 \leq \cos \theta \leq 1$. Provided  $\ell$ is a nonnegative
integer, with $|m| \leq \ell$, it generates the associated Legendre functions 

\[
  \Theta(\cos \theta) = A_{\ell m} P_\ell^m (\cos\theta) 
\] \vspace{3px}

These functions are defined in terms of the Legendre Polynomials $P_\ell(x)$, 

\[
P_\ell^m (x) \equiv (-1)^{m} (1-x^2)^{m/2} \left( \frac{d }{d x}  \right)^{m}
P_\ell (x) \qquad P_\ell(x) = \frac{1}{2^{\ell} \ell!} \left( \frac{d }{d x}
\right)^{\ell} (x^2 - 1)^{\ell}  
\] \vspace{3px}

Once can generate the Legendre Polynomials from the starting values and
recursion relation 

\[
  P_0(x) = 1 \quad P_1(x) = x \quad P_{n+1} (x) = \frac{1}{n+1} [(2n + 1)x
  P_n(x) - nP_{n-1} (x) ]
\] \vspace{3px}

The overall solution up to normalization is thus 

\[
  \Phi(\phi)\Theta(\theta) = A_{\ell m} e^{im\phi} P_\ell^{m} (\cos\theta)
\] \vspace{3px}

The normalization is determined by the condition 

\[
\Psi(\vec{r}) \equiv R(r)\Phi(\phi)\Theta(\theta) \quad \Rightarrow \quad
1 = \int |\Psi(\vec{r})|^2 \, d\vec{r} = \int_{0}^{\infty} r^2 |R(r)|^2 \, dr
  \int |\Phi(\phi)\Theta(\theta)|^2 \, d\Omega = 1
\] \vspace{3px}

We impose the normalization constant on the radial and angular functions
separately, so 

\[
\int_{0}^{\infty} r^2|R(r)|^2 \, dr = 1 \qquad \int |\Phi(\phi)\Theta(\theta)
|^2 \, d\Omega = \int \sin\theta |A_{\ell m} P_\ell^{m}(\cos \theta)|^2 \,
d\theta d\phi =1
\] \vspace{3px}

The normalized solutions are called the \textit{spherical harmonics}, $Y_{\ell
m}(\theta, \phi)$. 

\begin{mainbox}{Spherical Harmonics}
  \[
  Y_{\ell m} \equiv \sqrt{\frac{2\ell+1}{4\pi} \frac{(\ell - m)!}{(\ell+m)!}}
  e^{im\phi} P_\ell^m (\cos\theta) \quad \int_{0}^{\pi} \sin\theta \, d\theta
  \int_{0}^{2\pi} |Y_{\ell m}(\theta, \phi)|^2 \, d\phi = 1
  \] 
\end{mainbox}

The $Y_{\ell m}$ are a complete orthonormal basis for the angular solutions of
Schr\"odinger equation in any case where $V$ is just a function of  $r$. The
orthonormality condition is 

\[
\int_{0}^{\pi} \sin\theta \, d\theta \int_{0}^{2\pi}  \, d\theta
\int_{0}^{2\pi} Y_{\ell' m'} (\theta,\phi) Y_{\ell m} \, d\phi = \delta{\ell'
\ell } \delta{m'm} \quad \ell = 0,1,2,\hdots \; m = -\ell, -\ell +1, \hdots,
\ell -1, \ell  
\] \vspace{3px}

From the formulas we have given above, one can verify 

\[
  Y_{\ell m}^* (\theta,\phi) = (-1)^{m} Y_{\ell -m} (\theta,\phi)
\] \vspace{3px}

Note the subscript on the right side isn't literally $\ell - m$, but $\ell
$ and $-m$ separately. 

The first few spherical harmonics are 

\[
Y_{00}(\theta, \phi) = \frac{1}{\sqrt{4\pi}}
\]

\[
Y_{1m}(\theta, \phi) = 
\begin{cases} 
-\sqrt{\frac{3}{8\pi}} \sin \theta e^{i\phi} & m = 1 \\
\sqrt{\frac{3}{4\pi}} \cos \theta & m = 0 \\
\sqrt{\frac{3}{8\pi}} \sin \theta e^{-i\phi} & m = -1
\end{cases}
\]

\[
Y_{2m}(\theta, \phi) = 
\begin{cases} 
\sqrt{\frac{15}{32\pi}} \sin^2 \theta e^{2i\phi} & m = 2 \\
-\sqrt{\frac{15}{8\pi}} \cos \theta \sin \theta e^{i\phi} & m = 1 \\
\sqrt{\frac{5}{16\pi}} (2\cos^2 \theta - \sin^2 \theta) & m = 0 \\
\sqrt{\frac{15}{8\pi}} \cos \theta \sin \theta e^{-i\phi} & m = -1 \\
\sqrt{\frac{15}{32\pi}} \sin^2 \theta e^{-2i\phi} & m = -2
\end{cases}
\]

But what are the spherical harmonics even about? Three important answers. 

\begin{itemize}
  \item[1.] They are the complete orthonormal basis for the solution of
    Laplace's equation on the unit sphere and consequently provide the angular
    wave function one needs to create a complete basis of stationary states for
    any problem in which $V$ solely depends on $r$. 
  \item[2.] They are the eigenstates of the total angular momentum operator
    $\vec{L}^2$ with eigenvalue $\hbar(\ell (\ell +1))$ and of the z-component
    of angular momentum $L_z$ with eigenvalue $m\hbar$ -- see below. 
  \item[3.] They provide additional quantum labels for our stationary states,
    augmenting the energy. 
\end{itemize}

\subsection{Angular Momentum}

The angular momentum operator in position space is 

\[
  \vec{L} = \vec{r} \times \vec{p} = \vec{r} \times \frac{\hbar}{i} \nabla
  \quad L_i = \frac{\hbar}{i} \epsilon_{ijk} r_j \partial_k, \quad \text{ where
  } \{i,j,k\} \in \{x, y, z \} 
\] \vspace{3px}

which we stress is an operator orthogonal to the radius vector and thus
associated with the tangential space of the sphere. We use the anti-symmetric
epsilon tensor, defined by 

\[
  \epsilon_{xyz} \equiv 1 \qquad \epsilon_{ijk} = -\epsilon{jik}
\] \vspace{3px}

This leads to 

\[
L_x = \frac{\hbar}{i} \left( y \frac{\partial }{\partial z} - z \frac{\partial
}{\partial y}  \right) \quad L_y = \frac{\hbar}{i} \left( z \frac{\partial
}{\partial x} - x \frac{\partial }{\partial z}  \right) \quad L_z
= \frac{\hbar}{i} \left( x \frac{\partial }{\partial y} - y \frac{\partial
}{\partial x}  \right)   
\] \vspace{3px}

And we can also define the total angular momentum operator 

\[
\vec{L}^2 \equiv \vec{L}_x^2 + \vec{L}_y^2 + \vec{L}_z^2
\] \vspace{3px}

Using these expressions, one can calculate the commutation relations among
these operators. One finds 

\[
  [\vec{L}^2, L_x] = [\vec{L}^2, L_y], [\vec{L}^2, L_z] = 0 \qquad [L_x, L_y]
  = i\hbar L_z \quad [L_y, L_z] = i\hbar L_x \quad [L_z, L_x] = i\hbar L_y
\] \vspace{3px}

Consequently one can select $\vec{L}^2$ and one of the others, by convention
this is $L_z$ -- as commuting Hermitian operators. One can form bases of
stationary states that are simultaneously eigenstates of both $\vec{L}^2$ and
$L_z$. 


In spherical coordinates, 

\[
L_x = i\hbar \left( \sin\phi \frac{\partial}{\partial \theta} + \cot\theta \cos\phi \frac{\partial}{\partial \phi} \right)
\]

\[
L_y = i\hbar \left( -\cos\phi \frac{\partial}{\partial \theta} + \cot\theta \sin\phi \frac{\partial}{\partial \phi} \right)
\]

\[
L_z = -i\hbar \frac{\partial}{\partial \phi}
\]

\[
\vec{L}^2 = L_x^2 + L_y^2 + L_z^2 = \hbar^2 \left( \frac{1}{\sin\theta} \frac{\partial}{\partial \theta} \left( \sin\theta \frac{\partial}{\partial \theta} \right) + \frac{1}{\sin^2 \theta} \frac{\partial^2}{\partial \phi^2} \right)
\]


With these spherical operator forms one can show 

\begin{subbox}{}
  The spherical harmonics are eigenstates of $\vec{L}^2$ with eigenvalues
  $\hbar^2\ell (\ell +1)$
  \[ \vec{L^2}Y_{\ell m} (\theta, \phi) = \hbar^2 \ell (\ell +1) Y_{\ell
  m}(\theta, \phi) \] \vspace{3px}

  The spherical harmonics are eigenstates of $L_z$ with eigenvalues $m\hbar$

   \[
     L_z Y_{\ell m} (\theta, \phi) = m\hbar Y_{\ell m}(\theta, \phi)
  \] 
\end{subbox}

The expression of $\vec{L}^2$ should look very familiar. We now see 

\[
\hat{H} = -\frac{\hbar^2}{2m}\nabla^2 + V(r) = -\frac{\hbar^2}{2m} \left[
\frac{1}{r^2} \frac{d }{d r} r^2 \frac{d }{d r} \right] + V(r)
+ \frac{1}{2mr^2}\vec{L}^2
\] \vspace{3px}


As $\vec{L}^2$ and $L_z$ only act on angular variables, $[\hat{H}, \vec{L}^2]
= [\hat{H}, L_z] = 0$ and $\vec{L}^2, L_z]  =0$. Thus $\hat{H}, \vec{L}^2$ and
$L_z$ are mutually commuting Hermitian operators. 

When we solve the 1D radial equation, which depends on $\ell $, we will
typically get solutions indexed by some radial quantum numbers we will call
$n$. Thus the energy eigenvalues can be labeled in general as $E_{n\ell} $.
Note that $\hat{H}$ does not depend on $L_z$, yet $L_z$ commutes with
$\vec{L}^2$. So $m$ is a wave function label, but as $E$ does not depend on
$m$, there is a $2\ell +1$ degeneracy of energy eigenvalues corresponding to
the possible values of $m$. Thus we conclude that the stationary state
solutions for any central potential will take the form 

\[ \psi = R_{n\ell }(r) Y_{\ell m}(\theta, \phi) \] 
\[ \hat{H}\psi_{n \ell m} = E_{n \ell } \psi_{n \ell m} \qquad \vec{L}^2
  \psi_{n \ell m} \psi_{n \ell m} = \hbar^2 \ell (\ell +1) \psi_{n \ell m}
  \qquad \hat{L}_z \psi_{n \ell m} \psi_{n\ell m} = m\hbar\psi_{n\ell m} \]
  \vspace{3px}

\section{3D Infinite Spherical Well}

We consider the spherical analog of the 3D infinite square well we discussed
previously. Namely, 

\[
 V(r) = \begin{cases}
   0 &\text{ if } r<a \\ \infty &\text{ otherwise }  
 \end{cases} 
\] \vspace{3px}

The radial equation for $r < a$ becomes 

\[
\left[ \frac{d }{d r} \left( r^2 \frac{d }{d r}  \right)
- \frac{2mr^2}{\hbar^2} (V(r) - E) \right] R_\ell  = \ell (\ell +1) R_\ell
\quad \Rightarrow \quad \left[ \frac{d }{d r} \left( r^2 \frac{d}{dr} \right)
+ \frac{2mr^2}{\hbar^2} E\right] R_\ell = \ell (\ell +1) R_\ell 
\] \vspace{3px}


\subsection{\textit{s-wave} case}

We first consider the $\ell = 0$ case, which we call the  \textit{s-wave} case.
(Hint: atomic $s$ orbitals). 

Substituting $u_\ell (r) = r R_\ell (r)$ in the above yields 

\[
\frac{d }{d r} r^2 \frac{d }{d r} \frac{u_\ell (r)}{r} = r \frac{d^2 }{d r^2}
u_\ell (r) \quad \Rightarrow \quad \frac{d^2 }{d r^2} u_\ell (r)
= -\frac{2mE}{\hbar^2}u_\ell (r) = -k^2 u_\ell (r) \quad k \equiv
\frac{\sqrt{2mE}}{\hbar}
\] \vspace{3px}

The general solution is 

\[
u(r) = A\sin kr + B\cos kr \qquad \Rightarrow R(r) = A \frac{\sin kr}{r}
+ B\frac{\cos kr}{r}
\] \vspace{3px}



We have boundary conditions at $r = 0$  and $r = a$. As the second term blows
up at $r=0$, we reject it as unphysical since $|R(r)|^2$ is infinite at the
origin.  The second constraint yields 

\[
\frac{\sin ka}{a} = 0 \quad \Rightarrow \quad ka = n\pi, \; n = 1,2,3,\hdots
\quad \Rightarrow \quad E_n = \frac{n^2\pi^2\hbar^2}{2ma^2}, \;
n = 1,2,3,\hdots
\] \vspace{3px}

The normalization condition is 

\[
1 = \int_{0}^{a} r^2|A|^2 \frac{\sin^2kr}{r^2} \, dr = |A|^2 \left( \frac{a}{2}
  - \frac{\sin 2ak}{4k}\right) = |A|^2 \left( \frac{a}{2} - \frac{\sin
2n\pi}{4k} \right) = |A|^2 \frac{a}{2} \quad \Rightarrow A = \sqrt{\frac{2}{a}}  
\] \vspace{3px}

\begin{mainbox}{$s$-wave Spherical Well Solution}
  \[ \ell =0: \quad \psi_{n, \ell = 0, m = 0}(\vec{r})
    = \sqrt{\frac{2}{a}}\frac{\sin\left( \frac{n\pi r}{a} \right)}{r} Y_{00}
    (\theta, \phi) = \frac{1}{\sqrt{2\pi a}} \frac{\sin \left( \frac{n\pi r}{a}
  \right)  }{r} \quad E_{n, \ell =0} = \frac{n^2\pi^2 \pi^2}{2ma^2} \]
\end{mainbox}

\subsection{General Case}

The full radial equation is 

\[
  \left[ \frac{d }{d r} \left( r^2 \frac{d }{d r}  \right) + k^2r^2 \right]
  R_\ell (r) = \ell (\ell +1) R_\ell (r) 
\] \vspace{3px}

the general solution of which are the spherical Bessel and spherical Neumann
functions

\[
  R_{n\ell }(r) = Aj_\ell (kr) + B\eta_\ell (kr)
\] \vspace{3px}

where $j_\ell $ is the solution regular at the origin 

\[
j_\ell (x) \rightarrow \frac{x^{\ell }}{(2\ell +1)!!} \qquad \eta_\ell (x)
\rightarrow - \frac{(2\ell -1)!!}{x^{\ell +1}}
\] \vspace{3px}

and thus is the solution we retain 

\[
j_0(x) = \frac{\sin x}{x} \quad j_1(x) = \frac{\sin x}{x^2} - \frac{\cos x}{x}
\qquad j_{\ell +1} (x) = \frac{2\ell +1}{x}j_\ell (x) - j_{\ell - 1}(x) 
\] \vspace{3px}

Our boundary condition at $r =a$ requires $j_\ell (kr) = 0$. Denoting the
ascending zeros, which one can find tabulated below, of the spherical Bessel
function by 

\[
  j_\ell (\beta_{n\ell})\equiv 0, \; ; n = 1,2,3\hdots \quad \Rightarrow ka
  = \beta_{n\ell }
\] \vspace{3px}

we then have the energy eigenvalues, 

\[
  E_{n\ell } = \frac{\hbar^2 k^2}{2m} = \frac{\hbar^2 \beta_{n \ell
  }^2}{2ma^2}, \quad  n = 1,2,3,\hdots \;\;\;\ell = 0,1, 2, \hdots
\] \vspace{3px}




\begin{figure}[H]
  \centering
    \includegraphics[width = 11cm]{bessel.pdf}
    \caption{First few $j_\ell (x)$}
\end{figure}

The first few zeros $\beta_{n\ell}$


\[ \text{ table of $\beta_{n \ell}$: } \qquad  
\begin{array}{c|c c c c}
 & n=1 & n=2 & n=3 & n=4 \\
\hline
\ell = 0 & \pi & 2\pi & 3\pi & 4\pi \\
\ell = 1 & 4.493 & 7.725 & 10.904 & 14.066 \\
\ell = 2 & 5.763 & 9.095 & 12.323 & 15.515 \\
\ell = 3 & 6.988 & 10.417 & 13.698 & 16.924 \\
\ell = 4 & 8.183 & 11.705 & 15.050 & 18.301 \\
\end{array}
\]




As the spherical harmonics are properly normalized, the normalization condition
is 

\[
  |A_{n\ell}|^2 = \int_{0}^{a} r^2 \left[ j_\ell (\beta_{n\ell
  } \frac{r}{a}\right]^2\, dr = |A_{n\ell }|^2 a^3 \int_{0}^{1} y^2 [j_\ell
  (\beta_{n\ell }y ]^2  \, dy = 1
\] \vspace{3px}

where we substituted $r = ay$ above. Thus our full solution is 

\begin{subbox}{Full Spherical Well Solution}
  \[ \psi_{n \ell  m} (r,\theta,\phi) = A_{n \ell } j_\ell \left(\beta{n \ell
    } \frac{r}{a}\right) Y_{\ell m} (\theta, \phi) \qquad E_{n \ell
} = \frac{\hbar^2\beta_{n \ell }^2}{2ma^2}\]
\end{subbox}


These solutions form a complete orthonormal basis of stationary states for any
wave packet satisfying the boundary condition that the wave function vanish at
$r=a$. The orthonormality condition 

\[
  1 = \int \psi^*_{n' \ell ' m'} (r, \theta, \phi) \psi_{n\ell m}(r, \theta,
  \phi) ,\ d\vec{r} = A^*_{n'\ell '} A_{n\ell } \int_{0}^{a} r^2 j_{\ell'}
  \left(\beta_{n'\ell } \frac{r}{a}\right) j_\ell \left( \beta_{n\ell
  } \frac{r}{a} \right)    \, dr \int Y_{\ell ' m'} (\theta, \phi) Y_{\ell m}
(\theta, \phi) \, d\Omega \] 

\[
  = \delta_{\ell' \ell} \delta_{m' m} A_{n' \ell }^* A_{n \ell } \int_{0}^{a}
  r^2 j_\ell \left( \beta_{n' \ell' } \frac{r}{a} \right) j_\ell \left(
  \beta_{n\ell } \frac{r}{a} \right)   \, dr
\] \vspace{3px}

This then requires 

\[
  A^*_{n' \ell} A_{n \ell } \int_{0}^{a} r^2 j_\ell \left( \beta_{n'\ell
  } \frac{r}{a} \right) j_\ell \left(\beta_{n\ell } \right)  \, dr
  = A^*_{n'\ell } A_{n\ell } a^3 \int_{0}^{1} y^2 j_\ell (\beta_{n' \ell} y)
  j_\ell (\beta_{n\ell } y) \, dy = \delta_{n'n} |A_{n\ell }|^2 
\] \vspace{3px}

which can be demonstrated by using recursion relations and partially
integrating. 

\begin{figure}[H]
  \centering
    \includegraphics[width = 12cm]{spherewell.pdf}
    \caption{The normalized radial solutions for $\ell = 2$, $A_{n\ell} j_\ell
    \left(\beta_{n\ell } \frac{r}{a}\right)$}
\end{figure}



It is amusing to compare the two closely related 3D problems we have completed.
In the case of the 3D Cartesian square well, there are three quantum numbers
$n_x, n_y, n_z$ associated with the three degrees of freedom, and $E = E (n_x,
n_y, n_z)$.  $E$ was proportional to $n_x^2 + n_y^2+ n_z^2$ so there is an
implicit energy degeneracy as several choices of these quantum numbers can
give the same energy. This degeneracy includes all distinct permutations of
$\{n_x, n_y, n_z\}$, but is not necessarily limited to such exchanges. These
permutations correspond to various 90-degree rotations of the cube into itself,
exchanging the axes. 

In the spherical case, there are also three quantum
numbers,  $n, \ell , m$, but $E = E_{n \ell }$. The degeneracy is explicit -- the
$2\ell +1$ degeneracy is associated with $m$. This reflects rotational
invariance -- the physics of a rotationally invariant Hamiltonian cannot
depend on the choice we make in locating the $z$-axis of our coordinate system. 

\section{The Hydrogen Atom}


Now we tackle a problem that in some sense started quantum mechanics. A couple
of preliminaries to simplify assumptions that Griffiths and/or we will make in
treating the hydrogen atom. 

\begin{itemize}
  \item[1.] To date we have discussed motions of particles in a fixed external
    field. However the hydrogen atom is more complex, consisting of an electron
    and a proton bound together, each orbiting the center of mass. However as
    $m_p \gg m_e$, the  center of mass is \textit{very} close to the proton,
    which allows Griffiths to treat the proton as an infinitely heavy static
    source of a Coulomb field. For consistency, we will do the same. However
    the true time-independent Schr\"odinger equation would be 

    \[
      \left[ -\frac{\hbar^2}{2m_e} \nabla_{\vec{r}_e}^2
        - \frac{\hbar^2}{2m_p}\nabla_{\vec{r}_p}^2 + V(|\vec{r}_e - \vec{r}_p
      |)\right] \psi(\vec{r}_e, \vec{r}_p) = E\psi(\vec{r}_e, \vec{r}_p)
    \] \vspace{3px}
    
    where here 

    \[
      \nabla_{\vec{r}_e}^2 = \frac{\partial^2 }{\partial x_e^2}
      + \frac{\partial^2 }{\partial y_e^2} + \frac{\partial^2 }{\partial z_e^2} 
    \] \vspace{3px}
    
    and similarly with $\nabla_{\vec{r}_p}^2$. The equation requires six
    coordinate degrees of freedom. But because the potential involves only the
    relative coordinate, one can simplify things by transforming the relative
    $\vec{r}_e $ and center of mass $\vec{R}$ coordinates as follows: 


    \[
    \vec{r} \equiv \vec{r}_e - \vec{r}_p \quad M\vec{R} = m_e \vec{r}_e + m_p
    \vec{r}_p \quad M \equiv m_e + m_p
    \] \vspace{3px}
    
    Using the chain rule one then finds 

    \[
    \left[ -\frac{\hbar^2}{2\mu} \nabla_{\vec{r}}^2 - \frac{\hbar^2}{2M}
    \nabla_{\vec{R}}^2 + V(r) \right] \psi(\vec{r}, \vec{R})
    = E\psi(\vec{r},\vec{R}) \quad \text{ where } \mu \equiv \frac{m_e m_p}{m_e
    + m_p}
    \] \vspace{3px}
    
    
    Then by writing $\psi(\vec{r}, \vec{R}) = \psi_\text{rel} (\vec{r})
    \psi_{CM} (\vec{R})$ one obtains 

    \[
      \left[ -\frac{\hbar^2}{2\mu} \nabla_{\vec{r}}^2 + V(r) \right]
      \psi_\text{rel} (\vec{r}) = E_\text{rel} \psi_\text{rel} (\vec{r})
      - \frac{\hbar^2}{2M} \nabla_{\vec{R}}^2 \psi_{CM} (\vec{R}) = E_{CM}
      \psi_{CM} (\vec{R}) \quad E = E_\text{rel} + E_{CM}    
    \] \vspace{3px}
    
    
    The second equation is easily solved. The center of mass of the atom
    travels as a plane and $E_{CM}$ is the associated kinetic energy. But in
    general this is of no interest -- we are instead concerned with the
    intrinsic excitations of the atom. 

  \item[2.] We will assume $V(r) \sim \frac{1}{r}$. The proton has a finite
    size, which modifies the $\frac{1}{r}$ behavior at very short distances of
    around $10^{-5}$ of the hydrogen atom's radius. Treating the Coulomb
    interaction as that from a point charge in the Schr\"odinger equation leads
    to solutions where the wave function remains finite at the origin -- but
    this is not the case for its relativistic analog, the Dirac equation. 

  \item[3.] Griffiths uses SI units. We will write the attractive Coulomb
    potential in the form 

    \[
      V(r)_\text{Griffiths} = -\left[ \frac{e^2}{4\pi\varepsilon_0}\right]
      \frac{1}{r} =-\left[ \frac{e^2}{4\pi\varepsilon_0\hbar c}\right]
      \frac{\hbar c}{r} = -\alpha \frac{\hbar c}{r} \equiv V(r)_\text{us}  
    \] \vspace{3px}
    
    
    Here $\alpha$ is the dimensionless fine structure constant $\alpha \sim
    \frac{1}{137}$ which Holger M\"uller has measured to exquisite precision.
    For atom physics applications, we use $\hbar c = 1973 \text{ eV \AA} $,
    where an \AA is $10^{-10} m$ as eV and \AA are the natural energy and
    distance units in an atom. 

\end{itemize}

    If we plug in $R_\ell (r) = u_\ell (r) / r$ into our generic 3D radial
    equation

    \begin{align} \label{}
      &\left[ -\frac{\hbar^2}{2m} \frac{1}{r^2} \frac{d }{d r} r^2 \frac{d }{d
        r} - \alpha \frac{\hbar c}{r} + \frac{\hbar^2}{2m} \frac{1}{r^2} \ell
        (\ell +1) \right] R_\ell = ER_\ell \quad \Rightarrow \\ 
      &\left[ -\frac{\hbar^2}{2m} \frac{d^2 }{d r^2} - \alpha \frac{\hbar c}{r}
      + \frac{\hbar^2}{2m}\frac{1}{r^2} \ell (\ell +1) \right] u_\ell
      = E u_\ell 
    \end{align}\vspace{3px}
 
The potential is shown below in Figure \ref{hydpotential}. It is much more
extended than either the square well or the harmonic oscillator


\begin{figure}[H]
  \centering
    \includegraphics[width = 11cm]{hydpotential.pdf}
    \caption{The Coulomb potential for the hydrogen atom, which generates
      an infinite number of bound and continuum states. The hydrogen Bohr
    radius $a_0$ is indicated. The $1s$ state binding energy is 13.6 eV.}
    \label{hydpotential}
\end{figure}

cases we have discussed previously, where binding energies were
proportional to $n^2$ and $n$ respectively. We will find the Coulomb
bound-state spectrum varies at $1 / n^2$ leading to an infinite number of
very weakly bound states. 

We look for bound-state solution. We introduce the dimensionless distance
$\rho$ and the dimensionless parameter $\rho_0$, 

\[
\rho \equiv \kappa r = \frac{\sqrt{-2m_e E}}{r} = \frac{\sqrt{2m|E|}}{r} \quad
\rho_0 \equiv \alpha \sqrt{\frac{2mc^2}{|E|}}
\] \vspace{3px}

where $|E|$ is the binding energy, which leads to 


\[
\left[ -\frac{\hbar^2}{2m} \frac{d^2}{dr^2} - \alpha\frac{\hbar c}{r}
+ \frac{\hbar^2}{2m} \frac{1}{r^2} \ell (\ell + 1) \right] u_{\ell}(\rho) = E u_{\ell}(\rho) 
\quad \Rightarrow \quad 
\]

\[
\left[ -\frac{\hbar^2}{2m} \frac{2m|E|}{\hbar^2} \frac{d^2}{d\rho^2} - \alpha \frac{\sqrt{2m|E|}}{\hbar}\frac{\hbar c}{\rho} + \frac{\hbar^2}{2m} \frac{2m|E|}{\hbar^2} \frac{1}{\rho^2} \ell (\ell + 1) \right] u_{\ell}(\rho) = -|E| u_{\ell}(\rho)
\quad \Rightarrow \quad 
\]

\[
\frac{d^2 u_{\ell}(\rho)}{d\rho^2} = \left[ 1 - \frac{\rho_0}{\rho} + \frac{\ell(\ell + 1)}{\rho^2} \right] u_{\ell}(\rho)
\]

We can get insight into the solutions by examining limiting behavior. As $\rho
\rightarrow \infty$, 

\[
\frac{d^2 u_\ell (\rho)}{d \rho^2} = \left[ 1 - \frac{\rho_0}{\rho}
+ \frac{\ell (\ell +1)}{\rho^2}\right] u_\ell (\rho) \quad \Rightarrow \quad
u_\ell (\rho) \sim Ae^{-\rho} + Be^{\rho} \quad \rightarrow \quad u_\ell (\rho)
\sim Ae^{-\rho} 
\] \vspace{3px}

as we want the solution to be normalizable. But for $\rho \rightarrow 0$, 

\[
\frac{d^2 u_\ell (\rho)}{d \rho^2} = \left[ 1 -\frac{\rho_0}{\rho} + \frac{\ell
(\ell + 1)}{\rho^2} \right] u_\ell (\rho) \quad \Rightarrow \quad \frac{d^2
u_\ell (\rho)}{d \rho^2} = \frac{\ell (\ell +1)}{\rho^2} u_\ell (\rho) \] 
\[ \Rightarrow \quad u_\ell (\rho) \sim C\rho^{\ell +1} + D\rho^{-\ell } \quad
\Rightarrow \quad u_\ell (\rho) \sim C\rho^{\ell +1}
\] \vspace{3px}

to avoid $R_\ell (r)$ blowing up as $r\rightarrow 0$. 

The combination of these two limits prompts us to try a solution of the form 

\[
  u_\ell (\rho) \sim \rho^{\ell +1} e^{-\rho} v_\ell (\rho) \quad \rightarrow 
\] \vspace{3px}


Plugging this in yields

\[ \rho^\ell e^{-\ell } \left[ \rho \frac{d^2 }{d \rho^2} + 2(\ell  + 1 - \rho)
\frac{d }{d \rho} + \frac{\ell (\ell +1)}{\rho} - 2(\ell +1) + \rho \right]
v_\ell (\rho) = \rho^\ell e^{-\rho} \left[ \rho - \rho_0 + \frac{\ell (\ell
+1)}{\rho} \right] v_\ell (\rho) \quad \Rightarrow \]

\[ \left[ \rho \frac{d^2 }{d \rho^2} + 2(\ell +1 - \rho) \frac{d }{d \rho} + \rho_0
- 2(\ell +1) \right] v_\ell (\rho) = 0 \] \vspace{3px}


Just as we did in solving the harmonic oscillator in the ``conventional way",
we look for a power series solution 

\[
v(\rho) = \sum_{j=0}^{\infty} c_j \rho^j
\] 
\[ \frac{d v(\rho)}{d \rho} = \textcolor{blue}{ \sum_{j=0}^{\infty} c_j
  j \rho^{j-1} } = \sum_{j=1}^{\infty} c_j j \rho^{j -1}
= \textcolor{red}{\sum_{j=0}^{\infty}c_{j+1} (j+1) \rho^j} \quad \text{ in the last
step taking }  j \rightarrow j+1 \]

\[ \frac{d^2 v(\rho)}{d \rho^2} = \sum_{j=0}^{} c_{j+1} j (j+1) \rho^{j-1}
    \] \vspace{3px}

Substituting this into our radial equation yields 

\[
  \sum_{j=0}^{\infty} \left[ c_{j+1} j(j+1) \rho^j + \textcolor{red}{ 2(\ell +1)
    c_{j+1} (j+1) \rho^j} \textcolor{blue}{- 2c_j j \rho^j } + (\rho_0 - 2(\ell
  +1))c_j p^j \right] = 0
\] \vspace{3px}

where the color coding indicated which expressing for $ \frac{d v(\rho)}{d
\rho}$ has been used where. As all terms carry the same power in $\rho^j$, we
conclude 

\[c_{j+1} j(j+1) + 2(\ell +1) c_{j+1} (j+1) - 2c_j j + (\rho_0 - 2(\ell +1))c_j
= 0\quad \Rightarrow \] 
\[ c_{j+1} (j+1)(j+2\ell +2) - c_j (2j + 2\ell + 2 - \rho_0) = 0 \quad
  \Rightarrow \quad c_{j+1} = \left[ \frac{2(j+\ell +1) - \rho_0}{(j+1)(j+2\ell + 2)}
  \right] c_j \] \vspace{3px}


For large $j$ this relationship reduces to 

\[
  c_{j+1} \sim \frac{2}{j+1}c_j \quad \text{ which we can iterate to get
  } \quad c_{j+1}
  \sim \frac{2^{j+1}}{(j+1)!}c_0
\] \vspace{3px}

Thus, 

\[
v(\rho) = \sum_{j=0}^{\infty} c_j \rho^j \sim c_0 \sum_{j=0}^{\infty}
\frac{(2\rho)^j}{j!} \sim c_0 e^{2\rho}
\] \vspace{3px}

in which case 

\[
  u_\ell (\rho) \sim \rho^{\ell +1} e^{-\rho} v_\ell (\rho) \sim \rho^{\ell +1}
  e^{-\rho} c_0 e^{2\rho} \sim c_0 \rho^{\ell +1} e^{\rho}
\] \vspace{3px}

and our solution would not be normalizable. We conclude that $v(\rho)$ must
truncate -- it must be a polynomial. Thus our solutions must be 

\begin{subbox}{}
  Bound-state Coulomb solutions will be obtained for $c_j = 0, j = 1, 2, 3,
  \hdots$
\end{subbox}

Let's start finding out what these polynomials are. 


\subsection{The Hydrogen Atom: Bound State Radial Wave Function}

Let's quickly summarize what we just found. If we plug in $R_\ell (r) = u_\ell
(r) / r$ in our generic 3D radial wave equation 

\[\left[ -\frac{\hbar^2}{2m} \frac{1}{r^2} \frac{d }{d r} r^2 \frac{d }{d r} - \alpha
\frac{\hbar c}{r} + \frac{\hbar^2}{2m} \frac{1}{r^2} \ell (\ell +1) \right]
R_\ell = ER_\ell  \quad \Rightarrow \] 
\[\left[ -\frac{\hbar^2}{2m}  \frac{d^2 }{d r^2} - \alpha \frac{\hbar c}{r}
  + \frac{\hbar^2}{2m} \frac{1}{r^2} \ell (\ell +1) \right] u_\ell = Eu_\ell
\] \vspace{3px}


We look for bound-state solutions. We introduce the dimensionless $\rho$ and
the dimensionless parameter $\rho_0$, 

\[
\rho \equiv \kappa r = \frac{\sqrt{-2m_e E}}{\hbar}r
= \frac{\sqrt{2m|E|}}{\hbar}r \qquad \rho_0 \equiv \alpha
\sqrt{\frac{2mc^2}{|E|}}
\] \vspace{3px}

where $|E|$ is the binding energy, which leads to 

\[
\frac{d^2 u_\ell (\rho)}{d \rho^2} = \left[ 1 - \frac{\rho_0}{\rho}
+ \frac{\ell (\ell +1)}{\rho^2}\right] u_\ell (\rho)
\] \vspace{3px}

We examined this equation for $\rho \rightarrow \infty$ and found that it must
behave as 

\[
  u_\ell (\rho) \sim Ae^{-\rho}
\] \vspace{3px}

to be normalizable. We also examined the equation for $\rho \rightarrow 0$ and
found that to remain infinite, 

\[
  u_\ell (\rho) \sim C\rho^{\ell +1}
\] \vspace{3px}

to avoid $R_\ell (r)$ blowing up as $r\rightarrow 0$. 

We then decided to try a solution that builds in the proper large and small
$\rho$ behavior 

\[
  u_\ell (\rho) \sim \rho^{\ell +1} e^{-\rho} v_\ell (\rho) \quad \Rightarrow
\] \vspace{3px}

where $v_\ell (\rho)$ is yet to be determined. Plugging this in and doing
a far amount of algebra led to the recursion relation 

\[
  c_{j+1} = \left[ \frac{2(j+\ell +1) - \rho_0}{(j+1)(j+2\ell +2)} \right] c_j
\] \vspace{3px}

For large $j$ this relationship reduces to 

\[
  c_{j+1} \sim \frac{2}{j+1}c_j \quad \Rightarrow \quad c_{j+1} \sim
  \frac{2^{j+1}}{(j+1)!}c_0
\] \vspace{3px}

Thus, 

\[
v(\rho) = \sum_{j=0}^{\infty} c_j \rho^j \sim c_0 \sum_{j=0}^{\infty}
\frac{(2\rho)^j}{j!} \sim c_0 e^{2\rho}
\] \vspace{3px}
in which case 

\[
  u_\ell (\rho) \sim \rho^{\ell +1} e^{-\rho} v_\ell (\rho) \sim \rho^{\ell +1}
  e^{-\rho} c_0 e^{2\rho} \sim c_0 \rho^{\ell +1} e^\rho
\] \vspace{3px}

and our solution would not be normalizable. Therefore $v(\rho)$ must truncate
as a polynomial. 

If $c_{jmax}$ is the last nonzero coefficient, then 

\[
  2(j_{max} + \ell +1) = \rho_0, \quad j_{max} = 0, 1, 2, \hdots
\] \vspace{3px}

This is an \textit{eigenvalue equation}: $\rho$ depends on the energy and must
be an even positive integer (see below). Let's define $n \equiv j_{max} +  \ell
+1$ -- which we will call the principal quantum number. As $j_{max} $ runs from
0 onward, 

\[
\ell =0 \; \Rightarrow \; n = 1,2,3,\hdots \qquad \ell =1 \; \Rightarrow \;
n = 2,3,4,\hdots \qquad \text{ in general } \; \ell \; n = \ell + 1, \ell +2,
\hdots
\] \vspace{3px}


Then the condition is 

\[
2n = \rho_0 = \alpha \sqrt{\frac{2mc^2}{|E|}} \quad \Rightarrow \quad |E_n|
= \frac{\alpha^2 mc^2}{2n^2} = -\frac{|E_1|}{n^2} \quad \text{depends on just}
n
\] \vspace{3px}

\begin{subbox}{}
  \begin{align} \label{}
    &n = 1 &&|E_1| = \frac{\alpha^2mc^2}{2} &&&\ell =0 &&&& 1s \\ 
    &n = 2 &&|E_2| = \frac{|E_1|}{4} &&&\ell = 0, 1 &&&& 2s, 2p \\
    &n = 3 &&|E_3| = \frac{E_1|}{9} &&&\ell = 0, 1, 2 &&&& 3s, 3p, 3d
  \end{align}\vspace{3px}
\end{subbox}




As each $\ell $ has $2m+1$ \textit{magnetic} substrates -- we'll get to the
magnetic aspect later -- we can calculate the total degeneracy of the level
characterized by $ n$, 

\[
\text{number of states with energy} E_n \; : \; \sum_{\ell =0}^{n-1} (2\ell +1)
= n^2
\] \vspace{3px}

There is a natural distance scale for the hydrogen atom, based on the
n = 1 $s$-wave orbit 

\[
1 = a_0\kappa_1 \equiv a_0 \frac{\sqrt{2m|E_1|}}{\hbar} = a_0 \frac{\alpha
  mc^2}{\hbar c} \qquad a_0 = \frac{\hbar c}{\alpha mc^2} \sim \frac{(137)
(1973 \text{ eV \AA} ) }{511000 \text{ eV} } \sim 0.529 \text{\AA} 
\] \vspace{3px}

This is called the \textit{Bohr radius} for hydrogen. Similarly $|E_1| \sim
13.6 \text{ eV} $. 

\begin{mainbox}{Basic Scales in Hydrogen}
  1 $s$ binding energy $\sim 13.6$ eV $\qquad$ Bohr Radius  $\sim 0.529 $ \AA
  $\quad $1 \AA = $10^{-10}$ m
\end{mainbox}

Note then that 

\[
\rho = \kappa r = \frac{\sqrt{2m|E_n|}}{\hbar}
r = \frac{\sqrt{2m|E_1|}}{n\hbar} r = \frac{1}{n} \frac{r}{a_0}
\] \vspace{3px}

So the spectroscopy of the hydrogen atom -- at the level where the interaction
is just the Coulomb potential is shown below in Figure \ref{hydspectra}. Note
that, apart from the ground state, there are multiple states of different
\textit{angular momenta} $\ell $ with the same energy. This degeneracy should surprise
you, as the radial equation whose solutions determine the energy eigenvalues
include an angular momentum barrier term $\ell (\ell +1) / r^2$ that one might
anticipate would distinguish eigenstates of different $\ell $. Also note the
pattern of bound states, the increasing density of states as one approaches $E
\rightarrow 0$ from below. 

Also remember there are continuum states of $E > 0$ that we have not discussed.
These are scattering states of an interacting free electron with an interacting
free proton. These states are not normalizable, but we could derive a basis,
analogous to plane waves, that (together with the bound states) would be
complete, allowing us to expand any wave packet describing a $E>0$ state of an
interacting electron and proton in terms of that basis. This basis would become
our familiar plane wave basis -- our momentum basis -- were we to turn off the
Coulomb interaction, by, for example, taking $\alpha \rightarrow 0$. 



\vspace{5px}

\begin{figure}[H]
  \centering
    \includegraphics[width = 13cm]{hydspectra.pdf}
    \caption{The spectroscopy of the hydrogen atom. Our treatment of the
      Hamiltonian includes just the Coulomb force between the proton and
      electron, treating the proton as a point particle. Corrections associated
      with the electron and nuclear spin -- fine structure and hyperfine
    structure can be added to this simple picture.}
    \label{hydspectra}
\end{figure}

\vspace{5px}




\subsubsection{\textit{s-wave} Solutions}

We can generate a few examples of $s$-wave radial states 

\[
  s \text{--wave:} \quad c_{j+1} = \left[ \frac{2(j+1)
  - 2n}{(j+1)(j+2)}\right] c_j
\] \vspace{3px}

and calling the $j_{max}$ as the last nonzero term $c_{jmax}$, 

\[ j_{max} = 0 \qquad n = 1 \qquad v(\rho) = c_0 \] 
\[ j_{max} = 1 \qquad n= 2 \qquad v(\rho) = c_0(1 - \rho) = c_0 \left(1
- \frac{1}{2}\frac{r}{a_0}\right) \] 
\[ j_{max} = 2 \qquad n = 3 \qquad v(\rho) = c_0 \left( 1 - 2\rho + \frac{2}{3}
    \rho^2\right) = c_0 \left( 1 - \frac{2}{3} \frac{r}{a_0} + \frac{2}{27}
  \left( \frac{r}{a_0} \right)^2  \right)  \] \vspace{3px}

and so on. Now as 

\[
  R_\ell (r) = \frac{1}{r} \rho^{\ell +1} e^{-\rho} v_\ell (\rho) \qquad
  \Rightarrow \qquad R_0(r) = \frac{1}{r} \rho e^{-\rho} v_0(\rho)
\] \vspace{3px}

So 

\[
  R_{n=1\ell =0} \sim c_0 e^{-r / a_0} \quad R_{n=2\ell =0} \sim c_0e^{-r/2a_0}
  \left( 1 - \frac{1}{2} \frac{r}{a_0} \right) \qquad R_{n=3\ell =0} \sim c_0
  e^{-r / 3a_0 } \left( 1-\frac{2}{3}\frac{r}{a_0} + \frac{2}{27}\left(
  \frac{r}{a_0} \right) ^2  \right)  
\] \vspace{3px}

Finally we determine $c_0$ via normalization in each case above 

\[
  \int_{0}^{\infty} r^2 |R_{n\ell =0}(r)|^2 \, dr = 1
\] \vspace{3px}

which yields 

\[ R_{n=1\ell =0} = \frac{2}{\sqrt{a_0^3}}e^{-r / a_0}  \qquad \qquad R_{n=2
  \ell =0} = \frac{2}{\sqrt{(2a_0)^3}}e^{-r / 2a_0} \left( 1 - \frac{1}{2}
\frac{r}{a_0} \right) \] 
\[ R_{n=3 \ell = 0} = \frac{2}{\sqrt{(3a_0)^3}}e^{-r / 3a_0} \left(
  1 - \frac{2}{3} \frac{r}{a_0} + \frac{2}{27} \left( \frac{r}{a_0} \right)^2
\right)  \] \vspace{3px}


and of course the full 3D normalized stationary bound states are 

\[ R_{n=1\ell =0} = \frac{2}{\sqrt{a_0^3}}e^{-r / a_0} Y_{00} (\theta, \phi) \qquad \qquad R_{n=2
  \ell =0} = \frac{2}{\sqrt{(2a_0)^3}}e^{-r / 2a_0} \left( 1 - \frac{1}{2}
\frac{r}{a_0} \right)Y_{00} (\theta, \phi) \] 
\[ R_{n=3 \ell = 0} = \frac{2}{\sqrt{(3a_0)^3}}e^{-r / 3a_0} \left(
  1 - \frac{2}{3} \frac{r}{a_0} + \frac{2}{27} \left( \frac{r}{a_0} \right)^2
\right)Y_{00} (\theta, \phi)  \] \vspace{3px}



\subsubsection{General Solution}

The recursion for the polynomial $v(\rho)$ can be evaluated for any $\ell $ and
$n$. The results are the associated Laguerre polynomials 

\[
  v(\rho) = L_{n - \ell -1}^{2\ell +1} (2\rho) \quad \text{ where } \quad
  L_q^p(x) \equiv (-1)^p \left( \frac{d }{d x}  \right) ^p L_{p+q}(x) 
\] \vspace{3px}

Here $L_{p+q}(x)$ is the Laguerre polynomial, 

\[
  L_q(x) = \frac{e^x}{q!} \left( \frac{d }{d x}  \right) ^q (e^{-x} x^q) \qquad
  L_0(x) = 1 \quad L_1(x) = 1-x \quad L_2(x) = 1 - 2x + \frac{1}{2}x^2 + \cdots
\] \vspace{3px}

Combining these and bringing derivatives across the first exponential yields
simpler expressions

\[
  L_q^p(x) = \frac{x^{-p} e^x}{q!} \left( \frac{d }{d x}  \right) ^q e^x
x^{p+q} \]  

\[ v(\rho) = \frac{1}{(n - \ell - 1)!} \rho^{-2\ell - 1} e^{2\rho} \left(
\frac{d }{d \rho}  \right) ^{n - \ell - 1} e^{-2\rho} \rho^{n + \ell } \]
\vspace{3px}

The normalization can be done, and the full 3D stationary states are formed.
The result is 

\begin{subbox}{Hydrogen Bound-State Wave Functions}
  \[ \psi_{n\ell m} = \sqrt{ \left( \frac{2}{na_0} \right) ^3 \frac{(n - \ell
    - 1)!}{2n(n+\ell)! } }e^{-r / na_0} \left( \frac{2r}{na_0} \right) ^\ell
    \left[ L_{n - \ell -1}^{2\ell +1} (2r / na_0) \right] Y_{\ell m}(\theta,
    \phi) \] \[ \int r^2 \psi^*_{n\ell m}(\vec{r}) \psi_{n \ell m}(\vec{r}) \,
  dr d\Omega = \delta_{n'n}\delta_{\ell ' \ell } \delta{m'm} \] \vspace{3px}
\end{subbox}


A lot remains to be discussed. Our goal is now to understand the electron's quantum mechanical
\textit{spin}. 




