Here we will go through early developments that helped define attributes of
the theory that Schr\"odinger would later capture in his wave equation. These
observations were in addition to some already mentioned, such as the regular
but discrete patterns of lines in photoabsorption or photoemission processes of
atoms.

\section{The Stefan-Boltzmann Law}
\subsection{Rayleigh-Jeans}

A black body consists of a cavity in which electromagnetic standing waves in
the cavity interior have reached thermal equilibrium with the cavity walls. The
walls are perfect absorbers, absorbing all incident radiation regardless of
frequency, and perfect emitters, radiating energy isotropically in a spectrum
we discuss below. Physics has some remarkable examples of black bodies, with
the cosmic microwave background left over from the Big Bang one of the most
spectacular, as the spectral deviation from a perfect black body is on the
order of one part in a million (and extremely interesting from the perspective
of what they tell us about the structure of the infant universe). One can
envision probing the radiation by making a pin-prick in the cavity to monitor
the radiation. 

\begin{figure}[H]
  \centering
    \includegraphics[width = 4cm]{blackbox.pdf}
    \caption{A cubic cavity with walls held at temperature $T$, and with
      a volume $L^3$. The walls perfectly absorb all incident energy, then
      readmit that radiation, maintaining an equilibrium between the contained
    radiation and the walls characteristic of $T$} 
\end{figure}

The Stefan-Boltzmann Law was deduced experimentally in the late 19th century:
the power $P$ radiated per unit surface area $A$ of a black body (of course,
summed over all wavelengths) is 

\[
  P / A = \sigma T^4
\] \vspace{3px}

where $\sigma \sim 5.6703 \times 10^{-8}$ Watts/$m^2K^4$. This law nicely
reproduces observations on systems that range from heated filament in the
laboratory to the surface of stars. 
Rayleigh and Jeans attempted to derive this law -- and thus obtain an
expression for the Stefan-Boltzmann constant $\sigma$ -- from first principles,
by explicitly summing over the electromagnetic standing waves in a box. This
requires one to calculate the number of standing electromagnetic modes in the
box of volume $L^3$. Electromagnetic waves satisfy Laplace's equation which in
Cartesian coordinates is 

\[
\frac{\partial^2 \psi}{\partial x^2} + \frac{\partial^2 \psi}{\partial y^2}
+ \frac{\partial^2 \psi}{\partial z^2} + k^2\psi = 0
\] \vspace{3px}

One writes $k^2 = k_x^2 + k_y^2 + k_z^2$ and separates the equation into
a product of solutions in the $x, y, z$ directions, each of which must vanish
at the boundaries at $0$ and $L$. The solutions are 

\[
  \psi[n_x, n_y, n_z] = N\sin \left( \frac{\pi n_x}{L}x \right) \sin \left(
  \frac{\pi n_y}{L}y\right) \sin \left( \frac{\pi n_z}{L}z \right)
  \] \[ k^2 = \frac{\pi^2}{L^2}(n_x^2 + n_y^2 + n_z^2) \]



where $(n_x, n_y, n_z)$ are positive integers. We want to count how many modes
$(n_x, n_y, n_z)$ there are, and we can do that by switching to spherical
coordinates and integrating over $k$, while assuming a large volume. Taking
into account that $k_i / (\pi / L) = n_i$, in the large volume limit we have

\[
N(k)dk = \frac{1}{8} \times 2 \times \frac{4\pi k^2 \, dk}{(\pi / L)^3}
= \frac{Vk^2 \, dk}{\pi^2}
\] \vspace{3px}

where of course, $V = L^3$. The factor of $\frac{1}{8}$ is needed as we only
want the fraction of the sphere where all $n_i > 0$, and the factor of $2$ is
needed because each standing wave supports both transverse electric and
transverse magnetic projections. 
One can use a classical Boltzmann distribution to calculate the average energy
per cavity mode. One finds 

\[
  \bar{E} = \frac{\int_0^\infty E e^{-E / k_B T} \, dE}{\int_0^\infty e^{-E
  / k_B T} \, dE} = k_B T
\] \vspace{3px}
The denominator is to account for normalization. This result applies separately
to the light quanta of the same frequency in a black body cavity. We can then
fold this with the expression for the number of standing wave modes, derived
above, to get the energy density, 

\[
  \frac{E}{V} = k_B T \int \frac{k^2\, dk}{\pi^2} = k_B T \int \frac{8\pi \nu^2
  \, d\nu}{c^3} = k_B T \int \frac{8\pi \, d\lambda}{\lambda^4}
\] \vspace{3px}

where we have used the relationship between wave number, frequency, and
wavelength $k = \frac{2\pi \nu}{c} = \frac{2\pi}{\lambda}$ to write equivalent
formulas. 

This result neither reproduces observation nor the Stefan-Boltzmann law. The
calculation is not self-consistent, as the integrals diverge for large $k$ or
large $\nu$, alternatively small $\lambda$. 

\subsection{Planck's Revision}

In 1900 Planck revised the Boltzmann result for the energy per mode by
replacing the classical Boltzmann integral over energy-weighted modes by
a discrete sum corresponding to energy quantized as $E = n h \nu$, $n = 0, 1,
2, \cdots$, where  $h$ is a new physical constant. This modifies the
energy/mode calculation in the following way: 

\begin{align*}
  \bar{E} &= \frac{\sum_{n=0}^\infty nh\nu e^{-nh\nu / k_B
    T}}{\sum_{n=0}^\infty e^{-nh\nu / k_B T}} \\ &= - \frac{1}{\sum_{n=0}^\infty
  e^{-n h\nu / k_BT}} \frac{d}{d \frac{1}{k_B T}} \sum_{n=0}^\infty e^{-n h\nu
/ k_B T}
\end{align*} \vspace{3px}

As the sum remaining is geometric, it can be done. A bit of algebra yields 

\[
  \bar{E} = \frac{h\nu}{e^{h\nu / k_B T} - 1}
\] \vspace{3px}

and thus Planck obtained (in frequency form)


\begin{align*}
  \frac{E}{V} &= \int_0^\infty \rho_E(\nu) \, d\nu \\
  \rho_E(\nu) &= \frac{8\pi h}{c^3} \frac{\nu^3}{e^{h\nu / k_B T} - 1}
  \rightarrow \begin{cases}
    k_B T \frac{8\pi\nu^2}{c^3} &\, \frac{h\nu}{k_B T} \ll 1 \\
    \frac{8\pi h}{c^3}e^{-h\nu / k_B T} \nu^3 &\, \frac{h\nu}{k_BT}\gg 1
  \end{cases} 
\end{align*} \vspace{3px}

The Rayleigh-Jeans (classical) result is obtained for small frequencies, so we
recognize $h \rightarrow 0$ as the classical limit of Planck's black-body
formula. But for high frequencies the energy density as a function of frequency
is now well-behaved, diminishing exponentially, very unlike the classical case.
Below I plot a comparison between Rayleigh-Jeans and Planck's derivation as
a function of wavelength. 



\begin{center}
  \begin{tikzpicture}[scale = 0.95]
  \message{^^JBlack body}
  \def\N{60}
  \def\xmax{3100}
  \def\ymax{1.36e10}
  \def\tick#1#2{\draw[thick] (#1+.01*\ymax) -- (#1-.01*\ymax) node[below=-.5pt,scale=0.75] {#2};}
  \begin{axis}[
      every axis plot/.style={
        mark=none,samples=\N,domain=5:\xmax,smooth},
      xmin=(-.05*\xmax), xmax=(1.05*\xmax),
      ymin=(-.08*\ymax), ymax=(1.08*\ymax),
      restrict y to domain=0:\ymax,
      axis lines=middle,
      axis line style=thick,
      %enlargelimits=upper, % extend the axes a bit to the right and top
      tick style={black,thick},
      ticklabel style={scale=0.8},
      %xtick style={draw=none},xticklabels=none,
      max space between ticks=26,
      xlabel={Wavelength $\lambda$ [nm]},
      ylabel={Power $P$ [kW/sr\,m$^2$\,nm]},
      xlabel style={at={(rel axis cs:0.5,0)},below=-1pt,font=\small},
      ylabel style={at={(rel axis cs:-0.11,0.5)},rotate=90},
      width=9cm, height=7cm,
      %clip=false
      tick scale binop=\times,
      every y tick scale label/.style={at={(rel axis cs:0,1)},anchor=south}]
    ]
    
    % RAINBOW
    \shade[shading=rainbow,shading angle=90,opacity=0.5] (380,0) rectangle (740,\ymax);
    \node[above=-1pt,scale=0.8] at (200,\ymax) {\strut UV}; % 10 - 400 nm
    \node[above=-1pt,scale=0.8] at (570,\ymax) {\strut optical}; % 380 - 740 nm
    \node[above=-1pt,scale=0.8] at (920,\ymax) {\strut IR}; % 740 - 1050 nm
    
    % PLANCK
    \addplot[very thick,samples=3*\N,blue] {planck(x,5000)};
    %\addplot[dashed,thick,red,domain=1000:4000]    {rayleighjeans(x,3000)};
    %\addplot[dashed,thick,orange,domain=1000:4000] {rayleighjeans(x,4000)};
    \addplot[dashed,thick,blue,domain=1000:4000]   {rayleighjeans(x,5000)};
    %\addplot[dashed,thick,red,domain=1000:4000]    {wien(x,3000)};
    %\addplot[dashed,thick,orange,domain=1000:4000] {wien(x,4000)};
    %\addplot[dashed,thick,blue,domain=1000:4000]   {wien(x,5000)};
    
    %% MAXIMUM (Wien's displacement law)
    %\addplot[mygreen,very thin,variable=T,domain=2500:6000]
    %  ({lampeak(T)},{planck(lampeak(T),T)});
    
    % LABELS
    \node[above right=-1pt,scale=0.75,blue] at (800,{planck(800,5000)}) {\SI{5000}{K}};
    \node[above right=-1pt,scale=0.75,blue] at (1500,{rayleighjeans(1500,5000)}) {\SI{5000}{K} Rayleigh-Jeans};
    
    %% TICKS
    %\tick{500,0}{500}
    %\tick{1000,0}{1000}
    %\tick{1500,0}{1500}
    %\tick{2000,0}{2000}
    %\tick{2500,0}{2500}
    %\tick{3000,0}{3000}
    
  \end{axis}
\end{tikzpicture}
\end{center}

Rayleigh-Jeans' prediction diverges as the wavelength of light becomes small
which goes against energy conservation. Planck's revision, however, viewing light as discrete
packets of energy known as \textit{quanta}, recovers the Stefan-Boltzmann law
and is well-behaved.  


\section{The Photoelectric Effect}

At about the same period when the issues with black body radiation were
confusing physicist, experimentalists were examining the emission of electrons
from a metal surface when UV light was focused on the surface.

\begin{figure}[H]
  \centering
    \includegraphics[width = 6cm]{photoelectric.pdf}
    \caption{Basics of a photoelectric effect experimental setup}
\end{figure}

Results from such experiments produced the following phenomenology: 

\begin{itemize}
  \item[1.] The number but not the energy of the photoelectrons depends on the
    light intensity; 
  \item[2.] Photoelectrons appear as soon as the light is turned on (within few
    nanoseconds), even when the light intensity is low;
  \item[3.] Photoelectron energy depends on the frequency of light, with
    a faint blue light (higher frequency) producing more energetic electrons
    than an intense red light (lower frequency). If the frequency of light is
    too low, no emission is seen.
\end{itemize}

These results are unexpected in the classical picture of light as a wave. And
at the time these experiments were done, there were many verifications of the
wave nature of light. In particular, energy \textit{from a wave} would be absorbed
across the metal surface, so that to knock out an electron, one would have to
wait until the area immediately around the electron had absorbed enough energy
to make that possible. If the frequency of the `wave' light were increased, with all
other parameters kept fixed, the necessary period might shorten, but when the
threshold for emission is reached, the electrons emitted would be similar in
energy to those produced with lower frequency light. 

Einstein in 1909 resolved this problem by proposing a wave-particle duality --
that light sometimes acts as a wave, and other times as a photon. Following up
on Planck, Einstein argued that the photoelectric effect observations were
consistent with a ballistic process in which individual quanta of light of
energy $h\nu$ were responsible for knocking out individual electrons from the
metal. Energy conservation then yields 

\begin{align} \label{eq:energy}
h\nu = KE_e + h\nu_0
\end{align} \vspace{3px}

Here $h\nu_0$ is the energy required to remove an electron from the metal --
the work function, which is a property of the specific metal being used -- and
consequently no photoelectrons are produced if the frequency of light $v < v_0$
In this picture, provided $\nu > \nu_0$, photoelectrons are expected
immediately on illumination, as each photon has the ability to dislodge an
electron. If the frequency of the light is increased, the energy of the
photoelectron increases linearly. If the frequency is held fixed but the
intensity is doubled, the photon flux and the number of photoelectron-producing
collisions doubles. 

This explanation is simple, yet seemed to contradict years of study of light
ways interfering and undergoing diffraction -- a wave phenomena. Thus the
intellectual leap was the hypothesis of the wave-particle duality of light --
that different aspects of light could be manifested in different experimental
settings. 

\section{de Broglie \& the Bohr atom} 

By the early 1900s, Rutherford had established that atoms had a dense nuclear
core and many experiments had been done observing the absorption and emission
of visible and other light from simple atoms, including hydrogen. As the
binding energy in hydrogen (neglecting fine structure) is $-13.6\, eV/n^2$, where
the principle quantum number takes on integer values $n = 1, 2, 3, \cdots,$ the
emission lines correspond to energies 

\[
  E_{n_i} - E_{n_f} = 13.6 \,eV \, \left(\frac{1}{n_f^2} - \frac{1}{n_i^2}
  \right), \; n_f < n_i
\] \vspace{3px}
The quantum number corresponds to the orbit an electron follows. The emission
lines correspond to specific wavelengths of emitted radiation as a result of an
electron falling from a higher energy level (higher quantum number $n_i$) to
a lower energy level (lower quantum number $n_f$). Various ``series" had been identified
\begin{align*}
  &\text{Balmer 1885 (visible): }  &&\; \Delta E = 13.6 \, eV \, \left(
  \frac{1}{2^2} - \frac{1}{n_i^2} \right) \\
  &\text{Lyman 1906 - 14 (UV): } &&\; \Delta E = 13.6 \, eV \, \left(
  \frac{1}{1^2} - \frac{1}{n_i^2} \right) \\
  &\text{Paschen 1908 (IR): } &&\; \Delta E = 13.6 \, eV \, \left(
  \frac{1}{3^2} - \frac{1}{n_i^2} \right) 
\end{align*}

While Rutherford had proposed a model of atoms as electrons orbiting and bound
to a nucleus, Bohr attempted to relate this idea to the emerging notion of
quantization in a 1913 model. He recognized that the phenomenology above could
be reproduced by a classical model of electrons in circular orbits about the
nucleus, where 

\[
|\vec{v} \times \vec{p}| = mvr = \frac{nh}{2\pi} = n\hbar
\] \vspace{3px}

where we have introduced the reduced Planck's constant $\hbar$ -- which
everyone calls ``h bar." This is the constant we will be using in quantum
mechanics \textit{much} more often. If one accepts Bohr's hypothesis as
a constraint and computes the energies, indeed one reproduces the emission
results above. 

The model also got two important matters right: 

\begin{itemize}
  \item[1.] Atomic systems can exist only in certain stationary or quantized
    states, each characterized by a definite energy; 
  \item[2.] Transitions between such states can occur via emission or
    absorption of radiation with energy $\Delta E= h\nu$, in agreement with how
    both Planck and Einstein treated radiation. 
\end{itemize}

But there as many unanswered questions as answered ones: 
\begin{itemize}
  \item[1.] Why are the stationary states stationary? Since a classical
    electron in circular motion radiates, the electron should lose energy,
    spiraling into the nucleus; 
  \item[2.] From our modern perspective of quantum mechanics, a classical orbit
    with a definite radius violates the uncertainty principle
\end{itemize}

In his 1924 thesis, de Broglie offered a possible explanation of the Bohr atom
that anticipated the quantum mechanics revolution about to overtake physics.
The details of how his suggestion supported the Bohr atom is not critical --
the idea behind it is. Noting that Einstein and Planck had treated
electromagnetic waves as particles, de Broglie suggested that perhaps
\textit{particles} (the electron in this case) sometimes behave as
\textit{waves}. This is quantum mechanics. For a photon, 
\[
p_\gamma = \frac{h\nu}{c} = \frac{h}{\lambda}
\] 
so perhaps a massive particle satisfies the same relationship:
\begin{align} \label{eq:deBroglie}
p_e = m_e v = \frac{h}{\lambda} \quad \Rightarrow \quad \lambda = \frac{h}{m_e v}
\end{align}
Where $\lambda$ in this case, is the particle's de Broglie wavelength. Yes,
particles have a wavelength. 

If one calculates the de Broglie wavelength of an electron moving at $v/c \sim
0.01$, one finds $\lambda \sim 2$ angstroms -- so about the circumference of an
  atom.

De Broglie was able to account for the Bohr model by assuming that electronic
orbits in hydrogen correspond to an integer number of de Broglie wavelengths. 

\begin{figure}[H]
  \centering
    \includegraphics[width = 6cm]{debroglie.pdf}
    \caption{de Broglie proposed that atomic orbits correspond to an integer
      number of de Broglie wavelengths. The figure shows a slight mismatch at
      the top which means that the radius should be adjusted to remove this
    discontinuity.} 
\end{figure}

\onecolumn
